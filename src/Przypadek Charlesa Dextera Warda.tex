% Teraz literówki, rozdział 2.5 !!!

\part{Powieść}

\chapter{Przypadek Charlesa Dextera Warda}
\index{Dzieła!Przypadek Charlesa Dextera Warda}
\index{Osoby!Charles Dexter Ward}
\index{Osoby!Dr. Willett}
\index{Osoby!Dr. Waite}
\index{Osoby! Dr. Lyman}
\index{Osoby!Joseph Curwen}
\index{Osoby!Jabez Bowen}
\index{Osoby!Dr. Checkley}
\index{Osoby!John Merritt}
\index{Osoby!Dutie Tillinghast}
\index{Osoby!Eliza Tillinghast}
\index{Osoby!Ezra Weeden}
\index{Osoby!Samuel Winson}
\index{Osoby!John Graves}
\index{Osoby!Cosmo Alexander}
\index{Osoby!Stephen Hopkins}
\index{Osoby!Joseph Brown}
\index{Osoby!Benjamin West}
\index{Osoby!Daniel Jenckes}
\index{Osoby!Eleazar Smith}
\index{Osoby!Manuel Arruda}
\index{Osoby!Harry Leshe}
\index{Osoby!Admiral Wallace}
\index{Osoby!Kolektor Robinson}
\index{Osoby!Kapitan James Mathewson}
\index{Osoby!Dr. Benjamin West}
\index{Osoby!Wielebny James Manning}
\index{Osoby!Stephen Hopkins}
\index{Osoby!John Carter}
\index{Osoby!John Brown}
\index{Osoby!Joseph Brown}
\index{Osoby!Nicholas Brown}
\index{Osoby!Moses Brown}
\index{Osoby!Dr. Jabez Bowen}
\index{Osoby!Kapitan Abraham Whipple}
\index{Osoby!Joseph Wanton}
\index{Osoby!Daniel Green}
\index{Osoby!Jedediah Orne}
\index{Osoby!Edward Hutchinson}
\index{Osoby!Luke Fenne}
\index{Osoby!Arthur Fenner}
\index{Osoby!Charles Slocum}
\index{Osoby!Simon Orne}
\index{Osoby!Wielebny Thomas Bernard}
\index{Osoby!Hepzibah Lawson}
\index{Osoby!Sędzia Hathorne}
\index{Osoby!Amity How}
\index{Osoby!Sędzia Gedney}
\index{Osoby!George Burroughs}
\index{Osoby!Sam Carew}
\index{Osoby!Mr. Parson}
\index{Osoby!Walter Dwight}

\section{Rozdział Pierwszy: Wynik i prolog}

\begin{center}
1
\end{center}

Z prywatnego szpitala dla chorych psychicznie, położonego w pobliżu Providence, Rhode Island, ostatnio zaginęła pewna szczególna osoba. Nosiła ona imię Charles Dexter Ward, i została umieszczona w zakładzie z wielkim żalem przez rozpaczającego ojca, któremu przyszło patrzeć, jak aberracja wzrasta ze zwykłego ekscentryzmu w mroczną manię zawierającą w sobie zarówno możliwość morderczych tendencji, jak i specyficznej zmiany w zawartości umysłu Charlesa. Psychiatrzy byli dosyć zaskoczeni tym przypadkiem, gdyż prezentował on dziwactwa zarówno psychologicznej, jak i fizjologicznej natury

Po pierwsze, pacjent wydawał się dziwnie starszy niż swoje 26 lat. Psychiczne rany, jeśli są prawdziwe, mogą postarzyć znacząco, lecz twarz tego młodego człowieka zawierała subtelne cechy, które normalnie uzyskują tylko ludzie wskutek starości. Po drugie, jego organiczne procesy wykazywały pewne dziwne właściwości, bez porównania z doświadczeniem medycznym doktorów. Oddychanie i praca serca wykazywały zaskakujący brak symetrii, głos jego został utracony, tak, że mógł on jedynie szeptać, trawienie trwało zaskakująco długo, w dodatku odbywało się w stopniu minimalnym, a reakcje neuronów na standardowe bodźce były nieporównywalne z wcześniejszą wiedzą medyczną, o organizmach zdrowych czy też chorych. Skóra Charlesa była sucha i o szarawym, nieświeżym odcieniu, a struktura komórkowa jego tkanek wydawała się być bardzo uszkodzona i ledwo się trzymająca razem. Nawet wielkie znamię w kształcie oliwki na jego prawym biodrze zaniknęło. W tym samym czasie, na jego klatce piersiowej uformował się wielki pieprzyk lub czarny punkt, którego tam wcześniej nie było. Ogólnie rzecz ujmując, wszyscy doktorzy zgadzali się ze sobą, że Ward posiadał metabolizm, który został uszkodzony w stopniu wykraczającym poza wcześniejsze precedensy.

Psychologicznie Charles Ward także był wyjątkowy. Jego szaleństwo nie miało sobie równych w zapiskach nawet najnowszych i najdokładniejszych rozpraw naukowych i encyklopedii medycznych, i towarzyszyła mu moc mentalna która mogłaby z niego uczynić geniusza lub przywódcę, gdyby nie wykrzywiono jej w dziwną i groteskową formę. Dr. Willett, który był lekarzem rodzinnym Wardów, potwierdza, że czyste zdolności psychiczne pacjenta, mierzalne poprzez jego reakcje na świat poza sferą jego szaleństwa, w zasadzie wzrosły od momentu pierwszego ataku. Ward, jest to prawdą, był zawsze uczonym i bibliofilem, ale nawet jego najgenialniejsze z wczesnych prac nie pokazywały jego jasności intelektu i przenikliwości, którymi się wykazywał podczas swoich rozmów z alienistami. Zaiste, było trudną rzeczą pozyskanie prawniczej zgody na przyjęcie go do szpitala, tak potężny i jasny zdawał się umysł tego młodego człowieka - tylko przy pomocy silnych dowodów i wielu anormalnym dziurom w jego zasobie informacji w kontraście do jego inteligencji, został on ostatecznie przyjęty do ośrodka. Aż do momentu jego zniknięcia był omni-czytelnikiem i tak dobrym rozmówcą, jak jego biedny głos tylko zezwolił. Uważni obserwatorzy, nie mogąc przewidzieć jego ucieczki, przewidzieli jednak, że nie minie dużo czasu, nim zostanie zwolniony ze szpitala.

\begin{center}
2
\end{center}

Tylko Dr. Willett, który odebrał poród Charlesa Warda i opiekował się wzrostem jego ciała i umysłu od tamtego czasu, wydawał się być przerażony na myśl o jego potencjalnej wolności. Miał on okropne doświadczenie i dokonał straszliwego odkrycia, którego nie ważył się wyjawić swym kolegom-sceptykom. Willett, zaiste, jest pomniejszą tajemnicą na swoich własnych zasadach w połączeniu z tym przypadkiem. Był ostatnim, który widział swego pacjenta przed jego ucieczką, i wyszedł z owej ostatniej konwersacji w stanie będącym mieszanką przerażenia i ulgi, co niektórzy wspominali, gdy ucieczka Warda stała się znana 3 godziny później. Ta ucieczka jest jedną z nierozwiązanych tajemnic szpitala Dr. Waite'a. Okno otwarte wysoko, na wysokości sześciu stóp raczej nie oferowało wyjaśnienia, lecz dalej, po tamtej rozmowie z Willettem młodzieniec niewątpliwie zniknął. Willett nie miał żadnego oficjalnego wyjaśnienia do zaoferowania, choć wydawał się być dziwnie zrelaksowany w stosunku do czasu sprzed ucieczki. Wielu, owszem, czuło, że chciałby powiedzieć coś więcej, lecz milczał z obawy przed niewiarą. Odnalazł on Warda w jego pokoju, lecz krótko po jego odejściu pielęgniarze pukali na nic. Kiedy otworzyli drzwi, pacjenta już tam nie było, a wszystko, co odkryli, to otwarte okno, przez które wiał chłodny, kwietniowy wietrzyk, rozsiewając wokół chmurę błękitno-szarawego pyłu, który prawie ich zadusił. Owszem, psy wyły jakiś czas wcześniej, ale to było wtedy, gdy Willett dalej rozmawiał z pacjentem, a nikt nie znalazł żadnego człowieka lub innej rzeczy, która wyjaśniałaby zniknięcie. Ojciec Warda został poinformowany natychmiastowo za pomocą telefonu, ale wydawało się, że jest on bardziej zasmucony niż zaskoczony. Do czasu, gdy Dr. Waite zadzwonił osobiście, Dr. Willet już z nim zdążył porozmawiać i obydwoje nie wiedzieli nic o ucieczce lub pomocy w niej. Tylko od pewnych bardzo sekretnych przyjaciół Willetta i starszego Warda pozyskano pewne wskazówki, a nawet one były zbyt fantastyczne, by brać je poważnie. Jedyny fakt który nam pozostaje to to, że na czas obecny nie ma żadnego śladu uciekającego szaleńca.

Charles Ward był antykwariuszem od swoich lat dziecięcych, niewątpliwie pozyskując swój gust ze starego miasta wokół niego i z reliktów przeszłości, które wypełniały każdy kąt starej willi jego rodziców przy Prospect Street na wzgórzu. Wraz z latami, jego oddanie antycznym przedmiotom tylko się zwiększyło, tak, że historia, genealogia i studiowanie kolonialnej architektury, mebli i rzemiosła wyrzuciły z jego umysłu wszelkie inne zajęcia. Te gusta są ważne do zapamiętania, rozważając jego szaleństwo, gdyż choć nie stanowią jego całkowitego centrum, odgrywają w nim znaczącą rolę. Luki w wiedzy, które alieniści dostrzegli były wszystkie powiązane z współczesnymi sprawami i były zakryte przez znaczącą wiedzę o sprawach dawno minionych, co wykazały przesłuchania: ktoś mógłby pomyśleć, że pacjent przeniósł się tutaj dosłownie z jednej ze starych epok poprzez jakaś osobliwą formę auto-hipnozy. Dziwną rzeczą było to, że Ward nie wydawał się już dłużej zainteresowany antykami, które znał tak dobrze. Miał, wynika, stracić wszelkie uznanie dla nich ze względu na dobrą znajomość, i jego ostatnie wysiłki było wszystkie poświęcone zdobyciu mistrzostwa w powszechnych faktach dnia codziennego, które kompletnie wyparowały z jego umysłu. To, że to kompletnie zaćmienie zdolności mentalnych nastąpiło, było czymś, co starał się on ukryć. Było to jednak oczywiste dla każdego, kto go obserwował, że jego cały program czytania i konwersacji był zdeterminowany przez życzenie pozyskania wiedzy o swoim własnym życiu i zwykłych praktycznych i kulturowych sprawach XX wieku, zgodnie z jego narodzinami w 1902 i edukacją w szkołach naszych czasów. Alieniści teraz się zastanawiają, jak z punktu widzenia jego uszkodzonej wiedzy, pacjent-uciekinier radzi sobie we współczesnym świecie; dominuje opinia, że ``zaszył się gdzieś'' do czasu, aż jego wiedza o świecie współczesnym powróci.

Początki szaleństwa Warda są kwestią dyskusyjną pośród alienistów. Dr. Lyman, eminentny autorytet z Bostonu, umieścił ją gdzieś w 1919 lub 1920, podczas ostatniego roku chłopaka w Moses Brown School, gdy nagle zwrócił się od studiów nad historią w stronę studiów okultystycznych, i odmówił wysłaniu dokumentów na uniwersytet na bazie indywidualnych badań o znacznie większym znaczeniu dla niego. To z pewnością jest powiązane z jego zmienionymi nawykami w tamtym czasie, zwłaszcza jego ciągłemu przeczesywaniu zapisków historii miejskiej i starych miejsc pochówku szukając pewnego konkretnego grobu z 1771 - grobu jego przodka imieniem Joseph Curwen, którego niektóre dokumenty znalazł ukryte w starym domu na Olnej Court, na wzgórzu Stempers, o którym jest wiadome, że Curwen go zamieszkiwał. 

Jest niewątpliwie oczywiste, że zima 1919-1920 przyniosła wielką zmianę w Wardzie, gdyż zaprzestał swoich badań antykwariusza i skupił się wyłącznie na desperackich badaniach nauk okultystycznych zarówno w domu jak i za granicą, przetykane tylko jego dziwnie stałym poszukiwaniem grobu swego przodka.

Jednakże, w tym momencie pojawia się sprzeciw Dr. Willetta, bazując swoją opinię na bliskim i ciągłym kontakcie z pacjentem, i na pewnych przerażających badaniach i odkryciach, których sam dokonał. Te badania i odkrycia odcisnęły na nim swoje piętno - jego głos się załamuje, gdy o nich opowiada, takoż jego dłoń drży, gdy próbuje o nich pisać. Willett przyznaje, że zmiana z 1919-1920 normalnie oznaczałaby początek postępującej dekadencji, która zakończyła się okropnie smutną i niespotykaną alienację z 1928, ale wierzy osobiście, że bardziej jasny rozdział powinien zostać poczyniony. Zgadza się on, że temperament młodzieńca zawsze był podatny na choroby psychiczne, i że był nadmiernie chętny w swoich odpowiedziach na zjawiska wokół niego, odmawia on przyznania, że te wczesna zmiana oznaczała przejście od poczytalności do szaleństwa, powołując się na słowa samego Warda, że odkrył coś, co najpewniej było wspaniałe i kluczowe dla historii myśli ludzkiej.

Prawdziwe szaleństwo, doktor jest pewien, nastąpiło wraz z późniejszą zmianą - po tym jak portret Curwena i jego antyczne dokumenty ujrzały światło dzienne, po tym, jak wykonano podróże do dziwnych, zagranicznych lokacji, i pewne straszliwe inwokacje zostały wyrecytowane w dziwnych i sekretnych okolicznościach, po tym, jak pewne \textit{odpowiedzi} na te inwokacje zostały jawnie wskazane, a pośpieszny list napisany w kondycji niewyjaśnionej agonii został napisany, po fali wampiryzmu i tajemniczej plotce o Pawtuxet, i po tym, jak z pamięci pacjenta usunięte zostały wszelkie wspomnienia współczesnych czasów  przy jednoczesnym osłabieniu jego głosu i jego fizyczności subtelnie zmienionej, co już wykazano wyżej.

Dopiero w tamtym czasie, Willett wskazuje z dokładnością, koszmarne cechy stały się stalą częścią Warda, i doktor jest przerażająco pewny, że istnieje dostatecznie dużo twardych dowodów, by potwierdzić twierdzenie młodzieńca o ważnym odkryciu. Po pierwsze, dwóch pracowników wysokiej inteligencji widziało antyczne dokumenty Josepha Curwena. Po drugie, sam chłopak kiedyś pokazał mu owe dokumenty i stronę z pamiętnika Curwena, i każdy z tych dokumentów posiadał autentyczny wygląd. Znana jest lokacja dziury, w której Ward znalazł owe zapiski, a sam Willett miał bardzo przekonujący ostatni rzut oka na nie w okolicznościach, którym ciężko jest dać wiarę i których być może nigdy nie udowodnimy. Dalej, są tajemnice i zbiegi okoliczności listów Orne'a i Hutchinsona, i problem pisma odręcznego Curwena i to, co detektywi odkryli o Dr. Allenie - te rzeczy i okropna wiadomość w średniowiecznych zapiskach odnalezionych w kieszeni Willetta po tym, jak odzyskał świadomość po swoim szokującym doświadczeniu.

Najważniejsze jednak są dwa okropne \textit{wyniki}, które doktor pozyskał z pewnej pary mistycznych formuł podczas swojego ostatecznego śledztwa - wyniki które udowodniły autentyczność dokumentów i ich potworne implikacje w tym samym czasie, jak i to, że te dokumenty zrodzone były z ludzkiej wiedzy.

\begin{center}
3
\end{center}

Należy spojrzeć na wcześniejsze życie Charlesa Warda jak na coś, co należy do przeszłości tak mocno, jak antyki, który sobie tam mocno ukochał. W jesieni 1918, i z pokazaniem wyraźnego zapału względem treningu wojskowego tamtego okresu, zaczął on swój pierwszy rok w Moses Brown School, która znajduje się nieopodal jego domu. Stary budynek główny, wzniesiony w 1819, zawsze był urokliwy dla oczu młodego antykwariusza - a rozległy park w którym mieści się Akademia oferował piękne widoki. Jego aktywności społeczne były mocno ograniczone, a swe godziny spędzał głównie w domu, na spacerach, w swoich klasach i w poszukiwaniu danych o antykach i genealogii w Urzędzie Miejskim, Domie Państwowym, bibliotece publicznej, Ateneum, Towarzystwie Historycznym, bibliotekach Uniwersytetu Brown, i nowo otwartej Bibliotece Shepley'a na Benefit Street. Można go sobie łatwo wyobrazić w tych dniach - wysokiego, chudego i o blond włosach, z uczonym okiem i lekko pochylonego, ubranego nieco niezgrabnie, i dającego wrażenie raczej niegroźnej niezdarności niż atrakcyjności. 

Jego spacery były zawsze przygodami powiązanymi z historią, podczas których był w stanie wyobrazić sobie miliony reliktów chwalebnej historii starego miasta i połączyć je z minionymi wiekami. Jego domem była okazała posiadłość w stylu georgiańskim, na szczycie stromego wzgórza, które wznosiło się na wschód przy rzece, a z jego tylnych okien mógł spojrzeć na ciasno upakowane wieżyce, stropy, dachy i drapacze chmur niższego miasta aż do purpurowych wzgórz poza miastem. To tutaj się urodził i z kochanego, klasycznego ganku fasady z podwójnie wypiekanych cegieł jego opiekunka prowadziła go w wózku, obok małej, białej farmy założonej 200 lat wcześniej i w kierunku budynków akademickich przy wystawnej ulicy, której stare, kwadratowe posiadłości z cegieł i mniejsze drewniane domy z wąskimi gankami opierającymi się na ciężkich kolumnach doryckich, były ustawione pośród przestronnych podwórek i ogrodów.

Był także prowadzony na wózku wzdłuż sennej uliczki Congdon Street, jeden poziom niżej na stromym wzgórzu, wraz z jej wschodnimi domami z wysokimi tarasami. Małe drewniane domki były najstarsze w tym miejscu, gdyż z tego właśnie wzgórza wzrastało rosnące miasteczko. To w tych podróżach przesiąkał on czymś w rodzaju koloru starej, kolonialnej wioski. Opiekunka miała w zwyczaju zatrzymać się i usiąść na ławkach Tarasu Prospekt, by porozmawiać z policjantami; jednym z pierwszym wspomnień tego dziecka był wielki, zachodni ocean dachów, kopuł i wież oraz odległe wzgórza które dostrzegł pewnego zimowego popołudnia z wielkiego wału z szynami, wszystkie fioletowe i mistyczne naprzeciwko rozgrzanego, apokaliptycznego zachodu słońca, malującego niebo czerwienią, złotem, purpurą i różnymi odcieniami zieleni. Szeroka, marmurowa kopuła Domu Stanowego wyróżniała się swoją masywną sylwetką, posąg będący jego ukoronowaniem otoczony fantastyczną aureolą poprzez szparę w jednej z stratusowych chmur, które lśniły na ognistych niebiosach. 

Gdy był starszy, rozpoczęły się jego słynne spacery; najpierw z jego niecierpliwie zaciągniętą opiekunką, a potem samotnie, w marzycielskiej medytacji. Głębiej i głębiej w ulice tego prawie pionowego wzgórza miał się zapuszczać, za każdym razem sięgając starszych poziomów tego antycznego miasta. Zatrzymywałby się niepewnie przy wejściu do Janckes Street z jej murami z tyłu i kolonialnymi szczytami sięgającymi aż do ciemnej Benefit Street, gdzie ukazywał mu się drewniany antyczny dom z parą drzwi w stylu jońskim, a obok niego był prehistoryczny dom z dachem mansardowym, gdzie zachowała się jeszcze resztka farmy, a także dom wielkiego sędziego Durfee z jego upadłymi pozostałościami stylu Georgiańskiego. To tutaj miały powstać slumsy, ale wielkie drzewa wiązów rzucały odżywczy cień na to miejsce, a chłopiec zwykł udawać się stąd na południe, wzdłuż długiej linii domów sprzed czasów Rewolucji Amerykańskiej, z wielkimi strychami i klasycznymi portalami. Po wschodniej stronie budynki te były osadzone wysoko ponad piwnicami, z podwójnymi schodami z kamiennymi stopniami i młody Charles mógł z wielką łatwością wyobrazić je sobie takie, jakie musiały być, gdy ulica ta była jeszcze młoda, gdy frontony były świeżo pomalowane, a nie widocznie zużyte, jak w dniu obecnym. 

Od strony zachodniej, wzgórze schodziło w dół prawie tak stromo jak wyżej, aż do starej Town Street, którą założyciele miasta umieścili przy krawędzi rzeki w 1636 r. To tutaj mieściły się niezliczone małe pasy z pochylonymi małymi domkami wielkiej antyczności; i, choć był wielce zafascynowany, minęło wiele czasu, nim odważył się spenetrować ich archaiczny wzrost z lęku przed wejściemf w sen lub wrota do nieznanych horrorów. Odkrył on, iż jest znacznie mniej straszliwe spacerowanie dalej wzdłuż Benefit Street aż do żelaznych wrót ukrytego kościółka Św. Johna i tyłu Domu Kolonialnego z 1761 r. i do zbutwiałego cielska karczmy Golden Ball, gdzie niegdyś zatrzymał się Waszyngton. Na Meeting Street  - zwanej Gaol Lane i King Street w innych czasach - szukałby na wschodzie i dostrzegłby łuk schodów, wspinających się na wzgórze, a w dół, w kierunku zachodnim, dostrzegłby Szkołę Kolonialną ze starych cegieł,  śmiejącą się z antycznego znaku z głową Szekspira, gdzie \textit{Providence Gazette} i \textit{Country-Journal} były drukowane przed Rewolucją. Następnie pojawiał się przecudowny Kościół Pierwszych Baptystów z 1775 r., luksusowy widok z jego niepowtarzalną wieżą Gibbsa i Georgianskimi dachami i kopułami, unoszącymi się wysoko. Tutaj i w kierunku południowym sąsiedztwo stało się lepsze, w ostateczności wykwiłwszy w cudowną grupę wczesnych posiadłości; lecz dalej małe antyczne paski prowadziły w kierunku zachodnim, wydawawszy się duchami z ich wielo-spiczastymi archaizmami, i ociekającymi od błyszczącego rozkładu, gdzie stare nadbrzeża wspominają dumnie czasy Indii Wschodnich wśród poliglotów, rozkładających się sterów, sklepów żeglarskich z zamglonymi witrynami i nazwami ulic z dawna, które przetrwały aż po dziś dzień, takimi jak Packet, Bullion, Gold, Silver, Coin, Doubloon, Sovereign, Guilder, Dollar, Dime, i Cent.

Czasami, po tym, jak urósł wyższy i bardziej żądny przygód, młody Ward ruszał głębiej w dół, ku burzy chwiejnych domów, złamanych poprzecznic, schodów, wygiętych balustrad, ciemnych twarzy i nienazwanych zapachów, ciągnących się od South Main do South Water, szukając doków, gdzie zatoka i statki parowe się spotykały, i wracał w stronę północną niższymi poziomami magazynów z 1816 o stromych dachach i szeroką drogą Wielkiego Mostu. To tutaj rynek z 1773 dalej stoi na jego antycznych łukach. Na tej szerokiej drodze zwykł się zatrzymywać, by wypić z czary piękna starego miasta gdy to wznosiło się w stronę wschodnią, pełne Georgiańskich szczytów i ukoronowane nową kopułą Christian Science tak, jak Londyn jest koronowany kopułą katedry św. Paula. Najbardziej lubił docierać do tego punktu późnym popołudniem, gdy światło słoneczne dotykało rynku i antycznych dachów domów na wzgórzu i ich dzwonnic, malując je złotem, i rzuca swą magię pośród wyśnionych nadbrzeży, gdzie niegdyś indianie z Providence zwykli zarzucać kotwice. Po dłuższym przyjrzeniu się temu widokowi, zakręciłoby mu się w głowie z poetycką miłością, i wtedy zacząłby drogę powrotną do domu, mijając po drodze stary, biały kościół i wąskie, strome uliczki, gdzie złote błyski odbijałyby się w oknach i naświetlach umieszczonych wysoko ponad podwójnymi schodami z balustradami z kutego żelaza. 

Innymi razy, i w późniejszych latach, szukał on wyrazistych kontrastów: spędzając połowę swojego spaceru w rozpadających się kolonialnych regionach na północny wschód od swego domu, gdzie wzgórze opada na najniższy poziom do Stempers Hill z jego gettem murzyńskim\footnote{Ogólnie rzecz ujmując nie lubię ``słowa na M'' i wierzę, że nie powinno się z niego korzystać w języku polskim. Tutaj użyłem go, niejako wbrew sobie, gdyż (A) Lovecraft użył w oryginale słowa ``Negro'' i (B) wiem doskonale, że za życia posiadał on uprzedzenia rasowe. Pragnę jednak zaznaczyć, że właściwy język we współczesnej polszczyźnie to ``osoba czarnoskóra'' a nie słowo na M. (przyp. tłum.)}, gromadzącym się wokół miejsca, gdzie dyliżansy do Bostonu zwykły rozpoczynać swą podróż przed czasami Rewolucji. Drugą połowę swoich spacerów spędzał on na południowych ziemiach wokół ulic George'a, Benevolent, Power i Williamsa, gdzie stare wzgórze jest gruntem dla pięknych włości i odgrodzonych murami ogrodów oraz zielonych ścieżek, z którymi wiąże się tyle drogich wspomnień. Te podróże, razem z pilnymi studiami, które im towarzyszyły, z pewnością przyczyniły się do wielkiej wiedzy historycznej zamieszkującej umysł Charlesa Warda. Ilustruje to także mentalną podstawę, niejaki grunt, na który padłu nasiona tej pamiętnej zimy 1919-1920, która to poruszyła wydarzenia, które miały tak dziwny i straszliwy finał.

Dr. Willett jest pewien, że do tej okropnej zimy, zainteresowania historyczne Charlesa Warda były wolne od zaburzonej psychiki. Cmentarze były wtedy dla niego bez szczególnego znaczenia, poza ich wartością historyczną, a cokolwiek w stylu przemocy lub dzikiego instynktu było poza jego psyche. Wtedy, sunąc powoli, acz nieubłaganie, pojawił się ostateczny wynik jednego z jego badań genealogicznych z poprzedniego roku; wtedy, gdy odkrył wśród swoich przodków po stronie matki pewnego długowiecznego człowieka imieniem Joseph Curwen, który przybył z Salem w marcu 1692, i o którym krążyły plotki i historie, które trudno powtarzać w dobrym towarzystwie.

Pra-pra-pradziadek Warda, Welcome Potter, w 1785 wziął ślub z pewną ``Ann Tillinghast, córką Pani Elizy oraz Kapitana Jamesa Tallinghasta'', o której rodowodzie rodzina nie posiadała żadnych szczegółowych informacji. Pod koniec 1918, podczas szukania woluminu o pierwotnych danych historycznych, młody genealog natrafił na wpis opisujący legalną zmianę nazwiska, które w 1772 pani Eliza Curwen, wdowa po Josephcie Curwenie, przyjęła razem ze swoją siedmioletnią córką Ann, której nazwiskiem panieńskim było Tillinghast, uzasadniając ową zmianę ``iż imię jej Męża stało się publiczną Obrazą dla Rozumu, bazując na wiedzy o jego Chorobie, która potwierdza pewne powszechne, antyczne Plotki, tak więc nie pragnę być znana jako jego lojalna Żona, dopóki owe plotki nie okażą się Bzdurne ponad wszelką wątpliwość''. Ten wpis ujrzał światło dzienne po przypadkowej separacji dwóch kartek, które ostrożnie sklejono razem i które były wyłączone z skądinąd poprawnej numeracji stron.

Było od razu oczywiste dla Charlesa Warda, że odkrył nieznany wcześniej sekret swojego pra-pra-pra-pradziadka. Odkrycie podnieciło go szczególnie, gdyż już słyszał ogólne raporty i widział rozsiane pogłoski o tej osobie, odnośnie której pozostało tak mało weryfikowalnych danych, pomijając te nieliczne, które ujrzały światło dzienne dopiero we współczesnych czasach, tak, że prawie zdawało się to być spiskiem, mającym na celu usunięcie go z ludzkiej pamięci. To, co jednak wydawało się płynąć z tych danych, miało tak prowokacyjną naturę, że nie dało się wyobrazić sobie, co powodowało, że owi kolonialni kronikarze byli tak pełni lęku i chętni, by ukryć i zapomnieć, lub by podejrzewać, że powody owego wykasowania były aż zanadto właściwe. 

Przed tym, Ward był wielce kontent, mogąc wyobrażać sobie starego Josepha Curwena jako coś zawieszonego w powietrzu - ale odkrywszy swoją własną relację z ową ``wymazaną'' figurą, zaczął szukać o nim danych tak systematycznie, jak to tylko możliwe. W tym ekscytującym zadaniu, ostatecznie odniósł on sukces poza swoimi największymi spekulacjami, gdyż stare listy, pamiętniki i nieopublikowane wspomnienia znalezione w zakurzonych strychach Providence i innych miast i miasteczek były obfite w wiele oświecających zapisków, odnośnie których ich autorzy nie uznali za stosowne, by je zniszczyć. Jedna ważna informacja pochodziła ze źródła tak odległego jak Nowy York, gdzie pewne zapiski kolonizatora z Rhode Island przechowywane były w Muzeum Frances' Tavern. Najważniejszą rzeczą jednak, i było to coś, co zdaniem Doktora WIlleta stanowiło ostateczne źródło klęski Warda, było coś, co znalazł w sierpniu 1919 za panelami starego domu w Olney Court. To było to, ponad wszelką wątpliwość, co otworzyło przed nim owe ciemne wizje, które kończyły się daleko poza dnem piekielnym. 

% TUTAJ SKOŃCZONE

\newpage

\section{Rozdział Drugi: Przodek i Horror}

\begin{center}
1
\end{center}

Joseph Curwen, co wyjawiły szeptane legendy odkryte przez Warda, był niezwykłym, enigmatycznym, w tajemniczy sposób okrutnym indywiduum. Uciekł z Salem do Providence - ostatecznego schronienia dla dziwnych, wolnych i niezgadzających się z powszechnie przyjętym konsensusem - na początku wielkiej paniki wiedźm. Bał się on prześladowań ze względu na samotniczy tryb życia i dziwne chemiczne czy też alchemiczne eksperymenty. Był on osobą o szarej skórze około trzydziestki, i szybko został obywatelem Providence, zakupiwszy później dom na północ od włości Gregory'ego Dextera na wysokości Olney Street. Jego dom był wybudowany na Stempers Hill na zachód od Town Street, w miejscu, które później zostało Olney Court. W 1761 zamienił ten dom na większy, przy tej samej ulicy, który dalej stoi w tamtym miejscu.

Warto zaznaczyć, że pierwszą dziwną rzeczą odnośnie Josepha Curwena było to, że zdawał się nie starzeć. Był zaangażowany w handel, zakupił miejsce na łódkę przy Mile-End Cove, pomógł odbudować Wielki Most w 1713 i Kościół Kongregacji na wzgórzu, ale zawsze posiadał wygląd mężczyzny, który nie miał więcej niż trzydzieści, może trzydzieści pięć lat. W miarę, jak dekady upływały jedna za drugą, ta cecha zaczęła przyciągać uwagę ludu. Curwen zawsze wyjaśniał, że pochodzi z długoletniego rodu, a za pośrednictwem prostego żywota uzyskał świetne zdrowie. Jak ową prostotę można pogodzić z niewyjaśnionymi zakupami sekretnego kupca i z dziwnymi światłami w oknach jego domu przez całą noc, nie było zbyt oczywiste dla mieszkańców miasta. Zamiast tego, postulowali one inne wyjaśnienie jego ciągłej młodości i długoletności. Istniał powszechny konsensus, że mieszanie przez Curwena tajemniczych chemikaliów było odpowiedzialne za jego stan. Plotki głosiły o dziwnych substancjach, które kupował z Londynu i Indii, transportowanych na jego statkach lub zakupionych w Newport, Bostonie lub Nowym Yorku, a gdy stary doktor Jabez Bowen przybył z Rehoboth i otworzył swoją aptekę po drugiej stronie Wielkiego Mostu, zwaną Pod Jednorożcem i Moździerzem, gorące szepty nie mogły zamilknąć o lekach, kwasach i metalach, które długowieczny odludek zakupił lub zamówił u niego. Działając w przekonaniu, iż Curwen posiadał cudowne i tajemnicze medyczne zdolności, wielu chorych ciągnęło do niego z prośbami o pomoc, lecz choć zdawał się on zachęcać ich wierzenia odnośnie samego siebie w luźny sposób, i zawsze dawał im mikstury w dziwnych kolorach w odpowiedzi na ich prośby, zauważono, że jego dary dla innych rzadko kiedy przynosiły im korzyści. Po upływie lat pięćdziesięciu, i bez większej zmiany niż 5 lat na jego licu, szepty ludzi stały się znacznie mroczniejsze. W efekcie, stał się jeszcze większym odludkiem.

Prywatne listy i pamiętnik z tego okresu ujawniają znacznie więcej powodów dla których Joseph Curwen był obiektem zdumienia, strachu, a w końcu potępiany niczym jakaś plaga. Posiadał pasję do cmentarzy, na których można go było dostrzec o każdej porze dnia i w każdych warunkach. Był z niej znany, choć nikt nie dostrzegł go nigdy czyniącego cokolwiek, co mogłoby być uznane za upiorne. Posiadał on farmę na Pawtuxet Road, na której spędzał czas w lecie, i na której często można było go dojrzeć, jeżdżącego konno o dziwnych godzinach dnia i nocy. Jego jedynymi znanymi sługami, farmerami i dozorcami była smutna para Indian z plemiona Narragansett - mąż głupi i pokryty bliznami, a żona z bardzo obrzydliwą twarzą, najpewniej ze względu na to, iż posiadała w sobie domieszkę czarnej krwi. W dobudówce do tego domu znajdowało się laboratorium, gdzie dokonywano większości chemicznych eksperymentów. Ciekawscy kurierzy, którzy dostarczali butelki, torby lub pudełka poprzez małe drzwi z tyłu rozsiewali plotki o fantastycznych flaszkach, tyglach, alembikach i piecach, które widzieli w niskim pomieszczeniu pełnym półek. Przepowiadali oni szeptem, że cichy ``chymik'' - przez co mieli na myśli \textit{alchemika} - wkrótce odkryje Kamień Filozoficzny. Najbliższe sąsiedzi tej farmy - Fennersowie, którzy znajdowali się o 1/4 mili dalej - mieli jeszcze dziwniejsze opowieści o pewnych dźwiękach, które, byli pewni, dochodziły z domu Curwena w środku nocy. Były to krzyki, mówili oni, i przeciągłe ryki. Nie lubili oni także wielkiej ilości zwierząt, które tłoczyły się na pastwiskach, gdyż tak wielka ich liczba nie była potrzebna by zapewnić staremu człowiekowi i paru sługom mięso, mleko i wełnę. Poszczególne zwierzęta w stadzie zdawały się zmieniać z tygodnia na tydzień, gdy nowe zwierzęta były zakupywane od rolników z Kingstown. Coś dziwnego było także odnośnie pewnego wielkiego, kamiennego budynku na uboczu, którego wysokie, wąskie szczeliny robiły za okna. 

Ludzie spotykani w pobliżu Wielkiego Mostu mieli dużo do powiedzenia o domu Curwena na Olney Court - nie o tym nowym wybudowanym w 1761, kiedy ten mężczyzna liczył sobie prawie wiek, lecz pierwszym - o niskim strychu pozbawionym okien i ścianami krytymi gontem, które spalił do cna po jego zdemolowaniu. Tutaj było mniej tajemnic, to prawda, ale godziny, o których widziano światła, sekretność dwóch obcokrajowców, którzy byli jedynymi sługami, ohydne szepty i dźwięki wydawane przez niesamowicie starego Francuza na usługach Curwena, wielkie ilości posiłków wchodzące przez drzwi, za którymi żyło tylko 4 ludzi, i ogólny \textit{wydźwięk} pewnych głosów często słyszanych w szeptanych rozmowach w dziwnych czasach - wszystko to razem połączone z tym, co było wiadome o farmie w Pawtuxet dało jej złowróżebną sławę.
 
W lepszych kręgach dom Curwena również był tematem rozmów. Nowoprzybyły,  naturalnie czynił znajomości w kościele i życiu społecznym miasta, znajomości lepszego sortu, których towarzystwo i konwersacje w oczywisty sposób sprawiały mu przyjemność. Jego urodzenie było wiadomie dobre, gdyż Curwenowie lub Carwenowie z Salem nie wymagali przedstawienia nikomu w Nowej Anglii. Było ewidentne, że Joseph Curwen podróżował dużo za młodu, żyjąc przez pewien czas w Anglii i pokonując co najmniej dwie podróże do Orientu. Jego mowa, gdy tylko miał taką potrzebę, była językiem wykształconego, kulturalnego Anglika. Ale z pewnego powodu Curwen nie dbał o swoją pozycję społeczną. Choć nigdy nie odmawiał otwarcie gościom, zawsze posiadał wokół siebie ścianę rezerwy, tak, że mało kto mógł pomyśleć o czymkolwiek do powiedzenia do niego, by nie zabrzmieć niczym szaleniec. 

Wydawało się, że sposób, w jaki Curwen się nosi, zawiera w sobie jakąś sekretną, sardoniczną arogancję, jak gdyby odkrył, że wszystkie istoty ludzkie są nużące po tym, jak bywał wśród dziwniejszych i bardziej potężnych istot. Kiedy Dr. Checkley, słynny w swym fachu, przybył do Bostonu w 1738, by zostać proboszczem w King's Church, nie omieszkał on wezwać tego, o którym słyszał tak wiele, ale odszedł po krótkiej chwili, której potrzebował, by wyczuć nutę zła w słowach jego gospodarza. Charles Ward powiedział swemu ojcu, że kiedy dyskutowali o Curwenie pewnego zimowego wieczora, że dałby wiele, by dowiedzieć się, co tajemniczy starszy mężczyzna powiedział młodemu kapłanowi, ale wszelcy autorzy pamiętników byli zgodni, że Dr. Checkley nie powtórzył niczego, co usłyszał. Ten dobry człowiek był naprawdę zaszokowany, i nigdy nie wspominał Josepha Curwena bez pewnej widocznej straty radości, z której był znany.

Istnieje jednak bardziej określony powód, dla którego inny człowiek wielkiego smaku i urodzenia unikał uczonego odludka. W 1746 pan John Merritt, starszy angielski dżentelmen o wykształceniu naukowym i literackim, przybył z Newport do miasta, które tak szybko się rozwijało i zaczął się budować w miejscu w Neck, które obecnie jest sercem najlepszej dzielnicy rezydenckiej. Żył on w wielkim stylu i komforcie, posiadając pierwszy dyliżans w mieście i zatrudniając dużo sług, i będać bardzo dumnym ze swojego teleskopu, mikroskopu i bardzo selektywnej biblioteki pełnej angielskich i łacińskich książek. Usłyszawszy, że Curwen jest posiadaczem najlepszej biblioteki w Providence, pan Merritt chętnie złożył mu wizytę, i został przyjęty grzeczniej niż większość innych gości w jego domu. Jego zachwyt półkami gospodarza wypełnionymi klasyką w angielskim, łacinie i grece z dodatkiem tekstów filozoficznych, matematycznych i naukowych, w których wliczał się Paracelsus, Agricola, Van Helmont, Salvius, Glauber, Boyle, Boerhaave, Becher i Stahl, sprawił, że Curwen zaproponował wizytę do swojej farmy i laboratorium, do których nigdy wcześniej nie zaprosił nikogo. Obydwoje ruszyli natychmiast w dyliżansie pana Merritta.

Pan Merritt zawsze twierdził, że nie dostrzegł niczego w oczywisty sposób bluźnierczego na farmie, ale utrzymywał, że tytuły książek w specjalnej bibliotece taumaturgicznej, alchemicznej i teologicznej, którą Curwen trzymał w przednim pokoju, były same w sobie dostateczne, by wywołać w nim uczucie głębokiego dyskomfortu. Być może jednak, to wyraz twarzy właściciela księgozbioru przyczynił się bardzo mocno do tego uprzedzenia. Niecna kolekcja, poza zbiorem standardowych dzieł, które nie zaalarmowały pana Merritta, posiadała w sobie niemal wszystkie dzieła kabalistyczne, demonologiczne i magiczne znane człowiekowi, i była prawdziwa skarbnicą wiedzy odnośnie wiedzy w wątpliwej domenie alchemii i astrologii. Hermes Trismogistus w edycji Mesnarda, \textit{Turba Philosopharum}, \textit{Liber Investigationis} Gabera i \textit{Klucz do Mądrości} Artephousa - wszystkie tutaj były, razem z kabalistycznym \textit{Zoharem}, zestawem \textit{Albertus Magnus} Petera Jamma, \textit{Ars Magna et Ultima} Raymonda Lully'ego w edycji Zetznera, \textit{Thesaurus Chemicus} Rogera Bacona, \textit{Clavis Alchimiae} Fludda, \textit{De Lapide Philosophico Trithemusa}. Średniowieczni Żydzi i Arabowie byli obecni w księgozbiorze aż zanadto, i pan Merritt był wstrząśnięty, gdy chwyciwszy z półki wolumin podejrzenie nazwany \textit{Qanooon-e-Islam}, odkrył, iż był to w istocie zakazany \textit{Necronomicon} szalonego Araba Abdula Alhazreda, o którym słyszał takie potworne rzeczy szeptane parę lat wcześniej po kontakcie z nienazwanymi rytuałami w dziwnej, małej wiosce Kingsport w zatoce Massachusetts.

Lecz co najdziwniejsze, szlachetny dżentelmen poczuł się najbardziej zniesmaczony przez malutki detalik. Na wielkim stole z z mahoniu leżała odkładką w dół znoszona kopia Borellusa, nosząca wiele tajemniczych, odręcznych zapisków stworzonych ręką Curwena. Książka była otwarta mniej-więcej w połowie, i jeden z paragrafów był podkreślony tak wyraźnymi liniami, że gość nie mógł się powstrzymać i zaczął go czytać. Niezależnie od tego, czy była to natura podkreślonego akapitu, czy gorączkowa natura linii tworzących podkreślenie - pan Merritt sam nie wiedział - ale coś w tej kombinacji wpłynęło na niego w sposób znaczący i złowieszczy. Mógł przywołać ten akapit z pamięci do końca swoich dni, zapisawszy go w swoim pamiętniku i raz spróbowawszy wyrecytować go swojemu bliskiemu przyjacielowi, Dr. Checkley'owi, choć przestał, zauważywszy, jak bardzo proboszcz czuł się poruszony owym cytatem .Stanowił on, iż:

\begin{displayquote}

Esencjalne Sole Zwierzęce mogą być tak przygotowane i przechowane, by inteligentny Człowiek posiadał całą Arkę Noego w swoim Studium, i wskrzesił Kształt Zwierzęcia z jego Popiołów dla swojej własnej Przyjemności, i podobną Metodą, z esencjalnych Soli ludzkiego Pyłu, Filozof może, bez zbrodni Nekromancji, wezwać Kształt dowolnego martwego Przodka z Pyłu w który jego Ciało się obróciło.

\end{displayquote}

To jednak doki w południowej części Town Street były miejscem, gdzie krążyły najgorsze pogłoski o Josephie Curwenie. Marynarze są bardzo przesądni i doświadczeni żeglarze, którzy przywozili nieskończone ilości rumu, niewolników i słupy melasy, szelmowscy korsarze i wielkie brygi Brownsów, Crawforsów i Tilliinghastów, wszyscy czynili znaki ochronne, kiedy widzieli chudego, podejrzenie młodo wyglądającego człowieka o żółtych włosach, lekko pochylonego, gdy wchodził do magazynu na Doubloon Street lub rozmawiającego z kapitanami na długim nabrzeżu, na którym statki Curwena płynęły bez ustanku. Księgowi i kapitanowie pracujący dla Curwena nienawidzili i bali się go, a wszyscy jego marynarze byli wzięci z Martinique, St. Eustatius, Havany lub Port Rolay. W pewien sposób, to częstotliwość, z jaką ci marynarze byli zmieniani, była powodem, który prowokował najbardziej materialną część lęku, który towarzyszył staremu człowiekowi. Jego załoga mogłaby się rozejść po mieście lub wybrzeżu, niektórzy z jej członków musieliby zrobić to czy tamto, a gdy ponownie by się spotkali, z całą pewnością brakowałoby im mężczyzny lub dwóch. Wiele owych rzeczy, które musieliby zrobić, dotyczyłoby farmy na Pawtuxet Road, i tak mało marynarzy kiedykolwiek wróciło z tego miejsca. Ten fakt tkwił wyraźnie w pamięci żeglarzy. W pewnym momencie przyczyniło się to do znacznego utrudnienia zamorskich biznesów Curwena. Ostatecznie paru z nich zostawiłoby statki za swoimi plecami, po usłyszeniu plotki o nabrzeżach Providence, i ich zastępcy musieli pochodzić aż z Indii Zachodnich, co stało się dużym problemem dla kupca.

W 1760 Joseph Curwen był ostatecznym wyrzutkiem, podejrzewanym o nieokreślone zbrodnie i demoniczne przymierza, które wydawały się tym straszniejsze, że nie mogły być nazwane, zrozumiane, lub choćby udowodnione, jakoby istniały. Ostatni gwóźdź do trumny był problem z zaginionymi żołnierzami z 1758, gdyż w marcu i kwietniu tego roku  dwa królewskie regimenty stacjonowały w Providence, na trasie do Nowej Francji. Żołnierze ci znikali bez śladu ponad spodziewaną liczbę dezercji. Od razu podniosły się plotki o częstotliwości, z jaką Curwen był widziany, rozmawiając z obcymi w czerwonych płaszczach, a gdy paru z nich zostało uznanych za zaginionych, ludzie wrócili myślami do dziwnych przypadłości jego własnych marynarzy Co by się stało, gdyby regimenty nie ruszyły dalej, tego nikt nie jest w stanie powiedzieć. 

W międzyczasie, materialne sprawy kupca prosperowały. Posiadał w praktyce monopol na handel saletry potasowej, czarnym pieprzem i cynamonem, i z łatwością przewodził handlowi innymi dobrami, z pierwszeństwem być może Brownsów w jego imporcie indygo, wyrobów mosiężnych, bawełnianych i wełnianych, soli, żelaza, papieru i wszelkich dóbr korony brytyjskiej. Tacy sklepikarze jak James Green z Słonia w Cheapside, Russelowie z Złotego Orła po drugiej stronie Mostu, lub Clark i Nightingale z Ryby i Patelni w pobliżu New Coffee-House, polegali prawie całkowicie na nim, by zaopatrywać się w towary, i jego biznesy z lokalnymi warzelniami alkoholu, mleczarzami i hodowcami konii z plemienia Narragansett i twórcami świec z Newport, uczyniły z niego jednego z głównych eksporterów w całej Kolonii.

Choć był on niewątpliwie ofiarą ostracyzmu społecznego, potrafił być miły, w pewien sposób. Kiedy Dom Kolonialny spalił się do fundamentów, pomógł wtedy skrzywdzonym pokaźną sumą, która pozwoliła wybudować nowy budynek, z cegieł - ciągle stojący niczym przewodniczący paradzie na starej głównej ulicy. Działo się to w 1761 r. W tym samym roku, pomógł on odbudować Wielki Most po załamaniu chmury w październiku. Odbudował on wielką część księgozbioru biblioteki publicznej, którą pochłonął pożar Domu Kolonialnego. Wsparł także loterię, która dała błotnistemu Market Parade i Town Street chodniki z wielkich, obłych kamieni, aby można było się komfortowo poruszać na piechotę. Mniej-więcej w tym samym czasie, wybudował on prosty lecz piękny nowy dom, którego drzwi były tak wielką zdobyczą sztuki płaskorzeźby. Kiedy wierni z Whitefield odseparowali się od kościoła na wzgórzu Dr. Cottona w 1743 i założyli kościół w Deacon Snow po drugiej stronie Mostu, Curwen wyruszył z nimi, choć jego wiara religijna szybko nie okazała się zbyt żarliwa. Teraz, jednak, stał się wierzący po raz kolejny, jakby próbując odrzucić cień, który wymusił na nim izolację i który wkrótce miał zacząć niszczyć jego biznesy, gdyby czegoś z tym nie zrobił. 

Widok tego dziwnego, bladego mężczyzny, wyglądającego na będącego ledwie w wieku średnim, lecz z pewnością nie młodszym niż stulecie, który zapragnął wreszcie wyłonić się z chmury lęku i zohydzenia zbyt uogólnionych, by je dookreślić lub zanalizować, był zarówno żałosny, dramatyczny i ohydny. Jednakże, taka jest moc bogactwa i powierzchownych gestów, że zaiste, objawiło się niskie obniżenie w widocznej awersji pokazywanej mu przez gmin - zwłaszcza po tym, jak nagłe zaginięcia jego żeglarzy przeminęły jak z bicza strzelił. Musiał on także zacząć stosować ekstremalną ostrożność i sekretność odnośnie swoich wypraw na cmentarz, gdyż nigdy później już go nie przyłapano na krążeniu po nim. Podobnie, pogłoski o dziwnych dźwiękach i światłach w jego posiadłości w Pawtuxet także stały się mniej częste. Jego zapotrzebowanie na żywność i bydło pozostało niewytłumaczalnie wysokie. Dopiero w czasach Charlesa Warda, którzy prześwietlił jego rachunki i dokumenty w Shepley Library, uderzyło go, jako jedynego, że można porównać wielką liczbę Czarnych z Guinei, których importował aż do 1766 z bardzo małą ich liczbą, która była oddawana handlarzom niewolników w Wielkim Moście lub do plantacji w Narragansett Country. Z pewnością, bystrość i geniusz tej okropnej, nieludzkiej osoby były oczywiste, gdy już uświadomiło się ich ogrom.

Ale oczywiście, efekt tych wszystkich zabiegów naprawczych był raczej znikomy. Curwen dalej był unikany przez swoich pobratymców - zaiste, wszyscy zazdrościli mu długowieczności. Mógł on dostrzec, że w efekcie jego bogactwa z pewnością musiały na tym ucierpieć. Jego skomplikowane studia i eksperymenty, niezależnie od tego, czego dotyczyły, najwyraźniej wymagały wielu pieniędzy, aby móc je kontynuować. Gdyby plotki wreszcie wpłynęły na jego zarobki - gdyby pozbawić go przychodu z handlu, które pozyskał, nie pomogłoby mu zaczęcia od nowa w innym miejscu. Osąd sytuacyjny wymagał, że powinien naprawić swoje relacje z mieszkańcami miasta Providence, tak, aby jego obecność już nie była powodem szeptanych konwersacji, wybitnie oczywistych wymówek nagłych obowiązków i ogólnej atmosfery powstrzymania się i niezręczności. Jego pracownicy, zredukowani obecnie do bezczynnych i zmęczonych cieni, których nikt inny by nie zatrudnił, przysparzali mu wielu problemów. Co zaś do jego kapitanów i marynarzy, trzymał ich przy sobie dzięki przebiegłości i przewagom, które pozyskał nad nimi - takiej jak weksle, hipotekę lub posiadanie informacji potrzebnych, by ich szantażować. W wielu przypadkach, twórcy pamiętników odnotowywali z niejakim zdziwieniem, że Curwen posiadał prawie magiczną moc śledzenia rodzinnych sekretów, które następnie wykorzystywał w wątpliwych celach. Podczas ostatnich 5 lat jego życia, wydawało się, że tylko bezpośrednie rozmowy z dawno zmarłymi ludźmi mogły mu zapewnić ogrom informacji, które trzymał na końcu swego języka. 

Mniej-więcej w tym samym czasie, zręczny uczony wykonał ostatnią próbę, by wrócić do łask lokalnej społeczności. Do tego momentu kompletny pustelnik, teraz zapragnął zawrzeć korzystny związek małżeński, biorąc za żonę kobietę, której wysoka pozycja społeczna uczyniłaby wszelkie próby ostracyzmu względem niego niemożliwymi. Mozliwe, że miał również inne, ważniejsze powody, by zawrzeć takie małżeństwo - powody na tyle oddalone od normalnej logiki, że tylko zapiski znalezione półtora wieku później, po jego śmierci, sprawiły, że ktokolwiek mógł ich podejrzewać. O tym jednak nic pewnego nie może zostać w pełni poznane. Naturalnie, był on świadom horrorów i upokorzeń związanych z uwodzeniem, które spotkałyby jego osobę, tak więc szukał odpowiedniej kandydatki, na której rodzicach mógłby wywrzeć stosowną presję. Takie kandydatki, odkrył, nie były tak proste do odnalezienia, gdyż miał on swoje bardzo konkretne wymagania odnośnie ich piękna, osiągnięć i poważania w społeczeństwie. Jego wymagania zawęziły możliwy wybór do domu jednego z jego najlepszych i najstarszych kapitanów, wdowca wysokiego urodzenia i nieposzlakowanej opinii imieniem Dutie Tillinghast, którego jedyna córka Eliza posiadała wszelkie możliwe przewagi, z wyjatkiem zostania dziedziczką rodu. Kapitan Tillinghast był kompletnie zdominowany przez Curwena - i zgodził się, po koszmarnym wypytywaniu w swoim kopulastym domu na wzgórzu Power's Lane, pobłogosławić ten bluźnierczy mezalians. 

Eliza Tillinghast miała w tamtym czasie 18 lat i została wychowana tak łagodnie, jak pozwoliły na to warunki w skromnym domu jej ojca. Chodziła do szkoły Stephena Jacksona po drugiej stronie Court House Parade i była uczona przez swoją matkę, zanim ta umarła na ospę w 1757 r., odnośnie wszelkich sztuk domowego żywota. Przykład jej kunsztu, stworzony w 1753 r. można dalej znaleźć w pomieszczeniach Towarzystwa Historycznego Rhode Island. Po śmierci jej matki zajmowała się ona domem, wspierana tylko przez jedną starą, czarnoskórą kobietę. Jej kłótnie z jej ojcem odnośnie propozycji malżeńskiej Curwena musiały być zaiste burzliwe, lecz nie mamy o nich żadnych świadectw. Jest jednak pewne, że jej narzeczeństwo względem młodego Ezry Weedena, drugiego oficera na statku Enterprise pod dowództwem Crawforda, zostało odwołane, a jej związek z Josephem Curwenem wszedł w życie siódmego marca 1763 r. Miało to miejsce w kościele Baptystów, w obecności najznamienitszych gości, jakich tylko miasto mogło zgromadzić, ceremonii zaś przewodził młody Samuel Winson. The Gazette wspomniała o zawarciu małżeństwa w paru zdaniach, a w większości kopii, które przetrwały, ta krótka wzmianka była wycięta lub wyrwana. Ward odnalazł jedną jedyną kompletną kopię po wielu godzinach poszukiwań w archiwach prywatnego kolekcjonera, rozbawiony bezsensownym miejskim językiem wzmianki:

\begin{displayquote}

W poniedziałek wieczorem, Pan Joseph Curwen, z naszego Miasta, Kupiec, ożenił się z Panną Elizą Tillinghast, Córką Kapitana Dutie'go Tillinghasta, młodą panienką prawdziwej Wartości, dodanej do Piękna jej Osoby, co uświętni Stan Małżeński i pomnoży jego Szczęście.

\end{displayquote}

Kolekcja listów Durfee-Arnolda, odkryta przez Charlesa Warda tuż przed jego pierwszym załamaniem nerwowym w prywatnej kolekcji Melville'a F. Petersa z George Street, i obejmująca tamten i poniekąd wcześniejszy okres, rzuca wyraźne światło na publiczne oburzenie na ten źle dobrany związek małżeński. Społeczne wpływy Tillinghastów, jednakże, były niezaprzeczalne, i po raz kolejny Joseph Curwen odkrył, że jego dom stał się odwiedzany przez osoby, których w innych okolicznościach w ogóle nie przepuściłby przez swój próg.  Jego akceptacja tego faktu nie była wszak kompletna, a jego żona cierpiała społecznie ze względu na wymuszone przedsięwzięcie, ale poprzez kolejne wydarzenia udało im się uniknąć ostracyzmu społecznego. W jego traktowaniu swojej żony, dziwny pan młody zaskoczył zarówno nią, jak i szerszą społeczność, poprzez ukazywanie ekstremalnego taktu i wdzięczności. Nowy dom w Olney Court był teraz zupełnie wolny od dziwnych manifestacji, a choć Curwen w dużej mierze porzucił farmę w Pawtuxet, której jego żona nigdy nie odwiedzała, wyglądał teraz bardziej jak normalny człowiek niż w jakimkolwiek innym momencie swojego wcześniejszego, długiego życia. Tylko jedna osoba pozostała z nim w otwartej wrogości - wspomniany młody oficer, z którym zaręczyny Eliza Tillinghast zerwała. Ezra Weeden całkiem otwarcie poprzysiągł zemstę Curwenowi i, choć zazwyczaj był łagodnego usposobienia i manier, nabierał teraz nienawistnego poczucia celu, które nie wróżyło dobrze mężowi-uzurpatorowi.

Siódmego dnia maja 1765 roku, urodziło się jedyne dziecko Curwena, Ann, ochrzczona przez Wielebnego Johna Gravesa z King's Church, do którego zarówno mąż, jak i żona zapisali się wkrótce po ślubie, jako kompromis pomiędzy ich dwoma wcześniejszymi wspólnotami religijnymi - Congregationalnej i Baptystów. Zapisy tych narodzin, jak i ślubu 2 lata wcześniej, zostały usunięte z większości kopii kościelnych i dokumentów miejskich, gdzie powinny były się pojawić. Charles Ward zdołał odnaleźć obydwa z wielką trudnością po jego odkryciu, iż wdowa zmieniła swe imię, co poinformowało go o jego własnej relacji z nim, i zagroziło gorączkowymi poszukiwaniami, które miały swoją kulminację w jego szaleństwie. Zapiski o porodzie, zaiste, zostały odkryte bardzo ciekawe poprzez korespondencję ze spadkobiercami lojalisty Dr. Gravesa, który zabrał ze sobą duplikaty zapisków kiedy opuścił miasto na początku Rewolucji. Ward spróbował zasięgnąć informacji u tego źródła, będąc w pełni świadomym, że jego pra-pra-babka, Ann Tillinghast Potter, należała do wspólnoty Episkopalnej.

Wkrótce po narodzinach swojej córki, wydarzeniu, które świętował z przytupem odmiennym od swojej zwykłej oziębłości, Curwen postanowił zapozować do portretu. Został on namalowany przez bardzo utalentowanego Szkota imieniem Cosmo Alexander, które w tamtym czasie zamieszkiwał Newport, słynny od tamtego czasu jako nauczyciel Gilberta Stuarta. Podobiezna, mówili ludzie, została wykonana na drewnianym panelu ściennym w bibliotece domu w Olney Court, lecz żaden z dwóch pamiętników, które o nim wspominały, nie informował o jego ostatecznym losie. W tamtym okresie nasz kapryśny uczony wykazywał oznaki nienaturalnej abstrakcji, i spędzał tak wiele czasu, jak to tylko możliwe na swojej farmie przy Pawtuxet Road. Zdawał się on, było mówione, być w stanie ukrywanego podniecenia lub oczekiwania; tak jakby spodziewał się jakiejś fenomenalnej rzeczy lub był w przededniu ważnego odkrycia. Zdawało się, że chemia lub alchemia odgrywa w tym wielką rolę, gdyż zabierał on ze swojego domu do farmy wielką liczbę woluminów w tych właśnie tematach.

Jego zainteresowanie sprawami miasta się nie zmniejszyło, i nie tracił żadnej okazji, by pomóc liderom pokroju Stephena Hopkinsa, Josepha Browna lub Benjamina Westa w ich wysiłkach, by ubogacić kulturowo miasto, które w tamtych czasach było znacznie poniżej Newport jeśli chodzi o patronat nad sztukami wyzwolonymi. Pomógł on Danielowi Jenckesowi otworzyć jego księgarnię w 1763, i był później jej najlepszym klientem - pomógł także Gazette w kłopotach, ukazującej się każdej środy w Pod Głową Szekspira. W polityce, żarliwie wspierał gubernatora Hopkinsa przeciwko Wardom, którzy głównie siedzieli w Newport, a jego naprawdę elokwentna przemowa w Hacher's Hall w 1765 przeciwko separacji Północnego Prowidence jako osobnego miasta, stanowiła główny z powodów, dla których spadły w  mieście uprzedzenia względem niego. Ale Ezra Weeden, który obserwował go z bliska i uważnie, prychał cynicznie pod nosem na jego wszelkie aktywności życia publicznego, i jawnie przysięgał, że było to nic więcej jak ledwie maska, skrywająca pod sobą najczarniejszą z istot Tartaru. Mściwy młodzianin rozpoczął systematyczne studium mężczyzny i jego działań, kiedykolwiek był w porcie, spędzając godziny w nocy razem z łowcami wielorybów, w oczekiwaniu na światła z posiadłości Curwena, i podążając za małą łódeczką, która czasami usuwała się po cichu z zatoki. Trzymał on też rękę na pulsie jeśli chodzi o farmę Pawtuxet, co kiedyś zakończyło się poważnym pogryzieniem przez psy, które stara para Indian puściła na niego. 

\begin{center}
2
\end{center}

Do jesieni roku 1770 Weeden zadecydował, że należy działać nagle i zwrócił na siebie uwagę ciekawskich mieszczan, gdyż w powietrzu unosił się aromat niespodziewanych, nagłych wydarzeń, zrzucony niczym stary płaszcz, dając miejsce trudno ukrywanemu wyniesieniu perfekcyjnego triumfu. Curwen wydawał się mieć problem z powstrzymaniem siebie samego przed publicznymi przemówieniami o tym, co on odkrył lub czego się dowiedział, lecz najwyraźniej, potrzeba utrzymanie owej rzeczy w tajemnicy była większa niż jego pragnienie, ażeby się pochwalić odkryciem, gdyż nigdy nie zaoferował on żadnego wyjaśnienia. To po tych wydarzeniach, które najpewniej miały miejsce we wczesnym lipcu, złowróżebny badacz zaczął zaskakiwać ludzi posiadaniem informacji, które tylko ich dawno zmarli przodkowie mogli posiadać.

Lecz sekretne aktywności Curwena w żadnym wypadku nie zostały wstrzymane wraz z tą zmianą. Wprost przeciwnie, raczej się one zwiększyły, tak, że więcej i więcej jego biznesów pozostawało w rękach kapitanów statków, którzy byli teraz z nim związani zarówno więzami strachu, równie potężnymi jak i wizją bankructwa. Porzucił on w zupełności handel niewolnikami, twierdząc, że przychody z niego ciągle się zmniejszały. Każdą wolną chwilę spędzał on na farmie w Pawtuxet, choć tu i ówdzie pojawiały się pogłoski o jego obecności w miejscach, które, choć nie były technicznie w pobliżu cmentarzy, to były z nimi powiązane na tyle mocno, iż ludzie zaczęli się dziwić, czy aby stary kupiec nie powrócił do swoich starych nawyków - a może po prostu nigdy z nimi nie zerwał? Ezra Weeden, choć szpiegował go w krótkich okresach, ze względu na swoje morskie podróże, śledził go z mściwą wytrzymałością, której większość mieszczan i rolników nie posiadało, i badał sprawy Curwena z dokładnością, z którą nigdy wcześniej nie były one badane. 

Wiele dziwacznych manewrów statków kupca można było wytłumaczyć niespokojnością owych czasów, w których każdy kolonista starał się oprzeć założeniom Sugar Act, co skutkowało problemami w transporcie. Smuglowanie i unikanie zadomowiły się dobrze w Narragansett Bay, a nocne cumowania nielegalnych ładunków były bardzo upowszechnione. Ale Weeden, noc po nocy, podążał za światłami statków i małych łodzi, które wracały od magazynów Curwena przy docu Town Street, wkrótce też zdał sobie sprawę z tego, że nie tylko statki Jego Królewskiej Mości były tym, czego chciał uniknąć Curwen. Przed zmianą z 1766, te łodzie najczęściej zbierały skutych Murzynów, którzy byli transportowani poprzez zatokę i lądowali w mało znanym miejscu tuż na północ od Pawtuxet, skąd przemieszczali się do farmy Curwena, gdzie ich zamykano w ogromnym kamiennym budynku, który posiadał wąskie szczeliny zamiast okien. Potem jednak, cały program zmieniono. Zaprzestano importu niewolników, i przez pewien czas Curwen osierocił swe transporty o północy. Wtedy, około wiosny 1767, nowa polityka weszła w życie. Po raz kolejny jego statki opuściły ciche, ciemne doki, i tym razem opuściły zatokę, unosząc się na morzu w pewnym dystansie, może nawet na wysokości Nanquit Point, gdzie miały się spotkać z dziwnymi statki wielkiego rozmiaru i różnorodnego wyglądu, ażeby odebrać od nich ładunki. Następnie, marynarze Curwena mieli zdeponować te ładunki w zwykłym miejscu na wybrzeżu, i przetransportować je lądem do jego farmy, zamykając je w tym samym tajemniczym budynku z kamienia, w którym wcześniej przebywali Murzyni. Ładunki te prawie w całości składały się z pudeł i pudełek, z których duża część była wielka i ciężka, oraz zaskakująco przypominająca wyglądem trumny.

Weeden zawsze obserwował farmę z nieustanną pilnością, odwiedzając ją każdej nocy przez dłuższy czas, i rzadko kiedy pozwalając, by minął tydzień bez jego obserwacji, z wyjątkiem dni, gdy grunt był pokryty śniegiem ukazującym ślady piechura. Nawet wtedy jednak, często podchodził tak blisko, jak to tylko możliwe, korzystając z dobrze uczęszczanych dróg lub lodu na pobliskiej rzece, by patrzeć na ślady, które inni mogli zostawić. Odkrywszy, że jego własne śledztwo staje na drodze jego morskich obowiązków, zatrudnił on lokalnego bywalca karczmy, imieniem Eleazar Smith, aby kontynuował badania, gdy zleceniodawca przebywał daleko. Obydwoje mogliby puścić w głos pewne niezwykle dzikie pogłoski. Nie zrobili tego tylko i wyłącznie dlatego, że znali efekt, jaki rozgłos mógłby wywołać na ich ofierze i uczynić dalszy postęp śledztwa niemożliwym. Zamiast tego, chcieli się dowiedzieć czegoś konkretnego zanim by zaczęli działać. To, czego się dowiedzieli, musiało zaiste być przerażające, i Charles Ward mówił wiele razy swoim rodzicom o swoim żalu z powodu tego, że Weeden później spalił swoje notatniki. Wszystko, co można powiedzieć o ich odkryciach to to, co Eleazar Smith zanotował w swoim niezbyt rozsądnym dzienniku,  i co inni pisarze pamiętników i listów skromnie powtarzali, gdy już pewne słowa się rozniosły - i zgodnie z którymi farma była tylko zewnętrzną skorupą jakiejś szeroko rozpowszechnionej i obrzydliwej okropności, której zasięg i głębia była zbyt ważna i nieuchwytna, by pozyskać coś więcej niż iluzję zrozumienia. 

Udało się potwierdzić, że Weeden i Smith byli przekonani, że obszerna sieć tuneli i katakumb znajdowała się pod farmą, i była zamieszkana przez rozlicznych ludzi, poza starą parą Indian. Dom ten był starym reliktem środka XVII wieku z pokaźnym kominem i oknami z kratami w kształcie diamentów, a laboratorium wysuwało się ku północy, w miejscu, gdzie grunt prawie stykał się z dachem. Ten budynek zdawał się przypominać każdy inny, lecz zważywszy na różne głosy słyszana czasami w jego wnętrzu, z pewnością musiał do niego prowadzić sekretny, podziemny korytarz. Te głosy, przed 1766, były ledwie szeptami i majakami Murzynów, niekiedy przechodzącymi w paniczne wrzaski, przemieszane z intrygującymi inwokacjami i modłami. Po tej dacie, jednakże, przyjęły one formę bardzo dziwnego i okropnego dźwięku, mieszaniny tępego przyzwolenia i wybuchów szaleńczej furii, wraz z konwersacjami i sapaniami oraz okrzykami protestu. Wydawało się, że mówione były różne języki, wszystkie znane Curwenowi, których akcenty były często nierozróżnialne - odpowiedzi, potwierdzenia czy groźby?

Czasami zdawało się, że parę osób musi być w domu: Curwen, pewni pojmani i strażnicy owych pojmanych. Były to głosy, których ani Weeden, ani Smith nigdy wcześniej nie słyszeli pomimo ich rozległej wiedzy o obcych portach, i wielu zdawało się należeć do tej czy innej narodowości. Natura tych konwersacji zawsze zdawała się być rodzajem katechizmu, tak jakby Curwen odzyskiwał pewne informacje od przerażonych lub buntowniczych więźniów. 

Weeden posiadał wiele dosłownych raportów z podsłuchanych fragmentów wypowiedzi w swoich notatkach, mówionych po angielsku, francusku i hiszpańsku, które to języki znał i które były często używane - lecz z owych notatek nic się nie ostało. Jednakże, powiedział on, że poza paroma grobowymi dialogami dotyczącymi przeszłych wydarzeń w życiach rodzin z Providence, większość pytań i odpowiedzi, które mógł zrozumieć, było historycznych lub naukowych; czasami dotyczyły one bardzo odległych miejsc lub czasów. Przy jednej okazji, przykładowo, na zmianę wściekła i pokorna osoba były przesłuchiwana po francusku o masakrę Czarnego Księcia w Limoges w 1370, zupełnie jakby był w tej historii jakiś sekret, który powinna ona znać. Curwen zapytał więźnia - jeśli był on więźniem - o to, czy rozkaz zabicia był wydany ze względu na Znak Kozy znaleziony na ołtarzu w antycznej rzymskiej krypcie pod katedrą, lub czy Czarny Człowiek z kowenu Haute w Wiedniu wymówił Trzy Słowa. Nie mogąc uzyskać odpowiedzi, inkwizytor najwyraźniej posunął się do ekstremalnych środków, gdyż rozległ się przerażający wrzask, a następnie cisza, szeptanie i odgłosy uderzania.

Żadne z tych rozmów nie zostały nigdy wizualnie potwierdzone, gdyż zasłony w oknach były zawsze ciężko zasunięte. Jednak raz, podczas rozmowy w nieznanym języku, można było dostrzec zza zasłon cień w oknie, który przeraził Weedera znacząco. Przypominał mu on o marionetkach w teatrzyku dla lalek, który zaobserwował na jesień 1764 w Hacher's Hall, kiedy mężczyzna z Germantown w Pensylwanii zaprezentował sprytny mechaniczny spektakl reklamowany jako ``Widok na Słynne Miasto Jerozolimę,, w którym można dostrzec Jerozolimę, Świątynię Salomona, Królewski Tron, ważne Wieże i Wzgórza, jak i Cierpienie Naszego Zbawcy z Ogrodu w Gethsemane na Krzyżu na wzgórzu Golgota, prawdziwie artystyczny Pomnik. Warte zobaczenia przez Ciekawych Świata''. To przy tamtej sposobności podsłuchiwacz, który zakradł się blisko okna z przodu, gdzie miała miejsce rozmowa, został przerażony przez parę starych Indian, którzy puścili na niego swoje psy. Po tym wydarzeniu nie było już słychać bluźnierczych konwersacji w domu, a Weeden i Smith uznali, że Curwen musiał przenieść swoje operacje w podziemne regiony domostwa.

To, że takie regiony musiały po prawdzie istnieć, wydawało się oczywiste z wielu powodów. Odległe krzyki i jęki dochodziły tu i ówdzie ze, zdawałoby się, solidniej ziemi pod stopami, w miejscach odległych od jakichkolwiek budowli. Jednocześnie, ukryta pośród krzaków przy rzece z tyłu, gdzie wysoki grunt przechodził w dolinę Pawtuxet, znajdowała się drewniana brama, którą, w momencie odkrycia, uznano za będącą wejściem do jaskiń pod wzgórzem. Kiedy i jak owe katakumby zostały wzniesione, Weeden nie mógł powiedzieć, ale często wskazywał na to, jak łatwo owe miejsce mogłoby zostać dosięgnięte przez bandy niewidzialnych pracowników rzecznych. Joseph Curwen pokierował swoich morskich przyjaciół do różnych celów, zaiste! Podczas ciężkich deszczy roku 1769 dwójka obserwatorów trzymała oko na rzece by sprawdzić, czy jakiekolwiek podziemne sekrety ujrzą światło dzienne, i nagrodzono ich, gdy zobaczyli zarówno kości zwierzęce, jak i ludzkie na brzegu rzeki. Oczywiście, mogło to mieć wiele wyjaśnień, w pobliżu farmy i starego cmentarza Indiańskiego, ale Weeden i Smith doszli do swoich własnych konkluzji. 

Był styczeń 1770, gdy Weeden i Smith dalej próżno debatowali co, jeśli cokolwiek, myśleć lub zrobić odnośnie całej tej zastanawiającej sprawy, kiedy nastąpił wypadek z Fortalezą. Rozczarowani z powodu utraty swojego zysku ze statku Liberty w Newport poprzedniego lata, flota celna dowodzona przez Admirała Wallaca zaadoptowała politykę uważniejszej kontroli odnośnie dziwnych statków, i z tej okazji uzbrojony szkuner Jego Wysokości, nazwany Cygnet, pod dowództwem kapitana Harry'ego Leshe'a, pochwycił, po krótkim pościgu pewnego poranka, statek Fortaleza z Barcelony w Hiszpanii, którym dowodził kapitan Manuel Arruda, a który przewoził ładunek z Cairu w Egipcie do Providence. Kiedy przeszukano go w celu znalezienia kontrabandy, statek ten okazał się przewozić wyłącznie egipskie mumie, do odbioru przez ``Marynarza A. B. C.'', który miał się pojawić i opróżnić ładownię w Nanquit Point, a którego tożsamość Kapitan Arruda czuł się związany honorem, by utrzymać w tajemnicy. Sąd Vice-Admiralski w Newport, nie wiedząc co zrobić w obliczu legalności całego transportu z jednej strony, a niepraworządną tajemniczością odbiorcy z drugiej, dokonał kompromisu i z rekomendacji Kolektora Robinsona pozwolił statkowi na wypłynięcie, ale zabronił mu dokować w wodach Rhode Island. Później niosły się plotki, że widziano go w przystani bostońskiej, choć nigdy oficjalnie nie wstąpił do portu bostońskiego.  

Ten niezwykły incydent był na językach w Providence i niewielu wątpiło w istnienie jakiegoś połączenia między transportem mumii a złowieszczym Josephem Curwenem. Jego egzotyczne badania i intrygujące eksperymenty chemiczne były wiedzą powszechną, a jego uwielbienie cmentarzy wywoływało w tym kontekście podejrzenia - nie wymagało wielkiej wyobraźni połączenie tego z wydarzeniem, które nie miało innego wyjaśnienia w mieście. Jakby świadom tego oczywistego faktu, Curwen dbał o to, by wypowiedzieć się niby mimochodem przy paru okazjach o chemicznej wartości balsamów znajdowanych w mumiach, być może sądząc, że mógłby on uczynić całe wydarzenie bardziej naturalnym i nieszkodliwym, jednakże zostało to przez wielu odebrane jako przyznanie się do winy. Weeden i Smith, oczywiście, byli przekonani o ważkości tego wydarzenia, i snuli najdziksze teorie o Curwenie i jego straszliwych sprawkach.

Następnej wiosny, jak w roku wcześniejszym, nastąpiły ciężkie ulewy, a obserwatorzy ostrożnie patrzyli na poziom wody w rzece za farmą Curwena. Wielkie sekcje ziemi zostały zmyte, i odkryto przy okazji pewną liczbę kości, ale nie istniało żadne wejrzenie w podziemnie komnaty lub kryjówki. Jednakże, jak głosiła plotka,  coś się wydarzyło w wiosce Pawtuxet, milę dalej, gdzie rzeka wpływa i spada ze skalistego tarasu, łącząc się ze spokojną, zamkniętą w lądzie pokrywą. W tym miejscu, pełnym starych domków wiejskich wspinających się na wzgórze ze staro-szkolnym mostkiem, i z rybackimi kutrami czekającymi w sennych dokach, pewien raport głosił, że pewne rzeczy płynęły w dół rzeki i ukazywały się oczom ludzi przez minutkę, gdy spadały wraz z wodospadem. Oczywiście, Pawtuxet to długa rzeka która płynie przez wiele osiedli ludzkich, gdzie istnieją także cmentarze, i oczywiście, wiosenne ulewy były bardzo intensywne, ale rybacy przy moście nie lubili dzikiego spojrzenia, którym jedno z ciał wpatrywało się w nich, gdy ginęło w wodzie poniżej, lub krzyku drugiego, choć jego stan zdawał się nie pozwalać na żadne krzyki. Ta plotka sprawiła, że Smith - Weeden był wtedy na morzu - ruszył szybko do brzegu rzeki za farmą, gdzie owszem, znalazł dowody zaawansowanego zapadnięcia się. Jednakże, nie było śladu dostępu do brzegu rzeki, gdyż miniaturowe lawiny zostawiły za sobą solidną ścianę mieszanki ziemi z krzakami z góry. Smith zdecydował się na eksperymentalne wykopki, ale zniechęcił go brak sukcesu - a może był to lęk przed możliwym sukcesem? Ciekawie jest zastanowić się, co wytrwały w swych dążeniach Weeden zrobiłby na jego miejscu, gdyby nie znajdował się na statku.

\begin{center}
3
\end{center}

W momencie, gdy jesień 1770 się rozpoczęła, Weeden zadecydował, że nadszedł czas, by poinformować innych o swoich odkryciach, gdyż zgromadził sporą liczbę faktów, a także posiadał drugiego świadka, gdyby pojawiły się oskarżenia, że wszystko wymyślił, kierowany zwykłą zazdrością. Jako pierwszą osobę, do której poszedł, wybrał Kapitana Jamesa Mathewsona, ze statku Enterprise, który z jednej strony znał go dostatecznie dobrze, by nie wątpić w jego słowa, a z drugiej był dostatecznie wpływowy w mieście, by zostać wysłuchanym z szacunkiem. Spotkanie odbyło się w górnym pokoju tawerny Sabina w pobliżu doków, z obecnym Smithem by potwierdzić każdą z opowieści - i można było dostrzec, że Kapitan Matthewson był pod wielkim wrażeniem. Jak prawie każda inna osoba w mieście, posiadał własne przemyślenia w temacie Josepha Curwena; tak więc potrzebował tylko tego potwierdzenia i większego zasobu informacji, by przekonać go absolutnie. Na końcu konferencji, był bardzo poważny, i poprosił resztę mężczyzn o zachowanie milczenia. Miał on, wyjaśnił, przekazać te informacje osobno około 10 najbardziej uczonym i ważnym osobom w Providence, zaznać ich porady i poznać ich poglądy, a następnie posłuchać ich zdania. Sekretność była tutaj kluczowa, gdyż nie były to problemy, z którymi lokalne siły porządkowe lub sądy mogłyby sobie poradzić. Także podekscytowany tłum powinien być trzymany w ignorancji, albowiem inaczej mogło dojść do drugich wydarzeń z Salem, co miało miejsce mniej niz wiek temu, a co sprowadziło Curwena do Providence.

Odpowiednio osoby, wyjaśnił, byłyby następujące: Dr. Benjamin West, którego traktat o późnym tranzycie Wenus udowodnił, że jest on uczonym i krytycznym myślicielem; Wielebny James Manning, Prezydent Kolledżu który właśnie się przeprowadził z Warren i chwilowo zamieszkiwał przy nowym budynku szkoły na King Street, czekając, aż zakończy się budowa na wzgórzu powyżej Presbyterian Lane; ex-Gubernator Stephen Hopkins, który był członkiem Towarzystwa Filozoficznego w Newport i był mężczyzną bardzo szerokich horyzontów; John Carter, wydawcą lokalnej Gazety; wszyscy 4 z braci Brown, John, Joseph, Nicholas i Moses, którzy stanowili uznanych lokalnych możnowładców, dodatkowo, Joseph był naukowcem-amatorem; stary Dr. Jabez Bowen, którego erudycja była sporych rozmiarów, i który posiadał wiedzę z pierwszej ręki o starych zakupach Curwena; i Kapitan Abraham Whipple, żeglarz fenomenalnej odwagi i energii, który mógłby przewodzić czynnościom, jeśli zajdzie potrzeba. Ci mężczyźni, jeśli by zapragnęli, mogliby zostać wezwani razem dla kolektywnych decyzji, i na nich spoczywałaby odpowiedzialność zadecydowania, czy poinformować lub nie Gubernatora Kolonii, Josepha Wantona z Newport, przed podjęciem akcji. 

Misja Kapitana Mathewsona odniosła sukces poza jego najdzikszymi oczekiwaniami, gdyż choć jeden albo dwóch z uczonych mężów pozostało sceptycznymi odnoście opowieści Weedena, nie było ani jednego, który nie uznał za stosowne podjęcie jakiegoś rodzaju sekretnej i skoordynowanej akcji. Curwen, co było jasne, stanowił zagrożenia dla dobra miasta i Kolonii, i powinno się go wyeliminować za wszelką cenę. Późno w grudniu 1770, grupa eminentnych mieszczan spotkała się w domu Stephena Hopkinsa i dabatowała o środkach, które należało podjąć. Notatki Weedena, które ten wręczył Kapitanowi Mathewsonowi, były ostrożnie i uważnie czytane, a sam Weeden i Smith zostali wezwani, ażeby odpowiedzieć na dodatkowe pytania. Coś podobnego do strachu zaczęło kiełkować w sercach zebranych nim spotkanie się skończyło, lecz także i poważna determinacja, najlepiej wyrażona przez Kapitana Whipple'a i jego twardą osobowość. Nie zamierzali oni poinformować Gubernatora, gdyż coś innego niż akcja natury prawnej jawiła im się jako rozwiązanie problemu. Z ukrytymi mocami niejasnej potęgi do swojej dyspozycji, Curwen nie był mężczyzną, któremu można było po prostu nakazać opuszczenie miasta. Co więcej, nawet, gydby kreatura opuściła miasto, byłoby to po prostu przesunięcie problemu w inne regiony geograficzne. Żyli oni w bezprawnych czasach, a ludzie, którzy przemycali przez lata ładunki poza percepcją Jego Królewskiej Mości, raczej nie robiliby sobie problemów z prawa. Curwen musiał zostać zaskoczony na swojej farmie w Pawtuxet przez dużą grupę doświadczonych korsarzy, i dostać ostatnią szansę, ażeby się obronił od oskarżeń. Gdyby okazał się szaleńcem, zabawiającym się w wykrzykiwane, wymyślone konwersacje różnymi głosami, należało znaleźć mu miejsce w odpowiednim przybytku. Jeśli cos mroczniejszego miało miejsce, a podziemne horrory miałyby okazać się prawdziwe, on i wszyscy u jego boku powinni umrzeć. Gdyby można było to zrobić cicho, należałoby tak uczynić, tak, by nawet wdowa i jej ojciec nie znali wszystkich szczegółów. 

Kiedy debatowano o tych poważnych krokach, koniecznych do podjęcia, nastąpił w mieście incydent tak straszliwy i niewyjaśnialny, że przez pewien czas nic innego nie było wspominane w obszarze wielu mili. W środku bezksieżycowej, styczniowej nocy z ciężkimi opadami śniegu, nad rzeką i wzgórzami rozległa się seria krzyków, która sprawiły, że śpiący pojawili się przy swoich oknach, a ludzie w okolicach Waybosset Point dostrzegli wielką białą rzecz, poruszającą się gorączkowo w źle oczyszczonej przestrzeni z przodu Turk's Head. W tle rozległo się szczekanie psów, lecz szybko się zakończyło, gdy dźwięki obudzonego miasta stały się głośniejsze. Grupy mężczyzn z muszkietami i latarniami ruszyły zobaczyć, co się wydarzyło, lecz ich poszukiwania niczego nie przyniosły. Następnego poranka, jednakże, wielkie, umięśnione i nagie ciało zostało znalezione na lodzie w południowej części Wielkiego Mostu, gdzie Długi Dok rozciągał się obok destylarni Abbotta, a tożsamość zmarłego stała się tematem niekończących się spekulacji i szeptów.Co ciekawe, to starsi mieszkańcy, a nie młodsi, szeptali między sobą, gdyż tylko pośród starszyzny ta zesztywniała twarz z oczami otwartymi w przerażeniu odwoływaa się do ich pamięci. Tak więc starsi, drżąc na ciele, wymieniali zaskoczone szepty zdziwienia i przerażenia, gdyż w tych okropnych rysach znaleźli zadziwiające podobieństwo, ba, wręcz identyczność z pewną osobą - osobą, która umarła całe 50 lat wcześniej.

Ezra Weeden był obecny podczas tego znaleziska, i pamiętając ujadanie psów poprzedniej nocy, ruszył wzdłuż Weybosset Street i w kierunku mostu Muddy Dock, skąd dochodził dźwięk. Miał ciekawe przeświadczenie, i nie był zaskoczony, kiedy dotarłszy na skraj dzielnicy mieszkaniowej, gdzie ulica zmieniała się w Pawtuxet Road, natrafił na intrygujące ślady w śniegu. Nagi gigant był śledzony przez psy i wielu mężczyzn w butach, a ślady powracających psów i ich panów mogły być łatwo wyśledzone. Poddali się oni dopiero, gdy dotarli zbyt blisko miasta. Weeden uśmiechnął się ponuro, i wyśledził ślady z powrotem do ich źródła - była to farma w Pawtuxet Josepha Curwena, zupełnie, jak się tego spodziewał. Oddałby on wiele, by podwórze nie było tak konfundująco zadeptane. Zważywszy na wszystko, nie chciał być zbyt zainteresowany w świetle dnia. Dr. Bowen, do którego Weeden zaraz się udał ze swoim raportem, dokonał autopsji dziwnego ciała, i odkrył pewne dziwności, które go kompletnie skołowały. Układ trawienny wielkiego człowieka zdawał się nigdy nie być w użyciu, a cała skóra była szorstka i luźno zwisała z ciała, czego nie dało się wyjaśnić w żaden sposób. Zaintrygowany tym, co starzy ludzi szeptali o podobieństwie ciała do dawno martwego kowala imieniem Daniel Green, którego prawnuk Aaron Hoppin służył na statku Curwena, Weeden zadawał pobieżne pytania, aż nie dowiedział się, gdzie Green został pochowany. Tej nocy grupa 10 mężczyzn odwiedziła North Burying Ground po drugiej stronie Herrenden's Lane i otworzyli ten grób. Znaleźli go pustym, dokładnie tak, jak się spodziewali.

W międzyczasie, poczyniono przygotowania i przechwycono korespondencją Josepha Curwena. Tuż przed inducentem z nagim ciałem, odnaleziono list od pewnego człowieka imieniem Jedediah Orne z Salem, który zmusił cooperujących obywateli do głębokiej zadumy. Część tego listu, skopiowana i przechowana w prywatnych archiwach rodziny, gdzie został znaleziony przez Charlesa Warda, brzmiał następująco: 

\begin{displayquote}

``Cieszy mnie, że kontynuujesz Swoje zainteresowanie Starymi Sprawami, lecz nie myśl, że inaczej było z Panem Hutchinsonem w wiosce Salem. Z pewnością, nie było tam Nic tylko najgorsze Okropności w tym, jak H. zebrał to, czego tylko część zebraliśmy my. To, co zrobiłeś nie Zadziałało, albo ponieważ Jakaś Rzecz zawiodła, albo ponieważ Twe Słowa nie były Dobrze Wymówione lub skopiowane. Sam nie Wiem. Nie posiadam Chemicznej wiedzy, by podążać za Borelliusem, i sam jestem skonfundowany przez Tę Księgę Nekronomiconu, którą polecasz. Ale proszę, Zauważ, że zwróciliśmy Ci Uwagę, żebyś Uważał kogo wzywasz, pamiętaj, co Pan Mather napisał w swych Uwagach ———, i zważ na to, jak ta Okropna rzecz jest raportowana. Mówię Ci ponownie, nie wzywaj Niczego, czego nie możesz odesłać, przez co mam na Myśli, Każdego, Kto może z Powrotem powstać przeciwko Tobie, gdy Twoje Potężne Urządzenie nie mogą być użyte. Pytaj o Pomniejszych, gdyż Wielcy nie będą chcieli Odpowiedzieć, i mogą mieć władzę większą od Twojej. Byłem przerażony, gdy przeczytałem, że wiesz co Ben Zaristnatmik zostawił w swym Hebanowym Podle, gdyż wiedziałem, kto musiał Ci to powiedzieć. I ponownie proszę, byś pisał do mnie per ``Jadediah'' a nie ``Simon''. W tej społeczności mężczyzna nie może żyć zbyt długo, a Ty znasz mój Plan, zgodnie z którym wróciłem jako mój Syn. Nie mogę się doczekać, aż mi Powiesz co ten Czarny Mężczyzna powiedział Ci o Sylvanusie Cocodiciusie w Twojej krypcie, pod Rzymskim murem, i będę zobligowanym, jeśli mi Pożyczysz tego MS. o którym tyle mówiłeś.''

\end{displayquote}

Inny list, niepodpisany, wysłany z Filadelfii, wywołał podobne zamieszania, zwłaszcza następujący fragment:

\begin{displayquote}

``Będę słuchał tego, co masz do Powiedzenia, ale tylko przez Twoich Posłańców, ale nie zawsze jestem pewien, kiedy się ich spodziewać. W Kwestii o której już mówiliśmy, proszę o tylko jedną rzecz więcej, lecz chcę do Ciebie podejść odpowiednio. Mówisz mi, że żadna Część nie może być zgubiona jeśli najlepszy Efekt ma się wydarzyć, ale nie możesz wiedzieć, jak ciężko być pewnym. Wydaje się to być wielkim Ryzykiem i Ciężarem, wziąć całe Pudło, a w Mieście (np: St. Peter's, St. Paul's, St. Mary, lub Christ Church) ciężko to zrobić w ogóle. Lecz znam wszystkie imperfekcje, które podniosłeś w ostatnim październiku, i jak wiele żywych Osobników musiałeś wykorzystać, by wejść w odpowiedni Stan w 1766, więc będę Cię słuchał we wszelkich Kwestiach. Nie mogę się doczekać Twojego Statku, i pytam oń co dzień w Mr. Biddle's Wharf''

\end{displayquote}

Trzeci podejrzany list był napisany w nieznanym języku, a nawet nieznanym alfabecie. W pamiętniku Smitha znalezionym przez Charlesa Warde, jedna często powtarzana kombinacja znaków została niezdarnie skopiowana; naukowcy w Brown University określili ten alfabet jako amharski lub abisyński, ale nawet oni nie rozpoznają tego słowa. Żadna z tych wiadomości nie została nigdy dostarczona do Curwena, lecz zaginięcie Jedediaha Orne'a z Salem została zanotowano jako nieco późniejsze wydarzenie, co pokazuje, że mężczyźni z Providence podjęli pewne ciche kroki. Towarzystwo Historyczne Pensylwanii także posiada pewien ciekawy list otrzymany przez Dr. Shippena odnośnie obecności niebezpiecznej postaci w Filadelfii.
Lecz ważniejsze kroki dopiero miały zostać podjęte, przez zgromadzenie zaprzysiężonych i przetestowanych marynarzy i korsarzy z magazynu w Bostonie w środku nocy - to był główny skutek odkryć Weedena. Powoli, lecz stanowczo, plan kampanii powstawał w sekrecie, plan kampanii która miała zostawić brak jakiegokolwiek śladu po tajemnicach Josepha Curwena. 

Curwen, pomimo wszystkich obostrzeń, najwyraźniej czuł, że coś się szykuje, gdyż wyglądał teraz na wielce zmartwionego. Jego powóz był widziany o wszelkich godzinach w mieście i na Pawtuxet Road, i trochę po trochu utracił swoją aurę wymuszonej dobrotliwości, którą starał się zwalczyć uprzedzenia mieszczan. Najbliżsi sąsiedzi jego farmy, Fannersowie, pewnej nocy zauważyli wielkie światło strzelające ku niebu z jakieś aparatury na szczycie tego tajemniczego kamiennego budynku z wysokimi, bardzo wąskimi oknami - wydarzenie to szybko zakomunikowali Johnemu Brownowi z Providence. Pan Brown był liderem wyselekcjonowanej grupy ludzi chętnych, by pokonać Curwena, i poinformował Fennerów, że jakaś akcja zostanie podjęta. Uznał to za konieczne, ze względu na niemożliwość, by nie dostrzegli ostatecznego napadu; i uzasadnił to tym, iż Curwen miał być rzekomo szpiegiem celników z Newport, przeciwko którym ręka każdego marynarza w Providence, handlarza i rolnika była otwarcie lub skrycie wzniesiona. Nie wiadomo, czy sąsiedzi Curwena w pełni uwierzyli w zapewnienia, gdyż widzieli wiele dziwnych rzeczy, ale w każdym wypadku Fennerowie byli chętni połączyć koncept zła z mężczyzną o tak wielu dziwnych przypadłościach. To im Pan Brown powierzył obowiązek obserwacji domostwa Curwena na farmie, i regularnego raportowania wszelkich dziwnych incydentów, które miały mieć miejsce. 

\begin{center}

4

\end{center}

Możliwość, że Curwen miał się na baczność i próbował swoich sztuczek, jak sugerował dziwny promień światła, przyspieszyła akcję tak ostrożnie przygotowywaną przez grupę zaniepokojonych obywateli. Zgodnie z pamiętnikiem Smitha, grupa około 100 osób spotkała się o 10 wieczorem w piątek, 20 kwietnia 1771 r. w wielkiej sali Tawerny Thurstona Pod Szyldem Złotego Lwa w Weybosset Point za mostem. Jeśli chodzi o prominentnych mieszkańców Providence, poza ich liderem Johnem Brownem, zgromadzeni byli Dr. Bowen, z jego teczką instrumentów chirurgicznych, Prezydent Manning bez swojej wielkiej peruki (największej w Koloniach) z której był znany, Gubernator Hopkins, z założonym czarnym płaszczem i w towarzystwie podróżującego po morzach brata Eseha, którego włączył do spiskowców w ostatniej chwili, przy zgodzie reszty, John Carter, Kapitan Mathewson i Kapitan Whipple, który miał przewodzić sile zbrojnej. Ci liderzy naradzali się w odosobnionym pomieszczeniu na tyłach karczmy, po czym kapitan Whipple przemieścił się do dużej komnaty i dał zgromadzonym marynarzom ostatnie instrukcje i porady. Eleazar Smith był z liderami, gdy Ci siedzieli w tylnym pomieszczeniu czekając na przybycie Erzry Weedena, którego obowiązkiem było śledzenie Curwena i poinformowanie, gdy jego dyliżans miał opuścić farmę. 

Około 10:30 ciężki stukot kopyt i kół rozległ się na Wielkim Moście, i o tej godzinie nie trzeba było czekać na Weedena, by wiedzieć, że przeklęty mąż rozpoczął ostatnią noc swego bluźnierczego czarnoksięstwa. W chwilę potem, w miarę, jak dyliżans przedostawał się przez Most Muddy Rock, Weeden się pojawił - konspiratorzy zamilkli w bojowym nastroju, po czym wyszli na ulicę, przygotowując swoje harpuny i broń palną. Weeden i Smith byli z grupą, a wraz z nimi Kapitan Whipple, który był liderem, jak i Eseh Hopkins, John Carter, Prizydent Manning, Kapitan Mathewson, i Dr. Bowen, razem z Mosesem Brownem, który przyszedł o 11 wieczór, choć nie był obecny w karczmie. Wszyscy ci ludzie, wraz z setką marynarzy, zaczęli maszerować bez opóźnień, poważni i przerażeni w głębi swym serc, gdy opuścili Muddy Rock i zaczęli się kierować poprzez Broad Street w kierunku Pawtuxet Road. Tuż za kościołem Eldera Snowa niektórzy z mężczyzn odwrócili się, by spojrzeć na Providence za ich plecami, oświetlane gwiazdami wczesnej wiosny. Wieże i szczyty miasta wzrastały ciemnokształtne w półmroku, a słona bryza wiała z zatoczki na północ od Mostu. Vega powoli się wspinała na niebie powyżej wielkiego wzgórza przy wodzie, a herb drzewny był załamany na linii dachu przez niedokończony budynek Koledżu. Przy podstawie owego wzgórza i wzdłuż wąskich uliczek stare miasto śniło - Stare Providence, dla którego bezpieczeństwa i poczytalności tak wielkie i bluźniercze zło miało zostać wymazane raz na zawsze.

Około godziny i kwadransa później, zbrojna grupa, jak wcześniej ustalono, przybyła do farmy Fennerów, gdzie wysłuchali ostatniego raportu o swojej ofierze. Dotarł on do swojej farmy ponad pół godziny wcześniej, a dziwne światło wkrótce potem znów wystrzeliło ku niebiosom, ale nie było świateł w żadnych oknach. Ostatnio często to się powtarzało. Gdy te wiadomości zostały przekazane, kolejny wielki błysk rozległ się na południu, i grupa zdała sobie sprawę, że zaiste, znaleźli się bliżej sceny nienaturalnych, wspaniałych cudów. Kapitan Whipple wydał rozkaz, aby jego siły podzieliły się na trzy dywizje: jedna z 20 mężczyznami pod dowództwem Eleazera Smitha miała baczyć na wybrzeże i chronić wszystkich przed możliwymi morskimi posiłkami dla Curwena, chyba, że przybyłby posłaniec w chwili wielkiej trwogi; druga grupa 20 ludzi pod dowództwem Kapitana Eseha Hopkinsa miała wejść do doliny rzecznej za domem Curwena i zdemolować siekierami i prochem strzelniczym dębowe wrota tuż przy rzece; trzecie grupa zaś miała zaatakować dom i poboczne budynki. Z tej dywizji, 1/3 była dowodzona przez Kapitana Mathewsona (ci mieli spenetrować tajemniczy kamienny dom z wysokimi, wąskimi oknami), kolejna 1/3 pod dowództwem samego Kapitana Whipple'a miała uderzyć w główną farmę, zaś ostatnie 1/3 miała utworzyć krąg wokół wszystkich budynków dopóki nie zostałby im wysłany ostateczny sygnał. 

Grupa przy rzece miała za zadanie przebić się przez drzwi we wzgórzu na dźwięk pojedynczego gwizdnięcia, czekając i przechwytując wszystko, co mogło się poruszyć z głębszych rejonów. Na dźwięk dwóch gwizdnięć. grupa ta miała ruszyć wgłąb otworu, by przeciwstawić się wrogowi lub połączyć z resztą kontyngentu. Grupa przy kamiennym budynku miała działać na te sygnały w analogiczny sposób - najpierw wyważając drzwi, a potem wchodząc do środka jakimkolwiek przejściem, które zamierzali odkryć i dołączając do rozproszonej lub skupionej walki wewnątrz podziemi. Trzeci sygnał trzech gwizdnięć miał za zadanie wezwać na pomoc rezerwę otaczającą miejsce pierścieniem - jej 20 ludzi miała się podzielić po równo i ruszyć obydwoma wejściami wgłąb farmy i kamiennego budynku. Wiara kapitana Whipple'a w istnienie katakumb była absolutna i nie rozważał żadnej innej opcji, gdy obmyślał swój plan. Miał ze sobą gwizdek wielkiej mocy i nie bał się pomyłki co do jego dźwięku. Ostateczna rezerwa była, oczywiście, prawie poza zasięgiem gwizdka, tak więc korzystanie z niego wymagałoby specjalnego posłańca, Moses Brown i John Carter dołączyli do Kapitana Hopkinsa przy brzegu rzeki, a Prezydent Manning czekał przy kamiennej budowli z Kapitanem Mathewsonem. Dr Bowen, z Erzą Weedenem, pozostał w grupie Kapitana Whipple'a, która miała za zadanie wejść w głąb głównego budynku farmy. Atak miał się rozpocząć tak szybko, jak posłaniec od Kapitana Hopkinsa miał dołączyć do Kapitana Whipple'a, by poinformować go o gotowości grupy przy rzece. Wtedy to lider miał wydać jeden pojedynczy dźwięk, a poszczególne grupy miały zaatakować z wielu stron jednocześnie, w trzech punktach. Tuż przed 1 w nocy, trzy dywizje opuściły farmę Fennerów, jedna by czekać przy rzece, druga przy drzwiach ze wzgórza wychodzących na rzekę, trzecia zaś by się podzielić i wejść w główne budynki farmy Curwena. 

Eleazer Smith, który towarzyszył grupie pilnującej brzegu, zanotował w swoim pamiętniku marsz bez żadnych niespodzianek i długie czekanie potem, w końcu przełamane czymś, co zdawało się być odległym dźwiękiem gwizdka, a potem przez zastanawiające pomieszanie ryczenia i płakania, przytłumione z oddali i dźwięk eksplozji prochu z, zdawało się, tego samego kierunku. Później jeden z mężczyzn wierzył, że usłyszał odległą strzelaninę, a jeszcze później Smith poczuł w sobie tytaniczne, dźwięczące słowa, unoszące się ponad nim w powietrzu. Tuż przed nastaniem poranka pojedynczy posłaniec z dzikim wzrokiem i okropnych, ciężkim do nazwania odorem własnych ubrań pojawił się i powiedział, by wszyscy się rozeszli i nigdy więcej nie myśleli lub mówili o wydarzeniach tej nocy lub o tym, kto był Josephem Curwenem. Coś odnośnie wyglądu i zachowania posłańca nosiło w sobie przekonanie, którego słowa nie dały rady wyrazić, ponieważ choć był on człowiekiem morza znanym wielu z nich, było odnośnie jego osoby coś zgubionego lub też, być może, pozyskanego, co od tej pory miało na zawsze zmienić jego psyche. Tak samo było z resztą kompanów, którzy udali się do tej strefy horroru. Większość z nich zyskała lub utraciła coś nienazwanego i nieokreślonego. Ujrzeli, usłyszeli lub poczuli coś, co nie było stworzone z myślą o ludzkich istotach, i nie mogli tego zapomnieć. Od tego momentu już nie plotkowano, gdyż nawet najzwyklejszy ze śmiertelnych instynktów wyczuwał tutaj pewną piekielną granicę. I od tego pojedynczego posłańca, gruba na brzegu rzeki odczuła nienazwany lęk, który praktycznie zamknął mu usta. Bardzo mało plotek pochodziło od tych ludzi, a pamiętnik Eleazera Smitha jest jedynym zapiskiem, który przetrwał tą ekspedycję, która wyruszyła z karczmy Pod Złotym Lwem pod Gwiazdami. 

Charles Ward, jednakże, odkrył inną ogólnikową relację, w korespondencji Fennerów, którą znalazł w Nowym Londynie, gdzie żyła pewna odnoga jago rodziny. Wyglądało na to, że Fennerowie, ze swojego domu, z którego było widać w oddali farmę Curwena, oglądali wyruszające kolumny spiskowców i słyszeli wyraźnie wściekłe ujadanie psów Curwena, po czym nastąpił dźwięk oznaczający atak. Po pierwszym znaku do ataku pojawiło się ponownie dziwne światło z kamiennego budynku, a w następnej chwili, po szybkim wydaniu drugiego znaku do ataku, nastąpiły przytłumione odgłosy strzelania z muszkietów , po których nastąpił ryk płaczu, który korespondent, Luke Fenner, określił onomatopeją ``Waaaaaahhhhrrrr-R'waaaahrrrr''. Ten krzyk, jednakże, posiadał pewną właściwość, której nie dało się przekazać na piśmie, i korespondent wspomina, że jego matka straciła przytomność, usłyszawszy go. Później go powtórzono, ciszej tym razem, i dalej, stłumiony przez odgłosy zapalanego prochu i strzelaniny - a także dźwięku głośnej eksplozji z kierunku rzeki. Około godziny później, wszystkie psy zaczęły jęczeć bojaźliwie, i spod ziemi zaczęły się rozpościerać dźwięki i wibracje, tak potężne, że zaczęły gasnąć świecie w świecznikach. Odnotowano potężny aromat siarki, a ojciec Luke Fennera powiedział, że usłyszał trzeci sygnał gwizdka, ale inni nic nie słyszeli. Przytłumione odgłosy wystrzałów nastąpiły ponownie, a później głęboki krzyk, mniej penetrujący uszy, lecz przez to bardziej przerażający niż poprzednie dźwięki - rodzaj gardłowego, niedającego się pokonać kaszlu lub gulgotania, którego kwalifikacji jako krzyk bardziej wynikała z ciągłości i psychologicznej ważkości, niż z zasadniczej wartości akustycznej.

Wtedy, ognista rzecz słupem powołała się do życia w miejscu, gdzie znajdowała się farma Curwena, a ludzkie krzyki desperacji i strachu poniosły się w powietrzu. Muszkiety błyskały i pękały, a ognista rzecz spadła na ziemię. Druga płomienista rzecz pojawiła się, i krzyki ludzkiego płaczu były jasno dostrzegalne. Fenner napisał, że mógł nawet rozróżnić parę panicznych słów: ``Boże Wszechmocny, chroń owcę!''. Pojawiło się więcej wystrzałów, i druga płonąca rzecz upadła. Po tym nastąpiła cisza przez około 3/4 godziny, na końcu której mały Arthur Fenner, brak Luke'a, stwierdził, że ujrzał ``czerwoną mgłę'' wspinającą się do gwiazd z przeklętej farmy w dystansie. Nikt poza dzieckiem nie może tego potwierdzić, ale Luke przyznaje, że miał miejsce znaczący zbieg okoliczności, zasugerowany przez panikę kompulsywnego strachu, który w tym samym momencie postawił włos na głowie trzem kocurom w pomieszczeniu. 

Pięć minut później zawiał chłodny wiatr, a powietrze wypełniło się tak nieznośnym smrodem, iż tylko silna świeżość napływająca ze strony morza mogłaby sprawić, iż nie zostałby dostrzeżony przez atakujących ludzi lub potencjalnie obudzonych ludzi w wiosce Pawtuxet. Ten odór był kompletnie nowym doświadczeniem dla Fennerów, i wywołał drapiący w gardło, amorficzny strach przypominający grób lub grabarza. Tuż po nim rozległ się okropny głos, którego żaden nieszczęśliwy świadek nie będzie w stanie nigdy zapomnieć.  Grzmiał on z niebios niczym posłaniec zagłady, a okna trzęsły się w ramach, gdy jego echo ginęło. Był on głęboki i muzykalny, potężny jak organy kościelne, lecz złowróżebny niczym zakazane arabskie księgi. Zaden z ludzi nie wie, co ów głos mówił, gdyż przemawiał on w nieznanym języku. Jest to zapis, który Luke Fenner spisał, odnośnie brzmienia demonicznych inkantacji: ``DEESMEES—JESHET—BONEDOSEFEDUVEMA—ENTTEMOSS''. Dopiero w roku 1919 ludzka dusza połączyła ten zapis z czymkolwiek w wiedzy śmiertelników, gdyż wtedy Charles Ward, blady na licu, rozpoznał w tym to, co Mirandola zapisał, drżącą dłonią, o ostatecznym horrorze posród inkantacji czarnej magii.

Krzyk, z pewnościa ludzki, odpowiedział na bluźnierczy cud nad farmą Curwena, po czym nieznany odór stał się bardziej zlożony, przy czym dodatkowe wonie były równie ohydne. Płacz, wyraźnie odmienny od krzyku, rozległ się w tamtym momencie, raz się wnosząc, to znów upadając. Czasami był prawie rozpoznawalny, lecz żaden ze słuchaczy nie mógł określić żadnych słów, w pewnym momencie zaś zaczął przypominać on diaboliczny, histeryczny śmiech. Następnie rozległ się krzyk ostatecznego, pierwotnego strachu, i szaleństwo rozległo się, głoszone przez ludzkie gardła, krzyk silny i czysty pomimo głębi, z którą musiał wybrzmieć, po czym ciemność i cisza rządziły resztą nocy. Spirale czarnego dymu zasnuły niebo, blokując gwiazdy, choć nikt nie dostrzegł żadnego ognia, a budynki wydawały się być bez żadnego szwanku lub krzywdy w świetle następnego dnia.

Blisko świtu, dwóch przerażonych posłańców z potwornymi, nienazwanymi odorami, które przeniknęły głęboko w ich ubrania, zapukało do Fennerów i poprosili o rum, za którzy zapłacili zaiste hojnie. Jeden z nich powiedział rodzinie, że sprawa z Josephem Curwenem była definitywnie skończona, i by nie wspominali więcej o wydarzeniach minionej nocy. Choć rozkaz wydawał sią arogancki, osoba, która go wydała, zdawała się wyglądać na przerażającą, i wydała go z niewątpliwym autorytetem, tak, że te listy Luke'a Fennera, które przetrwały, zawierały prośbę do jego krewnego w Connecticut, aby zostały zniszczone, jednakże, tak się nie stało. Nieposłuszeństwo krewnego, dzięki któremu te listy przetrwały do czasów współczesnych, uchowały je od litościwego zapomnienia. Charles Ward miał jeden detal, który pragnął dodać, jako efekt swoich długich poszukiwań. Stary Charles Slocum z tej wioski powiedział, że była znana jego dziadkowi pewna dziwna plotka o spalonym, zmaltretowanym ciele znalezionym na polach w tydzień po śmierci Josepha Curwena. To, co sprawiało, że plotka była głoszona, to to, że ciało, na tyle, na ile można było je dostrzec w jego spalonej, wykrzywionej formie, nie było ani w pełni ludzkie, ani nie należało do żadnego zwierzęcia, o którym ludzie z Pawtuxet kiedykolwiek słyszeli lub czytali.

\begin{center}
5
\end{center}

Żaden z mężczyzn, którzy brali udział w tej okropnej napaści, nigdy nie wypowiedział o nim ani słowa, a wszelkie fragmenty szczątkowych zeznań, które przetrwały, pochodzą od tych, którzy nie spenetrowali farmy Curwena. Jest coś przerażającego w dbałości, którą napastnicy wykazali się, niszcząc wszelkie zapiski, które choćby luźno wspominały o tej wyprawie. 

Ośmiu marynarzy zostało zabitych, ale choć ich ciała nie zostały zwrócone ich rodzinom, były one zadowolone informacjami, iż nastąpiły pewne problemy przy odprawie celnej. Te same wyjaśnienia padały, by wyjaśnić różnorodne rany, wszystkie szczelnie zabandażowane i leczone tylko przez Dr. Jabeza Bowena, który towarzyszył najeźdźcom. Trudniejsze do wyjaśnienia były nienazwane odory, które przyległy do wszystkich spiskowców, rzecz, o której debatowano tygodniami. Jeśli chodzi o liderów najazdu, Kapitan Whipple i Moses Brown byli najbardziej ranni, a listy ich żon wyjawiają, iż zaiste, bardzo dbali o swoją prywatność i sekretność odnośnie tego, jak wyglądały odniesione przez nich rany. Psychologicznie, wszyscy członkowie ekspedycji byli postarzeni, wstrząśnięci i bolejący. Na szczęście, wszyscy byli silnymi mężczyznami i prostymi, ortodoksyjnymi wyznawcami, gdyż z większą skłonnością do subtelnej introspekcji i mentalną zlożonością mogliby wyjść z tego w znacznie gorszym stanie. Prezydent Manning był najbardziej dotknięty - ale nawet on dał sobie radę z najciemniejszym cieniem i ukoił wspomnienia modlitwami. Każdy członek liderów miał swoją rolę do odegrania, w późniejszych latach, i jest to być może szczęśliwe. Nie dłużej niż 12 miesięcy później Kapitan Whipple przewodził tłumowi, który spalił statek Gaspee, i tym odważnym aktem poczynił kok w kierunku wymazania z pamięci i historii niepokojących obrazów. 

Do wdowy po Josephie Curwenie dostarczono zapieczętowaną ołowianą trumnę zagadkowego wyglądu, oczywiście znalezioną na miejscu w momencie potrzeby, w której, jak jej powiedziano, znajdowało się ciało jej męża. Miał on, zostało jej wyjaśnione, zostać zabitym podczas bitwy celnej, o której ze względów politycznych nie można było powiedzieć czegoś konkretniejszego. Ponad to, żaden język nie głosić plotek o końcu Josepha Curwena, a Charles Ward miał tylko jedną poszlakę, na której skonstruował swoją teorię. Tą poszlaką był najmniejszy ze śladów - drżące podkreślenie w skonfiskowanym liście Jedediaha Orne'a do Curwena, częściowo skopiowany odręcznym pismem Ezra'y Weedena. Tą kopię posiadali potomkowie Smitha, i zostaje nam zadecydować, czy Ezra dał ją swemu kompanowi po końcu, jako niemą wskazówkę co do abnormalności tego, co miało miejsce, czy może, co bardziej prawdopodobne, Smith miał ją już wcześniej, i dodał podkreślenie sam, z tego, co zdołał wydusić od swojego przyjaciela poprzez zgadywanie i zręczne przepytywanie. Podkreślenie głosiło: 

\begin{displayquote}

Przemawiam do Ciebie po Wtóre, nie Wzywaj niczego, czego nie możesz odesłać, przez co mam na myśli, Żadnego, co może wezwać coś przeciwko Tobie, gdzie Twoje najpoteżniejsze Urządzenia nie mogą być w użyciu. Pytaj o Pomniejszych, gdyż Najwięksi nie będą pragnęli Odpowiedzieć, I będą rządzili większymi od Ciebie.

\end{displayquote}

W świetle tej informacji, i rozważając, co za utraconych, niewspominanych sprzymierzeńców mógłby spróbować przywołać potępiony człowiek  w jego bezpośredniej akcji, Charles Ward mógł się zacząć zastanawiać, czy jakiś obywatel Providence nie zabił Josepha Curwena. 

Ostożne, metodyczne usunięcie każdego fragmentu pamięci o martwym człowieku z życia i notatek Providence było znacząco wspierane przez wpływowych spiskowców. Początkowo nie pragnęli być aż tak drobiazgowi, i pozwolili wdowie, jej ojcu i dziecku pozostać w ignorancji o prawdziwych wydarzeniach, ale Kapitan Tillinghast był bystrym czlowiekiem, i szybko odkrył dostatecznie dużo plotek, by horror jawił mu się wyraziście w jego oczach, i by zażądał, by jego córka i wnuczka zmieniły swe nazwisko, spalił bibliotekę i wszystkie pozostałe papierzyska, i wymazał inskrypcję na kamieniu nagrobkowym Josepha Curwena. Znał on Kapitana Whipple dobrze, i najpewniej pozyskał o niego pewne wskazówki, więcej niż ktokolwiek inny, kto przyczynił się do śmierci przeklętego czarnoksiężnika. 

Od tamtego czasu, usuwanie pamięci o Curwenie statło się znacznie poważniejsze, dotyczywszy również zapisków miejskich i plików w Gazecie. Można to porównać w duchu tylko do tego, co stało się z imieniem Oscara Wilde'a w dekadę po jego upokorzeniu, a w absolutności tylko do losu grzesznego Króla Runagura w opowieści Lorda Dunsany'ego, odnośnie którego bogowie zadecydowali, iż musi nie tylko przestać być, ale w ogóle istnieć w jakimkolwiek z czasów, wliczywszy przeszły. 

Pani Tillinghast, jak wdowę nazywano po roku 1772, sprzedała dom przy Oldney Court i zamieszkała ze swoim ojcem w Power's Lane do swojej śmierci w 1817 r. Farma w Pawtuxet, przeklęta przez wszelką żywą duszę, rozpadała się przez lata, i zdawała się czynić to z nienaturalną szybkością. Do roku 178- tylko kamienie i cegły pozostały na swoim miejscu, a do 1800 nawet one się rozpadły. Nikt nie badał skręconych krzaków na brzegu rzeki gdzie kiedyś miały znajdować się wrota we wzgórzu, i nikt również nie próbował określić miejsca, gdzie Joseph Curwen rozstał się z horrorami, które wydobył na ten śWiat.

Tylko stary Kapitan Whipple był słyszany, szepcząc raz na jakiś czas do siebie ``A cholerę z tym ———, ale nie miał interesu w śmianiu się, gdy krzyczał. To tak, jakby ten piekielny ——— miał coś w swoim rękawie. Za pół szylinga spaliłbym cały jego ——— dom''.

\section{Rozdział Trzeci: Poszukiwania i Ewokacja}

Charles Ward, jak widzieliśmy, po raz pierwszy dowiedział się w 1918 r. o swoim pochodzeniu w prostej linii od Josepha Curwena. Nie jest dziwne, że od razu zaczął się interesować tą antyczną tajemnicą, gdyż wszelka dostępna plotka o której słyszał o Curwenie, teraz stała się czymś, co dotyczyło także i jego, w którego żyłach płynęła jego krew. Żaden genealog  nie zrobiłby niczego innego, jak zaczął czynne i systematyczne zbieranie danych o Curwenie. 

W jego pierwszych poszukiwaniach nie było najmniejszego ślady sekretności, tak więc nawet Dr. Lyman nie chce określić daty szaleństwa młodzieńca na wcześniej niż koniec 1919 r. Rozmawiał on wtedy swobodnie ze swoją rodziną - choć jego matka nie była szczególnie szczęśliwa, posiadając takiego przodka jak Curwen - i z pracownikami wielu muzeów i bibliotek, które odwiedzał. W poszukiwaniach wśród obcych rodzin starych zapisków nie ukrywał niczego, i podzielał humorystyczny skeptycyzm, z którym podchodzono do rewelacji starych pamiętników i listów. Często wyrażał szczere zainteresowanie tym, co naprawdę miało miejsce półtora wieku temu na farmie w Pawtuxet, której lokalizację próbował ustalić, i co naprawdę wydarzyło się w życiu Josepha Curwena.  

Kiedy dostał on w swoje ręce pamiętnik Smitha i natknął się na list od Jedediaha Orne'a, zdecydował się on odwiedzić Salem i poszukać śladów wczesnej aktywności Curwena, jak i jego powiązań z innymi ludźmi. Wyprawę tę odbył podczas wiosennych ferii w 1919 r. W Essex Institute, znanym mu z poprzednich wypraw do wspaniałego starego miasta rozsypujących się, purytańskich szczytów i ciasno położonych dachów, został bardzo mile ugoszczony, i uzyskał tutaj dużo danych o Curwenie. Odkrył on, że jego przodek urodził się w wiosce Salem, obecnie znanej jako Danvers, 7 mil od miasta, 18-tego lutego 1662-3, i że uciekł on nad morze w wieku lat 15, nie zwracając na siebie niczyjej uwagi przez następne 9 lat, kiedy powrócił z mową, ubiorem i manierami prawdziwego Anglika i osiadł w mieście Salem. W tamtym czasie nie dbał zbytnio o swoją rodzinę, ale spędzał większość swych godzin z ciekawymi książkami, które przywiózł z Europy, i dziwnymi chemikaliami, które docierały do niego statkami z Anglii, Francji i Holandii. Pewne jego wycieczki w głąb kraju były obiektami wielkiej lokalnej ciekawości, i w szeptach łączono je z niejasnymi plotkami o ogniach nocami na wzgórzach. 

Jedynymi bliskimi przyjaciółmi Curwena byli Edward Hutchinson z wioski Salem i Simon Orne z miasta Salem. Z tymi to ludźmi często widywano go publicznie, ai wizyty między nimi nie należały do rzadkich. Hutchinson miał dom w pobliżu lasu, i nie był on lubiany wśród ludzi ze względu na odgłosy, które dochodziły z niego w środku nocy. Mówiło się, że gościł dziwnych gości, a światła widziane z jego okien nie były zawsze tego samego koloru. Wieza, którą się wykazywał odnośnie dawno martwych ludzi i historycznych wydarzeń, była uważana za niepokojącą. Zniknął on również z miasta mniej-więcej w momencie, gdy zaczęła się panika związana z wiedźmami, i nikt nigdy więcej o nim już nie słyszał. W tamtym czasie Joseph Curwen także zniknął, ale szybko dowiedziano się o jego pobycie w Providence. Simon Orne żył w Salem do roku 1720, gdy jego niezdolność do wyraźnego starzenia się zwróciła powszechną uwagę. Następnie i on zniknął, choć 30 lat później jego syn, wyglądający kropka w kropkę jak jego ojciec, pojawił się, by odzyskać swoje włościa. Było to możliwe na mocy dokumentu, napisanego ręką Simona Orne'a, i w ten oto sposób Jedediah Orne żył w Salem do roku 1771, gdy pewne listy od mieszkańców Providence trawiły do Wielebnego Thomasa Barnarda i innych osobistości sprawiły, że zniknął po cichu, odchodząc w nieznanym kierunku.  

Pewne dokumenty, napisane przez lub o tych dziwnych sprawach, były dostępne w Essex Institute, Courth House i Registrze Czynów, i zawierały zarówno nieszkodliwe, powszechne dokumenty, takie jak akty własności ziemi i rachunki wystawione za sprzedaż, jak i fragmenty znacznie bardziej prowokacyjnej natury. Istniał0 4 lub 5 bezpośrednich aluzji do nich podczas procesu o czary, np: gdy mężczyzna imieniem Hepzibah Lawson przysiągł 10 lipca 1692 przed sądem w Oyer i Terminem przed sędzią Hathorne, że ``cztery Wiedźmy i Czarny Pan mieli się spotykać w Lesie za domem pana Hutchinsona'' a Amity How zadeklarował podczas sesji ósmego sierpnia przed sędzią Gedneyem, że ``Pan C. B. (George Burroughs) podczas tamtej Nocy umieścił Znak Diabła na imionach Bridget S., Jonathan A., Simon O., Deliverance W., Joseph C., Susan P., Mehitable C., i Deborah B''.' Co więcej, istniał katalog dziwacznej biblioteki Hutchinsona, co odkryto po jego zaginięciu, a także niedokończony manuskrypt napisany jego ręką, zapisany w szyfrze, którego nikt nie był w stanie złamać. Ward posiadał fotostatyczną kopię tego manuskryptu i zaczął pracować nad jego zdeszyfrowaniem w wolnych chwilach, gdy tylko został mu on dostarczony. PO przyjściu sierpnia, jego praca nad nim stała się bardziej intensywna i gorączkowa, i istnieją powody, by wierzyć, bazując na jego mowie i sposobie bycia, że natrafił na klucz do deszyfracji przed październikiem lub listopadem. Nigdy jednakże nie powiedział na głos, czy jego próby okazały się sukcesem czy też nie. 

Jednak jego największe zainteresowanie wzbudziły materiały Orne'a. Zajęło Wardowi tylko krótką chwilę udowodnienie odnośnie jego pisma rzeczy, które wiedział, że są prawdziwe z listów Curwena - to znaczy, że Simon Orne i jego podobno syn to jedna i ta sama osoba. Jak Orne powiedział swojemu korespondentowi, było bardzo niebezpiecznie żyć zbyt długo w Salem, tak więc udał się na 30-letnią podroż w odległe strony, a swoje ziemie pragnął odzyskać pod przykrywką bycia przedstawicielem nowej generacji. Orne najwyraźniej był ostrożny i zniszczył większość swojej korespondencji, ale obywatele, którzy podjęli akcję w 1771 r. odnaleźli i przechowali parę z tych listów i papierów, których treść ich zastanowiła. Była tam tajemne formuły i diagramy spisane ręką jego i innych, które Ward ani nie skopiował, ani nie sfotografował, i jeden wyjątkowo tajemniczy list spisany odręcznym pismem, które poszukiwacz rozpoznał, porównując z księgami Registru Czynów, jako z pewnością należącym do Josepha Curwena. 

List od Curwena, choć nie zawierał roku napisania, był ewidentnie nie tym, na który odpisał kiedyś Orne, a który przechwycili spiskowcy z Providence, i z wewnętrznych dowodów Ward umiejscowił go na niewiele później niż 1750. Nie zaszkodzi umieścić ten tekst w pełni, jako przykład stylu jednego z tych, których historia była tak mroczna i okropna. Odbiorca jest określony jako ``Simon'', ale linia (stworzona ręką Curwena lub Orne'a, ciężko było to stwierdzić) biegła przez to słowo.

\begin{displayquote}

\begin{flushright}
Providence, 1 maja
\end{flushright}

Bracie:—

Mój szlachetny Antyczny przyjacielu, niech będzie Błogosławiony Ten, którego wiecznej Mocy Służymy. Przyszło mi na myśl, że powinieneś wiedzieć o pewnych kwestiach Ostatnich Ostatecznych Środków, i tym, co się z Nimi wiąże. Nie mogę podążyć za Tobą i odejść, gdyż w Providence jeszcze nie szuka się Podejrzliwie i nie poluje się na Wiedźmy, jak w innych Miastach. Jestem Zajęty Statkami i Dobrami,  i niem ogę ruszyć Twoim szlakiem. Poza tym, jak dobrze Wiesz, moja farma w Pawtuxet nie czekałaby na mój Powrót.

Ale nie jestem nieprzygotowany na ciężkie czasu, jak już Tobie powiedziałem, i od dawna pracuję nad tym, jak Powrócić, gdybym został Zgubiony. Późno w nocy uderzyły we mnie Słowa, które przywołają YOGGE-SOTHOTHE, i ujrzałem po raz Pierwszy tę twarz, o której wspominał   Ibn Schacabac w swoim ——————. I tam wspomniano, że III Psalm w Liber-Damnatus trzyma w sobie Klucz. Ze Słońcem w V Domu, Saturnem w Trzecim, narysuj Pentagram Ognia i powtórz 9-ty Wers po trzykroć. Powtarzaj to w każde Podwyższenie Krzyża Świętego i Wigilię Wszystkich Świętych, i te rzeczy będą się Pomnażać w Zewnętrznych Sferach.

I Nasienie Starszych się Narodzi, a ten, kto spojrzy Wstecz, nie odnajdzie tego, czego Szuka.

lecz, Nic się nie stanie, jeśli nie będzie Dziedzica, i jeśli Sole, lub Sposób, by Je przyrządzać, nie będą Gotowe w jego Dłoniach. I tutaj muszę przyznać, że nie podjąłem stosownych Kroków lub odkryłem Dużo. Ten Proces jest trudny w zakończeniu, i wykorzystuje tyle Okazów, że ciężko mi sprowadzić odpowiednią ich Liczbę, może z wyjątkiem Marynarzy, których sprowadzam z Indii. Ludzie jednakże będą ciekawi, i nie mogę sobie z tym poradzić. Nobliwi są gorsi od ogólnej Populacji, bardziej Dociekliwi w swoich Czynach, i bardziej religijni, niż chcą pokazać. Ten Parson i Pan Merritt mają nieco posłuchu i są gadatliwi, obawiam się, lecz nie są zbyt Niebezpieczni. Te Chemiczne substancje są łatwe w pozyskaniu od dobrych Chemików z Miasta, jak Dr. Bowen lub Sam Carew. Podażam za tym co Borellus powiedział, i mam Pomoc w Abdulu Al-Hazredzie i jego Książce. Cokolwiek otrzymam, Ty mieć będziesz. A w międzyczasie, nie zapomnij korzystać ze Słów, które Ci dałem. Mam je w porządku, ale jeśli Pragniesz ujrzeć JEGO, skorzystać z pism w  ——————, które zamieszczam w tej Przesyłce. Powtarzaj te wersety w każde Podwyższenie Krzyża Świętego i Wigilię Wszystkich Świętych, a jeśli Twa Linia nie wymrze, przeżyjesz w przyszłe Lata, spojrzysz wtedy wstecz i użyjesz Sól lub ich Substytutów, i powinieneś go zostawić. 

Cieszy mnie, że ponownie jesteś w Salem, i mam nadzieję, że niedługo znów Cię ujrzę. Mam dobrego Rumaka, i myślę o zakupie Dyliżansu, już jedna osoba (Pan Merritt) w Providence ma swój, lecz Drogi są okropne. Jeśli jesteś w stanie podróżować, wstąp do mnie. Z Bostonu pójdź drogą Poste, poprzez Dedham, Wrentham i Attleborough, wszystkie te Miasta posiadają dobre Tawerny. Zatrzymaj się u pana Bolcoma w Wrentham, gdzie Łóżka są lepsze niż u pana Hatcha, lecz posilaj się w drugim z Przybytków, gdyż ich kuchnia jest lepsza. Zwróć się ku Providence przy wodospadach Patucket, i idź drogą, przy której leży Tawerna pana Saylesa. Mój dom znajduje się po przeciwnej stronie Tawerny Olneya przy Towne Street, pierwszy po północnej stornie Olney's Court. Odległosć od bostonu to około XLIV mili. 

Panie, jestem Twoim starym i prawdziwym przyjacielem i sługą Almonsin-Metratona.

\begin{flushright}
Josephus C.
\end{flushright}

Do Pana Simona Orne'a
William's-Lane, Salem

\end{displayquote}

Ten list, dosyć zaskakująco, był tym, co dało Wardowi dokładną lokalizację domu Curwena w Providence; gdyż żaden inny zapisek wcześniej nie był na tyle konkretny. To odkrycie było zaiste szokujące, gdyż wskazało, iż istniał więcej niż nowszy dom Curwena, wybudowany w 1761 r. na miejscu starego, rozpadającego się budynku przy Olney Court i dobrze znany Wardowi z jego antykwariuszowych poszukiwań na Stempers Hill. To miejsce było o ledwie parę kroków od jego własnego domu na wyższym poziomie dużego wzgórza, w którym teraz mieszkała murzyńska rodzina znana z okazjonalnego mycia, odkurzania domów i dbania o kominki. Odkrycie, w dalekim Salem, takich nagłych dowodów na znaczenie tego dobrze znanego domu, i powiązanie go z historią rodzini Warda, była dla niego bardzo ważne. Posranowił on zbadać to miejsce natychmiast po powrocie. Bardziej mistyczne frazy tego listu, które uznał za jakiś ekstrawagancki symbolizm, nie przemówiły do niego, lecz odnotował, z pewnym podnieceniem, fragment z Biblii, do którego się odnosił - Hi 14-14 - był już mu znany. ``Ale czy zmarły ożyje? Czekałbym przez wszystkie dni mojej walki, aż taka chwila nadejdzie.''\footnote{Cytat za Biblią Tysiąclecia.}

\begin{center}
2
\end{center}

Młody Ward wrócił do domu w stanie przyjemnego podekscytowania, i spędził następną sobotę na długim i wyczerpującym badaniu domu przy Olney Court. To miejsce, teraz rozpadające się od upływu czasu, nigdy nie było posiadłością, ale były skromnym, dwu-i-pół poziomowym drewnianym domem w znajomym typie kolonialnym Providence, z prostym, spiczastym dachem, dużym centralnym kominem i artystycznie wyrzeźbionymi wrotami z promienistym świetlikiem, trójkątnym frontem, i ozdobnymi, doryckimi pilastrami. Mało rzeczy w nim zmieniono z zewnątrz, i Ward mógł poczuć, że patrzy na coś bardzo bliskiego złowróżebnym celom jego zadania.

Obecni murzyńscy mieszkańcy byli mu znani, i bardzo serdecznie został oprowadzony po wnętrzu przez starego Asę i jego żonę, grubą Hannę. W środku zaistniało więcej zmian, niż wskazywałby na to wygląd zewnętrzny, i Ward dostrzegł z żalem, że cała połowa pięknych kominków w kształcie znojów i urn oraz wyściółek szafek zdobionych rzeźbieniami w muszlach zniknęła na zawsze, a większość pięknych boazerii i sztukaterii nosiła teraz rysy, były pocięte lub wyżłobione, lub też całkowicie zakryte tanią tapetą. Ogólnie rzecz ujmując, przebadanie domostwa nie powiedziało Wardowi tak dużo, jak spodziewał się zrozumieć, ale i tak było ekscytujące stać wewnątrz ścian swego przodka, które gościły kiedyś takiego mężczyznę horroru jak Joseph Curwen. Ward zauważył z drżeniem, że monogram został ostrożnie zatarty ze starej, mosiężnej kołatki.   

Od tamtego momentu, aż do czasu po zamknięciu szkoły Ward spędzał swój czas na fotostatycznej kopii szyfru Huntchinsona i na gromadzeniu lokalnych danych o Curwenie. To pierwsze zadanie póki co nie wydało owoców, ale w tym drugim uzyskał tak wiele, i wskazówki do innych danych w innych miejscach, że był gotów by odbyć podróż do Nowego Londynu i Nowego Jorku, by skonsultować się ze starymi listami, których obecność w tych miejscach wskazywały jego własne badania. Ta podroż była zaiste wielce owocna, gdyż dała mu ona listy Fennerów z okropnym opisem rajdu na farmę w Pawtuxet i listy Nightningale'a-Talbora, z których dowiedział się o portrecie namalowanym na panelu w bibliotece Curwena. Zwłaszcza ów portret interesował go wielce, gdyż oddałby wiele, za wiedzę o tym, jak Joseph Curwen wyglądał. Wykonał on także powtórne badania domu przy Olney Court, by sprawdzić, czy nie ma tam jakiś śladów antycznych pozostałości pod farbą lub warstwami spleśniałej tapety. 

Te poszukiwania odbyły się we wczesnym sierpniu, i Ward ostrożnie przebadał ściany każdego pomieszczenia na tyle dużego, by być bibliotekę złego budowniczego. Szczególną uwagę poświęcił dużym panelom w miejscach nad kominkiami, które ciągle jeszcze zostały nietknięte, i był bardzo podekscytowany, gdyż po około godzinie, kiedy, na szerokiej przestrzeni nad kominkiem w dużym pokoju na poziomie gruntu, był pewien, iż powierzchnia ukazana poprzez zdejmowanie paru warstw farby była znacznie ciemniejsza niż jakakolwiek farba lub drewno miały prawo być. Parę innych ostrożnych testów przy pomocy cienkiego noża, i wiedział już, że natrafił na portret olejny wielkiego rozmiaru. Z prawdizwie naukowym zacięciem, młodzieniec powstrzymał się przed ryzykowaniem uszkodzenia, który natychmiastowe próba wyrwania ukrytego obrazu przy pomocy noża mogłaby uczynić, lecz zamiast tego udał się w poszukiwaniu eksperckiej pomocy. Po trzech dniach wrócił on, z artystą wielkiego doświadczenia, panem Walterem Dwightem,  którego pracownia znajdowała się niedaleki College Hill, i ten sprawny odnowiciel obrazów ruszył do pracy od razu z odpowiednią metodologią i chemicznymi substancjami. Stary Asa i jego żona byli prawdziwie podekscytowania swoimi dziwnymi gośćmi, i zwrócono im pieniądze za tę inwazję ich domowego ogniska.

W miarę, jak z dnia na dzień prace restauracyjne postępowały, Charles Ward patrzył z rosnącym zainteresowaniem na linie i odcienie powolutku odkrywane po ich długoletnim zapomnieniu. Dwight zaczął od dołu; ponieważ obraz miał proporcje 3/4, twarz nie wynurzyła się przez pewien czas. W międzyczasie można było dostrzec, że sportretowany człowiek był dobrze ukształtowany, w czarno-niebieskim płaszczu, haftowanej kamizelce, chusteczce z czarnej satyny, i skarpetkach z białego jedwabiu, posadzony w rzeźbionym krześle, z oknem jako tłem, za którym widać było statki. Kiedy głowa postaci wyłoniła się spod farby, nosiła ona biała brytyjską perukę i posiadała cienką, spokojną, nijaką twarz, która wydawała się zaskakująco znajoma zarówno dla Warda, jak i artysty, z którym pracował. Dopiero na samym końcu, jednakże, restaurator i jego klient zaczęły krzyczeć z zaskoczenia, gdy detale tej szczupłej, bladej twarzy zostały przez nich rozpoznane jako dramatyczny trik dziedzictwa. Gdyż wraz z ostatnim usunięciem farby i ostatnim zadrapaniem skrobaka wyłoniła się w pełni twarz ukryta przed setkami lat, i wtedy to też Charles Dexter Ward, badacz przeszłości, spojrzał na swe własne lico, oblicze jego okropnego pra-pra-pra-dziadka. 

Ward sprowadził swoich rodziców, by ujrzeli cud, który odkrył, i jego ojciec natychmiast postanowił zakupić obraz, pomimo jego wykonania na nieruchomej ścianie. Podobieństwo do chłopaka, pomimo wyglądu raczej starszego wieku, było niezwykłe - i można było uznać za jakiś trik atawizmu, że fizyczne cechy Josepha Curwena odnalazły swój dokładny duplikat po półtora wieku. Podobieństwo pani Ward do jej przodka nie było wielkie w ogóle, choć mogła ona przywołać w pamięci krewnych, którzy mieli pewne cechy twarzy podobne do zarówno jej syna, jak i starego Curwena. Nie cieszyła się ona na to odkrycie, i powiedziała swemu mężowi, że najlepiej byłoby, gdyby spalił obraz, zamiast snuć plany wzięcia go do domu. Było coś niezwykle niepokojącego odnośnie całego tego obrazu, podsumowała ona, nawet jeśli czuła to tylko podskórnie, w jego podobieństwie do Charlesa. Pan Ward, jednakże, był praktycznym mężczyzną mocy i układów - hodowca bawełniany z obszernymi włościami w Riverpoint w Dolinie Pawtuxet - i nie leżało w jego zwyczaju słuchać kobiecych narzekań. Obraz zaimponował mu bardzo podobieństwem do jego syna, i wierzył, że chłopak zasłużył na niego jako prezent w jego opinii - było zbyteczne, by to mówić, że Charles się zgodził. Parę dni później pan Ward zlokalizował właściciela domu - małą osobę podobną w wyglądzie do szczura, mówiącą gardłowym akcentem - i zakupił zarówno kominek, jak i panele ponad nim, z obrazem, kupując go za ustaloną z góry cenę, co oszczędziło wszystkim męki targowania się. 

Wszystko, co pozostało, to zdjęcie paneli z obrazem ze ściany i przeniesienie ich do domu Warda, gdzie postanowiono o kompletnej restauracji obrazu i umieszczeniu go nad elektrycznym kominkiem w bibliotece Charlesa na trzecim piętrze. Zadanie nadzoru przenoszenie obrazu zlecono Charlesowi i 28 sierpnia towarzyszył on dwóm pracownikom-ekspertom z firmy dekoracyjnej Crookera w wyprawie do domu przy Olney Court, gdzie kominek i obraz nad kominkiem odłączono od ściany z wielką ostrożnością, by przetransportować je firmową ciężarówką. Pod spodem zostały gołe cegły, pokrywające komin, a w nich młody Ward  dostrzegł pustą przestrzeń, mniej więcej wymiarów stopy sześciennej, która musiała się zjandował bezpośrednio za głową portretu. Ciekawy odnośnie tego, co mogło się weń znajdować, młodzieniec zbliżył się i spojrzał wgłąb, znajdując pod przykryciem kurzu i pyłu jakieś zazólcone papiery, gruby notes i pewne stare tekstylia, które ongiś mogły być wstążką, która wszystko to razem wiązała. Zdmuchnąwszy kurz, wziął książkę i spojrzał na grubą inskrypcję na jej okładce. Była ona spisana ręką, którą rozpoznał z Essex Institute, i inskrypcja głosiła, iż był to ``Dziennik i Notatki Jos. Curwena, Zamieszkałego na Plantacjach w Providence, Wcześniej Zaś Salem''.

Podekscytowany poza wszelką miarą swoim odkryciem, Ward pokazał książkę dwóm ciekawskim pracownikom obok siebie. Ich zeznania są absolutne co do natury i szczerości tego odkrycia, i Dr. Wilett polega na nich w swojej teorii, że młodzieniec nie był szalony kiedy zaczęły się jego ekscentryczne zachowania. 

Reszta papierów także została spisana ręką Curwena, a jeden z nich wydawał się szczególnie ważny, uwzględniwszy inskrypcję: ``Do Tego, Który Przyjdzie Potem, I Jak On Może się Dostać Poza Czas i Sfery''. Inny był zapisany szyfrem, tym samym, Ward miał nadzieję, co szyfr Hutchinsona, którego znaczenie dalej mu umykało. Trzeci, i tutaj poszukiwacz krzyknął z radości, zdawał się być kluczem do szyfru. Czwarty i piąty, z kolei, były zaadresowane do ``Edw: Hitchinsona, Szlachcica'' i ``Jedediaha Orne'a, Wielmożnego Pana'' lub ``Do Ich Spadkobierców, Lub Tych, Którzy Ich Reprezentują''. Szósty i ostatni był zatytułowany jako ``Josepp Curwen - jego Życie i Podróże, Lata 1678-1687 lub to, gdzie Podróżował, gdzie Nocował, co Widział i czego się Nauczył''. 

\begin{center}
3
\end{center}

Osiągamy teraz punkt, w którym bardziej akademickie szkoły alienistów datują szaleństwo Charlesa Warda. Po swoim odkryciu, młodzieniec sprawdził natychmiast parę stron w książce i manuskryptach, i najwyraźniej dostrzegł w nich coś, co wywarło na nim wielkie wrażenie. Zaiste, pokazując owe znaleziska robotnikom, zdawał się stróżować tekstu z wielką ostrożnością, czego nie wyjaśniały w pełni nawet jego zainteresowania genealoga i antykwariusza. Po powrocie do domu, podzielił się nowiną, prawie będąc zawstydzonym, jakby próbował przekazać wiadomości o wielkiej ważności, bez pokazywania dowodów nań. Nawet nie pokazał znalezisk swoim rodzicom, lecz tylko powiedział im, że odnalazł parę dokumentów spisanych ręką Curwena, ``głównie zaszyfrowane'', które należałoby przestudiować bardzo ostrożnie, zanim okażą one swe prawdziwe znaczenie. Jest wątpliwe, by pokazał swoje znalezisko pracownikom, z wyłączeniem okoliczności ich szczerego zainteresowania. I nie pozostawiało wątpliwości, że pragnął on uniknąć jakiegokolwiek pokazu rezerwy, który mógłby zapoczątkować dyskusję w tym temacie. 

That night Charles Ward sat up in his room reading the newfound book and papers, and when day came he did not desist. His meals, on his urgent request when his mother called to see what was amiss, were sent up to him; and in the afternoon he appeared only briefly when the men came to install the Curwen picture and mantelpiece in his study. The next night he slept in snatches in his clothes, meanwhile wrestling feverishly with the unravelling of the cipher manuscript. In the morning his mother saw that he was at work on the photostatic copy of the Hutchinson cipher, which he had frequently showed her before; but in response to her query he said that the Curwen key could not be applied to it. That afternoon he abandoned his work and watched the men fascinatedly as they finished their installation of the picture with its woodwork above a cleverly realistic electric log, setting the mock-fireplace and overmantel a little out from the north wall as if a chimney existed, and boxing in its sides with panelling to match the room's. The front panel holding the picture was sawn and hinged to allow cupboard space behind it. After the workmen went he moved his work into the study and sat down before it with his eyes half on the cipher and half on the portrait which stared back at him like a year-adding, century-recalling mirror. His parents, subsequently recalling his conduct at this period, give interesting details anent the policy of concealment which he practised. Before servants he seldom hid any paper which he might be studying, since he rightly assumed that Curwen's intricate and archaic chirography would be too much for them. With his parents, however, he was more circumspect; and unless the manuscript in question were a cipher, or a mere mass of cryptic symbols and unknown ideographs (as that entitled "To Him Who Shal Come After, etc." seemed to be) he would cover it with some convenient paper until his caller had departed. At night he kept the papers under lock and key in an antique cabinet of his, where he also placed them whenever he left the room. He soon resumed fairly regular hours and habits, except that his long walks and other outside interests seemed to cease. The opening of school, where he now began his senior year, seemed a great bore to him; and he frequently asserted his determination never to bother with college. He had, he said, important special investigations to make, which would provide him with more avenues toward knowledge and the humanities than any university which the world could boast.

%     więcej