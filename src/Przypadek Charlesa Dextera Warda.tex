\part{Powieść}

\chapter{Przypadek Charlesa Dextera Warda}\index{Dzieła!Przypadek Charlesa Dextera Warda}\index{Osoby!Charles Dexter Ward}\index{Osoby!Dr. Willett}\index{Osoby!Dr. Waite}

\section{Rozdział Pierwszy: Wynik i prolog}

\begin{center}
1
\end{center}

Z prywatnego szpitala dla chorych psychicznie w pobliżu Providence, Rhode Island, ostatnio zaginęła pewna szczególna osoba. Nosiła ona imię Charlesa Dextera Warda, i została umieszczona w zakładzie z wielkim żalem przez rozpaczającego ojca, któremu przyszło patrzeć, jak aberracja wzrasta ze zwykłego ekscentryzmu w mroczną manię zawierającą w sobie zarówno możliwość morderczych tendencji, jak i specyficznej zmiany w zawartości umysłu Charlesa. Psychiatrzy byli dosyć zaskoczeni tym przypadkiem, gdyż prezentował on dziwactwa zarówno psychologicznej, jak i fizjologicznej natury

Po pierwsze, pacjent wydawał się dziwnie starszy niż swoje 26 lat. Psychiczne rany, jeśli są prawdziwe, mogą postarzyć znacząco, lecz twarz tego młodego człowieka zawierała subtelne cechy, które tylko starzy ludzie normalnie uzyskują. Po drugie, jego organiczne procesy wykazywały pewne dziwne proporcje, bez porównania z doświadczeniem medycznym doktorów. Oddychanie i praca serca posiadały zaskakujący brak symetrii, głos jego został utracony, tak, żem ógł on jedynie szeptać, trawienie trwało zaskakująco długo, w dodatku odbywało się w stopniu minimalnym, a reakcje neuronów na standardowe bodźce były nieporównywalne z wcześniejszą wiedzą medyczną, o organizmach zdrowych czy też chorych. Skóra Charlesa była sucha i o szarawym, nieświeżym odcieniu, a struktura komórkowa jego tkanek wydawała się być bardzo uszkodzona i ledwo się trzymająca razem. Nawet wielkie znamię w kształcie oliwki na jego prawym biodrze zaniknęło. W tym samym czasie, na jego klatce piersiowej uformował się wielki pieprzyk lub czarny punkt, którego tam wcześniej nie było. Ogólnie rzecz ujmując, wszyscy doktorzy zgadzali się ze sobą, że Ward posiadał metabolizm, który został uszkodzony w stopniu wykraczającym poza wcześniejsze precedensy.

Psychologicznie Charles Ward także był wyjątkowy. Jego szaleństwo nie miało sobie równych w zapiskach nawet najnowszych i najdokładniejszych rozpraw naukowych i encyklopedii medycznych, i towarzyszyła mu moc mentalna która mogłaby z niego uczynić geniusza lub przywódcę, gdyby nie wykrzywiono jej w dziwną i groteskową formę. Dr. Willett, który był lekarzem rodzinnym Wardów, potwierdza, że czyste zdolności psychiczne pacjenta, mierzalne poprzez jego reakcje na świat poza sferą jego szaleństwa, w zasadzie wzrosły od momentu pierwszego ataku. Ward, jest to prawdą, był zawsze uczonym i bibliofilem, ale nawet jego najgenialniejsze z wczesnych prac nie pokazywały jego jasności intelektu i przenikliwości, którymi sie wykazywał podczas swoich rozmów z alienistami. Zaiste, było trudną rzeczą pozyskanie prawniczej zgody na przyjęcie go do szpitala, tak potężny i jasny zdawał się umysł tego młodego człowieka - tylko przy pomocy silnych dowodów i wielu anormalnym dziurom w jego zasobie informacji w kontraście do jego inteligencji, został on ostatecznie przyjęty do ośrodka. Aż do momentu jego zniknięcia był omni-czytelnikiem i tak dobrym rozmówcą, jak jego biedny głos tylko zezwolił. Uważni obserwatorzy, nie mogąc przewidzieć jego ucieczki, przewidzieli jednak, że nie minie dużo czasu, nim zostanie zwolniony ze szpitala.

\begin{center}
2
\end{center}

Tylko Dr. Willett, który odebrał poród Charlesa Warda i opiekował się wzrostem jego ciała i umysłu od tamtego czasu, wydawał się być przerażony na myśl o jego potencjalnej wolności. Miał on okropne doświadczenie i dokonał straszliwego odkrycia, którego nie ważył się wyjawić swym kolegom-sceptykom. Willett, zaiste, jest pomniejszą tajemnicą na swoich własnych zasadach w połączeniu z tym przypadkiem. Był ostatnim, który widział swego pacjenta przed jego ucieczką, i wyszedł z owej ostatniej konwersacji w stanie będącym mieszanką przerażenia i ulgi, co niektórzy wspominali, gdy ucieczka Warda stała się znane 3 godziny później. Ta ucieczka jest jedną z nierozwiązanych tajemnic szpitala Dr. Waite'a. Okno otwarte wysoko, na wysokości sześciu stóp raczej nie oferowało wyjaśnienia, lecz dalej, po tamtej rozmowie z Willettem młodzieniec niewątpliwie zniknął. Willett nie miał adnego publicznego wyjaśnienia do zaoferowania, choć wydawał się by ć dziwnie zrelaksowany w stosunku do czasu sprzed ucieczki. Wielu, owszem, czuło, że chciałby powiedzieć coś więcej, lecz milczał z obawy przed niewiarą. Odnalazł on Warda w jego domu, lecz krótko po jego odejściu pielęgniarze pukali na nic. Kiedy otworzyli drzwi, pacjenta już tam nie było, a wszystko, co odkryli, to otwarte okno, przez które wiał chłodny, kwietniowy wietrzyk, rozsiewając wokół chmurę blękitno-szarawego pyłu, który prawie ich zadusił. Owszem, psy wyły jakiś czas wcześniej, ale to było wtedy, gdy Willett dalej rozmawiał z pacjentem, a nikt nie znalazł żadnego człowieka lub innej rzeczy, która wyjaśniałaby zniknięcie. Ojciec Warda został poinformowany natychmiastowo za pomocą telefonu, ale wydawało się, że jest on bardziej zasmucony niż zaskoczony. Do czasu, gdy Dr. Waite zadzwonił osobiście, Dr. WIllet już z nim zdażtył porozmawiać i obydwoje nie wiedzieli nic o ucieczce lub pomocy w niej. Tylko z pewnych bardzo sekretnych przyjaciół Willetta i starszego Warda pozyskano pewne wskazówki, a nawet one były zbyt fantastyczne, by brać je poważnie. Jedyny fakt który nam pozostaje to to, że na czas obecny nie ma żadnego śladu uciekającego szaleńca.

Charles Ward był antykwariuszem od swoich lat dziecięcych