\part{Powieść}

\chapter{Przypadek Charlesa Dextera Warda}
\index{Dzieła!Przypadek Charlesa Dextera Warda}
\index{Osoby!Charles Dexter Ward}
\index{Osoby!Dr. Willett}
\index{Osoby!Dr. Waite}
\index{Osoby! Dr. Lyman}
\index{Osoby!Joseph Curwen}
\index{Osoby!Jabez Bowen}
\index{Osoby!Dr. Checkley}
\index{Osoby!John Merritt}
\index{Osoby!Dutie Tillinghast}
\index{Osoby!Eliza Tillinghast}
\index{Osoby!Ezra Weeden}
\index{Osoby!Samuel Winson}
\index{Osoby!John Graves}
\index{Osoby!Cosmo Alexander}
\index{Osoby!Stephen Hopkins}
\index{Osoby!Joseph Brown}
\index{Osoby!Benjamin West}
\index{Osoby!Daniel Jenckes}
\index{Osoby!Eleazar Smith}


\section{Rozdział Pierwszy: Wynik i prolog}

\begin{center}
1
\end{center}

Z prywatnego szpitala dla chorych psychicznie, położonego w pobliżu Providence, Rhode Island, ostatnio zaginęła pewna szczególna osoba. Nosiła ona imię Charles Dexter Ward, i została umieszczona w zakładzie z wielkim żalem przez rozpaczającego ojca, któremu przyszło patrzeć, jak aberracja wzrasta ze zwykłego ekscentryzmu w mroczną manię zawierającą w sobie zarówno możliwość morderczych tendencji, jak i specyficznej zmiany w zawartości umysłu Charlesa. Psychiatrzy byli dosyć zaskoczeni tym przypadkiem, gdyż prezentował on dziwactwa zarówno psychologicznej, jak i fizjologicznej natury

Po pierwsze, pacjent wydawał się dziwnie starszy niż swoje 26 lat. Psychiczne rany, jeśli są prawdziwe, mogą postarzyć znacząco, lecz twarz tego młodego człowieka zawierała subtelne cechy, które normalnie uzyskują tylko ludzie wskutek starości. Po drugie, jego organiczne procesy wykazywały pewne dziwne właściwości, bez porównania z doświadczeniem medycznym doktorów. Oddychanie i praca serca wykazywały zaskakujący brak symetrii, głos jego został utracony, tak, że mógł on jedynie szeptać, trawienie trwało zaskakująco długo, w dodatku odbywało się w stopniu minimalnym, a reakcje neuronów na standardowe bodźce były nieporównywalne z wcześniejszą wiedzą medyczną, o organizmach zdrowych czy też chorych. Skóra Charlesa była sucha i o szarawym, nieświeżym odcieniu, a struktura komórkowa jego tkanek wydawała się być bardzo uszkodzona i ledwo się trzymająca razem. Nawet wielkie znamię w kształcie oliwki na jego prawym biodrze zaniknęło. W tym samym czasie, na jego klatce piersiowej uformował się wielki pieprzyk lub czarny punkt, którego tam wcześniej nie było. Ogólnie rzecz ujmując, wszyscy doktorzy zgadzali się ze sobą, że Ward posiadał metabolizm, który został uszkodzony w stopniu wykraczającym poza wcześniejsze precedensy.

Psychologicznie Charles Ward także był wyjątkowy. Jego szaleństwo nie miało sobie równych w zapiskach nawet najnowszych i najdokładniejszych rozpraw naukowych i encyklopedii medycznych, i towarzyszyła mu moc mentalna która mogłaby z niego uczynić geniusza lub przywódcę, gdyby nie wykrzywiono jej w dziwną i groteskową formę. Dr. Willett, który był lekarzem rodzinnym Wardów, potwierdza, że czyste zdolności psychiczne pacjenta, mierzalne poprzez jego reakcje na świat poza sferą jego szaleństwa, w zasadzie wzrosły od momentu pierwszego ataku. Ward, jest to prawdą, był zawsze uczonym i bibliofilem, ale nawet jego najgenialniejsze z wczesnych prac nie pokazywały jego jasności intelektu i przenikliwości, którymi się wykazywał podczas swoich rozmów z alienistami. Zaiste, było trudną rzeczą pozyskanie prawniczej zgody na przyjęcie go do szpitala, tak potężny i jasny zdawał się umysł tego młodego człowieka - tylko przy pomocy silnych dowodów i wielu anormalnym dziurom w jego zasobie informacji w kontraście do jego inteligencji, został on ostatecznie przyjęty do ośrodka. Aż do momentu jego zniknięcia był omni-czytelnikiem i tak dobrym rozmówcą, jak jego biedny głos tylko zezwolił. Uważni obserwatorzy, nie mogąc przewidzieć jego ucieczki, przewidzieli jednak, że nie minie dużo czasu, nim zostanie zwolniony ze szpitala.

\begin{center}
2
\end{center}

Tylko Dr. Willett, który odebrał poród Charlesa Warda i opiekował się wzrostem jego ciała i umysłu od tamtego czasu, wydawał się być przerażony na myśl o jego potencjalnej wolności. Miał on okropne doświadczenie i dokonał straszliwego odkrycia, którego nie ważył się wyjawić swym kolegom-sceptykom. Willett, zaiste, jest pomniejszą tajemnicą na swoich własnych zasadach w połączeniu z tym przypadkiem. Był ostatnim, który widział swego pacjenta przed jego ucieczką, i wyszedł z owej ostatniej konwersacji w stanie będącym mieszanką przerażenia i ulgi, co niektórzy wspominali, gdy ucieczka Warda stała się znana 3 godziny później. Ta ucieczka jest jedną z nierozwiązanych tajemnic szpitala Dr. Waite'a. Okno otwarte wysoko, na wysokości sześciu stóp raczej nie oferowało wyjaśnienia, lecz dalej, po tamtej rozmowie z Willettem młodzieniec niewątpliwie zniknął. Willett nie miał żadnego oficjalnego wyjaśnienia do zaoferowania, choć wydawał się być dziwnie zrelaksowany w stosunku do czasu sprzed ucieczki. Wielu, owszem, czuło, że chciałby powiedzieć coś więcej, lecz milczał z obawy przed niewiarą. Odnalazł on Warda w jego pokoju, lecz krótko po jego odejściu pielęgniarze pukali na nic. Kiedy otworzyli drzwi, pacjenta już tam nie było, a wszystko, co odkryli, to otwarte okno, przez które wiał chłodny, kwietniowy wietrzyk, rozsiewając wokół chmurę błękitno-szarawego pyłu, który prawie ich zadusił. Owszem, psy wyły jakiś czas wcześniej, ale to było wtedy, gdy Willett dalej rozmawiał z pacjentem, a nikt nie znalazł żadnego człowieka lub innej rzeczy, która wyjaśniałaby zniknięcie. Ojciec Warda został poinformowany natychmiastowo za pomocą telefonu, ale wydawało się, że jest on bardziej zasmucony niż zaskoczony. Do czasu, gdy Dr. Waite zadzwonił osobiście, Dr. WIllet już z nim zdążył porozmawiać i obydwoje nie wiedzieli nic o ucieczce lub pomocy w niej. Tylko od pewnych bardzo sekretnych przyjaciół Willetta i starszego Warda pozyskano pewne wskazówki, a nawet one były zbyt fantastyczne, by brać je poważnie. Jedyny fakt który nam pozostaje to to, że na czas obecny nie ma żadnego śladu uciekającego szaleńca.

Charles Ward był antykwariuszem od swoich lat dziecięcych, niewątpliwie pozyskując swój gust ze starego miasta wokół niego i z reliktów przeszłości, które wypełniały każdy kąt starej willi jego rodziców przy Prospect Street na wzgórzu. Wraz z latami, jego oddanie antycznym przedmiotom tylko się zwiększyło, tak, że historia, genealogia i studiowanie kolonialnej architektury, mebli i rzemiosła wyrzuciły z jego umysłu wszelkie inne zajęcia. Te gusta są ważne do zapamiętania, rozważając jego szaleństwo, gdyż choć nie stanowią jego całkowitego centrum, odgrywają w nim znaczącą rolę. Luki w wiedzy, które alieniści dostrzegli były wszystkie powiązane z współczesnymi sprawami i były zakryte przez znaczącą wiedzę o sprawach dawno minionych, co wykazały przesłuchania: ktoś mógłby pomyśleć, że pacjent przeniósł się tutaj dosłownie z jednej ze starych epok poprzez jakaś osobliwą formę auto-hipnozy. Dziwną rzeczą było to, że Wad nie wydawał się już dłużej zainteresowany antykami, które znał tak dobrze. Miał, wynika, stracić wszelkie uznanie dla nich ze względu na dobrą znajomość, i jego ostatnie wysiłki było wszystkie poświęcone zdobyciu mistrzostwa w powszechnych faktach dnia codziennego, które kompletnie wyparowały z jego umysłu. To, że to kompletnie zaćmienie zdolności mentalnych nastąpiło, było czymś, co starał się on ukryć. Było to jednak oczywiste dla każdego, kto go obserwował, że jego cały program czytania i konwersacji był zdeterminowany przez życzenie pozyskania wiedzy o swoim własnym życiu i zwykłych praktycznych i kulturowych sprawach XX wieku, zgodnie z jego narodzinami w 1902 i edukacją w szkołach naszych czasów. Alieniści teraz się zastanawiają, jak z punktu widzenia jego uszkodzonej wiedzy, pacjent-uciekinier radzi sobie we współczesnym świecie; dominuje opinia, że ``zaszył się gdzieś'' do czasu, aż jego wiedza o świecie współczesnym powróci.

Początki szaleństwa Warda są kwestią dyskusyjną pośród alienistów. Dr. Lyman, eminentny autorytet z Bostonu, umieścił ją gdzieś w 1919 lub 1920, podczas ostatniego roku chłopaka w Moses Brown School, gdy nagle zwrócił się od studiów nad historią w stronę studiów okultystycznych, i odmówił wysłaniu dokumentów na uniwersytet na bazie indywidualnych badań o znacznie większym znaczeniu dla niego. To z pewnością jest powiązane z jego zmienionymi nawykami w tamtym czasie, zwłaszcza jego ciągłemu przeczesywaniu zapisków historii miejskiej i starych miejsc pochówku szukając pewnego konkretnego grobu z 1771 - grobu jego przodka imieniem Joseph Curwen, którego niektóre dokumenty znalazł ukryte w starym domu na Olnej Court, na wzgórzu Stempers, o którym jest wiadome, że Curwen go zamieszkiwał. 

Jest niewątpliwie oczywiste, że zima 1919-1920 przyniosła wielką zmianę w Wardzie, gdyż zaprzestał swoich badań antykwariusza i skupił się wyłącznie na desperackich badaniach nauk okultystycznych zarówno w domu jak i za granicą, przetykane tylko jego dziwnie stałym poszukiwaniem grobu swego przodka.

Jednakże, w tym momencie pojawia się sprzeciw Dr. Willetta, bazując swoją opinię na bliskim i ciągłym kontakcie z pacjentem, i na pewnych przerażających badaniach i odkryciach, których sam dokonał. Te badania i odkrycia odcisnęły na nim swoje piętno - jego głos się załamuje, gdy o nich opowiada, takoż jego dłoń drży, gdy próbuje o nich pisać. Willett przyznaje, że zmiana z 1919-1920 normalnie oznaczałaby początek postępującej dekadencji, która zakończyła się okropnie smutną i niespotykaną alienację z 1928, ale wierzy osobiście, że bardziej jasny rozdział powinien zostać poczyniony. Zgadza się on, że temperament młodzieńca zawsze był podatny na choroby psychiczne, i że był nadmiernie chętny w swoich odpowiedziach na zjawiska wokół niego, odmawia on przyznania, że te wczesna zmiana oznaczała przejście od poczytalności do szaleństwa, powołując się na słowa samego Warda, że odkrył coś, co najpewniej było wspaniałe i kluczowe dla historii myśli ludzkiej.

Prawdziwe szaleństwo, doktor jest pewien, nastąpiło wraz z późniejszą zmianą - po tym jak portret Curwena i jego antyczne dokumenty ujrzały światło dzienne, po tym, jak wykonano podróże do dziwnych, zagranicznych lokacji, i pewne straszliwe inwokacje zostały wyrecytowane w dziwnych i sekretnych okolicznościach, po tym, jak pewne \textit{odpowiedzi} na te inwokacje zostały jawnie wskazane, a pośpieszny list napisany w kondycji niewyjaśnionej agonii został napisany, po fali wampiryzmu i tajemniczej plotce o Pawtuxet, i po ty  ,jak z pamięci pacjenta usunięte zostały wszelkie wspomnienia współczesnych czasów  przy jednoczesnym osłabieniu jego głosu i jego fizyczności subtelnie zmienionej, co już wykazano wyżej.

Dopiero w tamtym czasie, Willett wskazuje z dokładnością, koszmarne cechy stały się stalą częścią Warda, i doktor jest przerażająco pewny, że istnieje dostatecznie dużo twardych dowodów, by potwierdzić twierdzenie młodzieńca o ważnym odkryciu. Po pierwsze, dwóch pracowników wysokiej inteligencji widziało antyczne dokumenty Josepha Curwena. Po drugie, sam chłopak kiedyś pokazał mu owe dokumenty i stronę z pamiętnika Curwena, i każdy z tych dokumentów posiadał autentyczny wygląd. Znana jest lokacja dziury, w której Ward znalazł owe zapiski, a sam Willett miał bardzo przekonujący ostatni rzut oka na nie w okolicznościach, którym ciężko jest dać wiarę i których być może nigdy nie udowodnimy. Dalej, są tajemnice i zbiegi okoliczności listów Orne'a i Hutchinsona, i problem pisma odręcznego Curwena i to, co detektywi odkryli o Dr. Allenie - Tt rzeczy i okropna wiadomość w średniowiecznych zapiskach odnalezionych w kieszeni Willetta po tym, jak odzyskał świadomość po swoim szokującym doświadczeniu.

Najważniejsze jednak są dwa okropne \textit{wyniki}, które doktor pozyskał z pewnej pary mistycznych formuł podczas swojego ostatecznego śledztwa - wyniki które udowodniły autentyczność dokumentów i ich potworne implikacje w tym samym czasie, jak i to, że te dokumenty zrodzone były z ludzkiej wiedzy.

\begin{center}
3
\end{center}

Należy spojrzeć na wcześniejsze życie Charlesa Warda jak na coś, co należy do przeszłości tak mocno, jak antyki, który sobie tam mocno ukochał. W jesieni 1918, i z pokazaniem wyraźnego zapału względem treningu wojskowego tamtego okresu, zaczął on swój pierwszy rok w Moses Brown School, która znajduje się nieopodal jego domu. Stary budynek główny, wzniesiony w 1819, zawsze był urokliwy dla oczu młodego antykwariusza - a rozległy park w którym mieści się Akademia oferował piękne widoki. Jego aktywności społeczne były mocno ograniczone, a swe godziny spędzał głównie w domu, na spacerach, w swoich klasach i w poszukiwaniu danych o antykach i genealogii w Urzędzie Miejskim, Domie Państwowym, bibliotece publicznej, Ateneum, Towarzystwie Historycznym, bibliotekach Uniwersytetu Brown, i nowo otwartej Bibliotece Shepley'a na Benefit Street. Można go sobie łatwo wyobrazić w tych dniach - wysokiego, chudego i o blond włosach, z uczonym okiem i lekko pochylonego, ubranego nieco niezgrabnie, i dającego wrażeniem raczej niegroźnej niezdarności niż atrakcyjności. 

Jego spacery były zawsze przygodami powiązanymi z historią, podczas których był w stanie wyobrazić sobie miliony reliktów chwalebnej historii starego miasta i połączyć je z minionymi wiekami. Jego domem była okazała posiadłość w stylu georgiańskim, na szczycie stromego wzgórza, które wznosiło się na wschód przy rzece, a z jego tylnych okien mógł spojrzeć na ciasno upakowane wieżyce, stropy, dachy i drapacze chmur niższego miasta aż do purpurowych wzgórz poza miastem. To tutaj się urodził i z kochanego, klasycznego ganku fasady z podwójnie wypiekanych cegieł jego opiekunka prowadziła go w wózku, obok małej, białej farmy założonej 200 lat wcześniej i w kierunku budynków akademickich przy wystawnej ulicy, której stare, kwadratowe posiadłości z cegieł i mniejsze drewniane domy z wąskimi gankami opierającymi się na ciężkich kolumnach doryckich, były ustawione pośród przestronnych podwórek i ogrodów.

Był także prowadzony na wózku wzdłuż sennej uliczki Congdon Street, jeden poziom niżej na stromym wzgórzu, wraz z jej wschodnimi domami z wysokimi tarasami. Małe drewniane domki były najstarsze w tym miejscu, gdyż z tego właśnie wzgórza wzrastało rosnące miasteczko. To w tych podróżach przesiąkał on czymś w rodzaju koloru starej, kolonialnej wioski. Opiekunka miała w zwyczaju zatrzymać się i usiąść na ławkach Tarasu Prospekt, by porozmawiać z policjantami; jednym z pierwszym wspomnień tego dziecka był wielki, zachodni ocean dachów, kopuł i wież oraz odległe wzgórza które dostrzegł pewnego zimowego popołudnia z wielkiego wału z szynami, wszystkie fioletowe i mistyczne naprzeciwko rozgrzanego, apokaliptycznego zachodu słońca, malującego niebo czerwienią, złotem, purpurą i różnymi odcieniami zieleni. Szeroka, marmurowa kopuła Domu Stanowego wyróżniała się swoją masywną sylwetką, posąg będący jego ukoronowaniem otoczony fantastyczną aureolą poprzez szparę w jednej z stratusowych chmur, które lśniły na ognistych niebiosach. 

Gdy był starszy, rozpoczęły się jego słynne spacery; najpierw z jego niecierpliwie zaciągniętą opiekunką, a potem samotnie, w marzycielskiej medytacji. Głębiej i głębiej w ulice tego prawie pionowego wzgórza miał się zapuszczać, za każdym razem sięgając starszych poziomów tego antycznego miasta. Zatrzymywałby się niepewnie przy wejściu do Janckes Street z jej murami z tyłu i kolonialnymi szczytami sięgającymi aż do ciemnej Benefit Street, gdzie ukazywał mu się drewniany antyczny dom z parą drzwi w stylu jońskim, a obok niego był prehistoryczny dom z dachem mansardowym, gdzie zachowała się jeszcze resztka farmy, a także dom wielkiego sędziego Durfee z jego upadłymi pozostałościami stylu Georgiańskiego. To tutaj miały powstać slumsy, ale wielkie drzewa wiązów rzucały odżywczy cień na to miejsce, a chłopiec zwykł udawać się stąd na południe, wzdłuż długiej linii domów sprzed czasów Rewolucji Amerykańskiej, z wielkimi strychami i klasycznymi portalami. Po wschodniej stronie budynki te były osadzone wysoko ponad piwnicami, z podwójnymi schodami z kamiennymi stopniami i młody Charles mógł z wielką łatwością wyobrazić je sobie takie, jakie musiały być, gdy ulica ta była jeszcze młoda, gdy frontony były świeżo pomalowane, a nie widocznie zużyte, jak w dniu obecnym. 

Od strony zachodniej, wzgórze schodziło w dół prawie tak stromo jak wyżej, aż do starej `Town Street'', którą założyciele miasta umieścili przy krawędzi rzeki w 1636 r. To tutaj mieściły się niezliczone małe pasy z pochylonymi małymi domkami wielkiej antyczności; i, choć był wielce zafascynowany, minęło wiele czasu, nim odważył się spenetrować ich archaiczny wzrost z lęku przed wejście w sen lub wrota do nieznanych horrorów. Odkrył on, iż jest znacznie mniej straszliwe spacerowanie dalej wzdłuż Benefit Street aż do żelaznych wrót ukrytego kościółka Św. Johna i tyłu Domu Kolonialnego z 1761 r. i do zbutwiałego cielska karczmy Golden Ball, gdzie niegdyś zatrzymał się Waszyngton. Na Meeting Street  - zwanej Gaol Lane i King Street w innych czasach - szukałby na wschodzie i dostrzegłby łuk schodów, wspinających się na wzgórze, a w dół, w kierunku zachodnim, dostrzegłby Szkołę Kolonialną ze starych cegieł,  śmiejącą się z antycznego znaku z głową Szekspira, gdzie \textit{Providence Gazette} i \textit{Country-Journal} były drukowane przed Rewolucją. Następnie pojawiał się przecudowny Kościół Pierwszych Baptystów z 1775 r., luksusowy widok z jego niepowtarzalną wierzą Gibbsa i Georgianskimi dachami i kopułami, unoszącymi się wysoko. Tutaj i w kierunku południowym sąsiedztwo stało się lepsze, w ostateczności wykwiłwszy w cudowną grupę wczesnych posiadłości; lecz dalej małe antyczne paski prowadziły w kierunku zachodnim, wydarłwszy się duchami w ich wielo-spiczastymi archaizmami, i ociekającymi od błyszczącego rozkładu, gdzie stare nadbrzeża wspominają dumnie czasy Indii Wschodnich wśród poliglotów, rozkładających się sterów, sklepów żeglarskich z zamglonymi witrynami i nazwami ulic z dawna, które przetrwały aż po dziś dzień, takimi jak Packet, Bullion, Gold, Silver, Coin, Doubloon, Sovereign, Guilder, Dollar, Dime, i Cent.

Czasami, po tym, jak urósł wyższy i bardziej żądny przygód, młody Ward ruszał głębiej w dół, ku burzy chwiejnych domów, złamanych poprzecznic, schodów, wygiętych balustrad, ciemnych twarzy i nienazwanych zapachów, ciągnących się od South Main do South Water, szukając doków, gdzie zatoka i statki parowe się spotykały, i wracał w stronę północną niższymi poziomami magazynów z 1816 o stromych dachach i szeroką drogą Wielkiego Mostu. To tutaj rynek z 1773 dalej stoi na jego antycznych łukach. Na tej szerokiej drodze zwykł się zatrzymywać, by wypić z czary piękna starego miasta gdy to wznosiło się w stronę wschodnią, pełne Georgiańskich szczytów i ukoronowane nową kopułą Christian Science tak, jak Londyn jest koronowany kopułą katedry św. Paula. Najbardziej lubił docierać do tego punktu późnym popołudniem, gdy światło słoneczne dotykało rynku i antycznych dachów domów na wzgórzu i ich dzwonnic, malując je złotem, i rzuca swą magię pośród wyśnionych nadbrzeży, gdzie niegdyś indianie z Providence zwykli zarzucać kotwice. Po dłuższym przyjrzeniu się temu widokowi, zakręciłoby mu się w głowie z poetycką miłością, i wtedy zacząłby drogę powrotną do domu, mijając po drodze stary, biały kościół i wąskie, strome uliczki, gdzie złote błyski odbijałyby się w oknach i naświetlach umieszczonych wysoko ponad podwójnymi schodami z balustradami z kutego żelaza. 

Innymi razy, i w późniejszych latach, szukał on wyrazistych kontrastów: spędzając połowę swojego spaceru w rozpadających się kolonialnych regionach na północny wschód od swego domu, gdzie wzgórze opada na najniższy poziom do Stempers Hill z jego gettem murzyńskim\footnote{Ogólnie rzecz ujmując nie lubię ``słowa na M'' i wierzę, że nie powinno się z niego korzystać w języku polskim. Tutaj użyłem go, niejako wbrew sobie, gdyż (A) Lovecraft użył w oryginale słowa ``Negro'' i (B) wiem doskonale, że za życia posiadał on uprzedzenia rasowe. Pragnę jednak zaznaczyć, że właściwy język we współczesnej polszczyźnie to ``osoba czarnoskóra'' a nie słowo na M. (przyp. tłum.)}, gromadzącym się wokół miejsca, gdzie dyliżansy do Bostonu zwykły rozpoczynać swą podróż przed czasami Rewolucji. Drugą połowę swoich spacerów spędzał on w na południowych ziemiach wokół ulic George'a, Benevolent, Power i Williams'a, gdzie stare wzgórze jest gruntem dla pięknych włości i odgrodzonych murami ogrodów oraz zielonych ścieżek, z którymi wiąże się tyle drogich wspomnień. Te podróże, razem z pilnymi studiami, które im towarzyszyły, z pewnością przyczyniły się do wielkiej wiedzy historycznej zamieszkującej umysł Charlesa Warda. Ilustruje to także mentalną podstawę, niejaki grunt, na który padłu nasiona tej pamiętnej zimy 1919-1920, która to poruszyła wydarzenia, które miały tak dziwny i straszliwy finał.

Dr. Willett jest pewien, że do tej okropnej zimy, zainteresowania historyczne Charlesa Warda były wolne od zaburzonej psychiki. Cmentarze były wtedy dla niego bez szczególnego znaczenia, poza ich wartością historyczną, a cokolwiek w stylu przemocy lub dzikiego instynktu było poza jego psyche. Wtedy, sunąc powoli, acz nieubłaganie, pojawił się ostateczny wynik jednego z jego badań genealogicznych z poprzedniego roku; wtedy, gdy odkrył wśród swoich przodków po stronie matki pewnego długowiecznego człowieka imieniem Joseph Curwen, który przybył z Salem w marcu 1692, i o którym krążyły plotki i historie, które trudno powtarzać w dobrym towarzystwie.

Pra-pra-pradziadek Warda, Welcome Potter, w 1785 wziął ślub z pewną ``Ann Tillinghast, córką Pani Elizy oraz Kapitana Jamesa Tallinghasta'', o której rodowodzie rodzina nie posiadała żadnych szczegółowych informacji. Pod koniec 1918, podczas szukania woluminu o pierwotnych danych historycznych, młody genealog natrafił na wpis opisujący legalną zmianę nazwiska, które w 1772 pani Eliza Curwen, wdowa po Josephcie Curwenie, przyjęła razem ze swoją siedmioletnią córką Ann, której nazwiskiem panieńskim było Tillinghast, uzasadniając ową zmianę ``iż imię jej Męża stało się publiczną Obrazą dla Rozumu, bazując na wiedzy o jego Chorobie, która potwierdza pewne powszechne, antyczne Plotki, tak więc nie pragnę być znana jako jego lojalna Żona, dopóki owe plotki nie okażą się Bzdurne ponad wszelką wątpliwość''. Ten wpis ujrzał światło dzienne po przypadkowej separacji dwóch kartek, które ostrożnie sklejono razem i które były wyłączone z skądinąd poprawnej numeracji stron.

Było od razu oczywiste dla Charlesa Warda, że odkrył nieznany wcześniej sekret swojego pra-pra-pra-pradziadka. Odkrycie podnieciło go szczególnie, gdyż już słyszał ogólne raporty i widział rozsiane pogłoski o tej osobie, odnośnie której pozostało tak mało weryfikowalnych danych, pomijając te nieliczne, które ujrzały światło dzienne dopiero we współczesnych czasach, tak, że prawie zdawało się to być spiskiem, mającym na celu usunięcie go z ludzkiej pamięci. To, co jednak wydawało się płynąć z tych danych, miało tak prowokacyjną naturę, że nie dało się wyobrazić sobie, co powodowało, że owi kolonialni kronikarze byli tak pełni lęku i chętni, by ukryć i zapomnieć, lub by podejrzewać, że powody owego wykasowania były aż zanadto właściwe. 

Przed tym, Ward był wielce kontent, mogąc wyobrażać sobie starego Josepha Curwena jako coś zawieszonego w powietrzu - ale odkrywszy swoją własną relację z ową ``wymazaną'' figurą, zaczął szukać o nim danych tak systematycznie, jak to tylko możliwe. W tym ekscytującym zadaniu, ostatecznie odniósł on sukces poza swoimi największymi spekulacjami, gdyż stare listy, pamiętniki i nieopublikowane wspomnienia znalezione w zakurzonych strychach Providence i innych miast i miasteczek były obfite w wiele oświecających zapisków, odnośnie których ich autorzy nie uznali za stosowne, by je zniszczyć. Jedna ważna informacja pochodziła ze źródła tak odległego jak Nowy York, gdzie pewne zapiski kolonizatora z Rhode Island przechowywane były w Muzeum Frances' Tavern. Najważniejszą rzeczą jednak, i było to coś, co zdaniem Doktora WIlleta stanowiło ostateczne źródło klęski Warda, było coś, co znalazł w sierpniu 1919 za panelami starego domu w Olney Court. To było to, ponad wszelką wątpliwość, co otworzyło przed nim owe ciemne wizje, które kończyły się daleko poza dnem piekielnym. 

% TUTAJ SKOŃCZONE

\newpage

\section{Rozdział Drugi: Przodek i Horror}

\begin{center}
1
\end{center}

Joseph Curwen, co wyjawiły szeptane legendy odkryte przez Warda, był niezwykłym, enigmatycznym, w tajemniczy sposób okrutnym indywiduum. Uciekł z Salem do Providence - ostatecznego schronienia dla dziwnych, wolnych i niezgadzających się z powszechnie przyjętym konsensusem - na początku wielkiej paniki wiedźm. Bał się on prześladowań ze względu na samotniczy tryb życia i dziwne chemiczne czy też alchemiczne eksperymenty. Był on osobą o szarej skórze około trzydziestki, i szybko został obywatelem Providence, zakupiwszy później dom na północ od włości Gregory'ego Dextera na wysokości Olnejh Street. Jego dom był wybudowany na Stempers Hill na zachód od Town Street, w miejscu, które później zostało Olney Court. W 1761 zamienił ten dom na większy, na tej samej ulicy, który dalej stoi w tamtym miejscu.

Warto zaznaczyć, że pierwszą dziwną rzeczą odnośnie Josepha Curwena było to, że zdawał się nie starzeć. Był zaangażowany w handel, zakupił miejsce na łódkę przy Mile-End Cove, pomógł odbudować Wielki Most w 1713 i Kościół Kongregacji na wzgórzu, ale zawsze posiadał wygląd mężczyzny, który nie miał więcej niż trzydzieści, może trzydzieści pięć lat. W miarę, jak dekady upływały jedna za drugą, ta cecha zaczęła przyciągać uwagę ludu. Curwen zawsze wyjaśniał, że pochodzi z długoletniego rodu, a za pośrednictwem prostego żywota uzyskał świetne zdrowie. Jak ową prostotę można pogodzić z niewyjaśnionymi zakupami sekretnego kupca i z dziwnymi światłami w oknach jego domu przez całą noc, nie było zbyt oczywiste dla mieszkańców miasta. Zamiast tego, postulowali one inne wyjaśnienie jego ciągłej młodości i długoletności. Istniał powszechny konsensus, że mieszanie przez Curwena tajemniczych chemikaliów było odpowiedzialne za jego stan. Plotki głosiły o dziwnych substancjach, które kupował z Londynu i Indii, transportowanych na jego statkach lub zakupionych w Newport, Bostonie lub Nowym Yorku, a gdy stary doktor Jabez Bowen przybył z Rehoboth i otworzył swoją aptekę po drugiej stronie Wielkiego Mostu, zwaną Pod Jednorożcem i Moździerzem, gorące szepty nie mogły zamilknąć o lekach, kwasach i metalach, które długowieczny odludek zakupił lub zamówił u niego. Działając w przekonaniu, iż Curwen posiadał cudowne i tajemnicze medyczne zdolności, wielu chorych ciągnęło do niego z prośbami o pomoc, lecz choć zdawał się on zachęcać ich wierzenia odnośnie samego siebie w luźny sposób, i zawwsze dawał im mikstury w dziwnych kolorach w odpowiedzi na ich prośby, zauważono, że jego dary dla innych rzadko kiedy przynosiły im korzyści. Po upływie lat pięćdziesięciu, i bez większej zmiany niż 5 lat na jego licu, szepty ludzi stały się znacznie mroczniejsze. W efekcie, stał się jeszcze większym odludkiem.

Prywatne listy i pamiętnik z tego okresu ujawniają znacznie więcej powodów dla których Joseph Curwen był obiektem zdumienia, strachu, a w końcu potępiany niczym jakaś plaga. Posiadał pasję do cmentarzy, na których można go było dostrzec o każdej porze dnia i w każdych warunkach. Był z niej znany, choć nikt nie dostrzegł go nigdy czyniącego cokolwiek, co mogłoby być uznane za upiorne. Posiadał on farmę na Pawtuxet Road, na której spędzał czas w lecie, i na której często można było go dojrzeć, jeżdżącego konno o dziwnych godzinach dnia i nocy. Jego jedynymi znanymi sługami, farmerami i dozorcami była smutna para Indian z plemiona Narragansett - mąż głupi i pokryty bliznami, a żona z bardzo obrzydliwą twarzą, najpewniej ze względu na to, iż posiadała w sobie domieszkę czarnej krwi. W dobudówce do tego domu znajdowało się laboratorium, gdzie dokonywano większości chemicznych eksperymentów. Ciekawscy kurierzy, którzy dostarczali butelki, torby lub pudełka poprzez małe drzwi z tyłu rozsiewali plotki o fantastycznych flaszkach, tyglach, alembikach i piecach, które widzieli w niskim pomieszczeniu pełnym półek. Przepowiadali oni szeptem, że cichy ``chymik'' - przez co mieli na myśli \textit{alchemika} - wkrótce odkryje Kamień Filozoficzny. Najbliższe sąsiedzi tej farmy - Fennersowie, którzy znajdowali się o 1/4 mili dalej - mieli jeszcze dziwniejsze opowieści o pewnych dźwiękach, które, byli pewni, dochodziły z domu Curwena w środku nocy. Były to krzyki, mówili oni, i przeciągłe ryki. Nie lubili oni także wielkiej ilości zwierząt, które tłoczyły się na pastwiskach, gdyż tak wielka ich liczba nie była potrzebna by zapewnić staremu człowiekowi i paru sługom mięso, mleko i wełnę. Poszczególne zwierzęta w stadzie zdawały się zmieniać z tygodnia na tydzień, gdy nowe zwierzęta były zakupywane od rolników z Kingstown. Coś dziwnego było także odnośnie pewnego wielkiego, kamiennego budynku na uboczu, którego wysokie, wąskie szczeliny robiły za okna. 

Ludzie spotykani w pobliżu Wielkiego Mostu mieli dużo do powiedzenia o domu Curwena na Olney Court - nie o tym nowym wybudowanym w 1761, kiedy ten mężczyzna liczył sobie prawie wiek, lecz pierwszym - o niskim strychu pozbawionym okien i ścianami krytymi gontem, które spalił do cna po jego zdemolowaniu. Tutaj było mniej tajemnic, to prawda, ale godziny, o których widziano światła, sekretność dwóch obcokrajowców, którzy byli jedynymi sługami, ohydne szepty i dźwięki wydawane przez niesamowicie starego Francuza na usługach Curwena, wielkie ilości posiłków wchodzące przez drzwi, za którymi żyło tylko 4 ludzi, i ogólny \textit{wydźwięk} pewnych głosów często słyszanych w szeptanych rozmowach w dziwnych czasach - wszystko to razem połączone z tym, co było wiadome o farmie w Pawtuxet dało jej złowróżebną sławę.
 
W lepszych kręgach dom Curwena również był tematem rozmów. Nowoprzybyły,  naturalnie czynił znajomości w kościele i życiu społecznym miasta, znajomości lepszego sortu, których towarzystwo i konwersacje w oczywisty sposób sprawiały mu przyjemność. Jego urodzenie było wiadomie dobre, gdyż Curwenowie lub Carwenowie z Salem nie wymagali przedstawienia nikomu w Nowej Anglii. Było ewidentne, że Joseph Curwen podróżował dużo za młodu, żyjąc przez pewien czas w Anglii i pokonując co najmniej dwie podróże do Orientu. Jego mowa, gdy tylko miał taką potrzebę, była językiem wykształconego, kulturalnego Anglika. Ale z pewnego powodu Curwen nie dbał o swoją pozycję społeczną. Choć nigdy nie odmawiał otwarcie gościom, zawsze posiadał wokół siebie ścianę rezerwy, że mało kto mógł pomyśleć o czymkolwiek do powiedzenia do niego, by nie zabrzmieć niczym szaleniec. 

Wydawało się, że sposób, w jaki Curwen się nosi, zawiera w sobie jakąś sekretną, sardoniczną arogancję, jak gdyby odkrył, że wszystkie istoty ludzkie są nużące po tym, jak bywał wśród dziwniejszych i bardziej potężnych istot. Kiedy Dr. Checkley, słynny w swym fachu, przybył do Bostonu w 1738, by zostać proboszczem w King's Church, nie omieszkał on wezwać tego, o którym słyszał tak wiele, ale odszedł po krótkiej chwili, której potrzebował, by wyczuć nutę zła w słowach jego gospodarza. Charles Ward powiedział swemu ojcu, że kiedy dyskutowali o Curwenie pewnego zimowego wieczora, że dałby wiele ,by dowiedzieć się, co tajemniczy starszy mężczyzna powiedział młodemu kapłanowi, ale wszelcy autorzy pamiętników byli zgodni, że Dr. Checkley nie powtórzył niczego, co usłyszał. Ten dobry człowiek był naprawdę zaszokowany, i nigdy nie wspominał Josepha Curwena bez pewnej widocznej straty radości, z której był znany.

Istnieje jednak bardziej określony powód, dla którego inny człowiek wielkiego smaku i urodzenia unikał uczonego odludka. W 1746 pan John Merritt, starszy angielski dżentelmen o wykształceniu naukowym i literackim, przybył z Newport do miasta, które tak szybko się rozwijało i zaczął się budować w miejscu w Neck, które obecnie jest sercem najlepszej dzielnicy rezydenckiej. Żył on w wielkim stylu i komforcie, posiadając pierwszy dyliżans w mieście i zatrudniając dużo sług, i będać bardzo dumnym ze swojego teleskopu, mikroskopu i bardzo selektywnej bibliotece tekstów angielskich i łacińskich książek. Usłyszawszy, że Curwen jest posiadaczem najlepszej biblioteki w Providence, pan Merritt chętnie złożył mu wizytę, i został przyjęty grzeczniej niż większość innych gości w jego domu. Jego zachwyt półkami gospodarza wypełnionymi klasyką w angielskim, łacinie i grece z dodatkiem tekstów filozoficznych, matematycznych i naukowych tekstów, w których wliczał się Paracelsus, Agricola, Van Helmont, Salvius, Glauber, Boyle, Boerhaave, Becher i Stahl, sprawił, że Curwen zaproponował wizytę do swojej farmy i laboratorium,do których nigdy wcześniej nie zaprosił nikogo. Obydwoje ruszyli natychmiast w dyliżansie pana Merritta.

Pan Merritt zawsze twierdził, że nie dostrzegł niczego w oczywisty sposób bluźnierczego na farmie, ale utrzymywał, że tytuły książek w specjalnej bibliotece taumaturgicznej, alchemicznej i teologicznej, którą Curwen trzymał w przednim pokoju, były same w sobie dostateczne, by wywołać w nim uczucie głębokiego dyskomfortu. BYć może jednak, to wyraz twarzy właściciela księgozbioru przyczynił się bardzo mocno do tego uprzedzenia. Niecna kolekcja, poza zbiorem standardowych dzieł, które nie zaalarmowały pana Merritta, posiadała w sobie niemal wszystkie dzieła kabalistyczne, demonologiczne i magiczne znane człowiekowi, i była prawdziwa skarbnicą wiedzy odnośnie wiedzy w wątpliwej domenie alchemii i astrologii. Hermes Trismogistus w edycji Mesnarda, \textit{Turba Philosopharum}, \textit{Liber Investigationis} Gabera i \textit{Klucz do Mądrości} Artephousa - wszystkie tutaj były, razem z kabalistycznym \textit{Zoharem}, zestawem \textit{ALbertus Magnus} Petera Jamma, \textit{Ars Magna et Ultima} Raymonda Lully'ego w edycji Zetznera, \textit{Thesaurus Chemicus} Rogera Bacona, \textit{Clavis Alchimiae} Fludda, \textit{De Lapide Philosophico Trithemusa}. Średniowiezni Żydzi i Arabowie byli obecni w księgozbiorze aż zanadto, i pan Merritt był wstrząśnięty, gdy chwyciwszy z półki wolumin podejrzenie nazwany \textit{Qanooon-e-Islam}, odkrył, iż był to w istocie zakazany \textit{Necronomicon} szalonego Araba Abdula Alhazreda, o którym słyszał takie potworne rzeczy szeptane parę lat wcześniej po kontakcie z nienazwanymi rytuałami w dziwnej, małej wiosce Kingsport w zatoce Massachusetts.

Lecz co najdziwniejsze, szlachetny dżentelman poczuł się najbardziej zniesmaczony przez malutki detalik. Na wielkim stole z z mahoniu leżała odkładką w dół znoszona kopia Borellusa, nosząca wiele tajemniczych, odręcznych zapisków stworzonych ręką Curwena. Ksiązka była otwarta mniej-więcej w połowie, i jeden z paragrafów był podkreślony tak wyraźnymi liniami, że gość nie mógł się powstrzymać i zaczął go czytać. Niezależnie od tego, czy była to natura podkreślonego akapitu, czy gorączkowa natura linii tworzących podkreślenie - pan Merritt sam nie wiedział - ale coś w tej kombinacji wpłynęło na niego w sposób znaczący i złowieszczy. Mógł przywołać ten akapit z pamięci do końca swoich dni, zapisawszy go w swoim pamiętniku i raz spróbowawszy wyrecytować go swojemu bliskiemu przyjacielowi, Dr. Checkley'owi, choć przestał, zauważywszy, jak bardzo proboszcz czuł się poruszony owym cytatem.Stanowił on, iż:

\begin{displayquote}

Esencjalne Sole Zwierzęce mogą być tak przygotowane i przechowane, by inteligentny Człowiek posiadał całą Arkę Noego w swoim Studium, i wskrzesił Kształt Zwierzęcia z jego Popiołów dla swojej własnej Przyjemności, i podobną Metodą, z esencjalnych Soli ludzkiego Pyłu, Filozof może, bez zbrodni Nekromancji, wezwać Kształt dowolnego martwego Przodka z Pyłu w który jego Ciało się obróciło.

\end{displayquote}

To jednak doki w południowej części Town Street były miejscem, gdzie krążyły najgorsze pogłoski o Josephie Curwenie. Marynarze są bardzo przesądni i doświadczeni żeglarze, którzy przywozili nieskończone ilości rumu, niewolników i słupy melasy, szelmowscy korsarze i wielkie brygi Brownsów, Crawforsów i Tilliinghastów, wszyscy czynili znaki ochronne, kiedy widzieli chudego, podejrzenie młodo wyglądającego człowieka o żółtych włosach, lekko pochylonego, gdy wchodził do magazynu na Doubloon Street lub rozmawiającego z kapitanami na długim nabrzeżu, na którym statki Curwena płynęły bez ustanku. Księgowi i kapitanowie pracujący dla Curwena nienawidzili i bali się go, a wszyscy jego marynarze byli wzięci z Martinique, St. Eustatius, Havany lub Port Rolay. W pewien sposób, to częstotliwość, z jaką ci marynarze byli zmieniani, powodem, który prowokował najbardziej materialną część lęku, który towarzyszył staremu człowiekowi. Jego załoga mogłaby się rozejść po mieście lub wybrzeżu, niektórzy z jej członków musieliby zrobić to czy tamto, a gdy ponownie by się spotkali, z calą pewnością brakowałoby im mężczyzny lub dwóch. Wiele owych rzeczy, które musieliby zrobić, dotyczyłoby farmy na Pawtuxet Road, i tak mało marynarzy kiedykolwiek wróciło z tego miejsca. Ten fakt tkwił wyraźnie w pamięci żeglarzy. W pewnym momencie przyczyniło się to do znacznego utrudnienie zamorskich biznesów Curwena. Ostatecznie paru z nich zostawiłoby statki za swoimi plecami, po usłyszeniu plotki o nabrzeżach Providence, i ich zastępcy musieli pochodzić aż z Indii Zachodnich, co stało się dużym problemem dla kupca.

W 1760 Joseph Curwen był ostatecznym wyrzutkiem, podejrzewanym o nieokreślone zbrodnie i demoniczne przymierza, które wydawały się tym straszniejsze, że nie mogły być nazwane, zrozumiane, lub choćby udowodnione, jakoby istniały. Ostatni gwóźdź do trumny był problem z zaginionymi żołnierzami z 1758, gdyż w marcu i kwietniu tego roku  dwa królewskie regimenty stacjonowały w Providence, na trasie do Nowej Francji. Żołnierze ci znikali bez śladu ponad spodziewaną liczbę dezercji. Od razu podniosły się plotki o częstotliwości, z jaką Curwen był widziany, rozmawiając z obcymi w czerwonych płaszczach, a gdy paru z nich zostało uznanych za zaginionych, ludzie wrócili myślami do dziwnych przypadłości jego własnych marynarzy Co by się stało, gdyby regimenty nie ruszyły dalej, tego nikt nie jest w stanie powiedzieć. 

W międzyczasie, materialne sprawy kupca prosperowały. Posiadał w praktyce monopol na handel saletry potasowej, czarnym pieprzem i cynamonem, i z łatwością przewodził handlowi innymi dobrami, z pierwszeństwem być może Brownsów w jego imporcie indygo, wyrobów mosiężnych, bawełnianych i wełnianych, soli, żelaza, papieru i wszelkich dóbr korony brytyjskiej. Tacy sklepikarze jak James Green z Słonia w Cheapside, Russelowie z Złotego Orła po drugiej stornie Mostu, lub Clark i Nightingale z Ryby i Patelni w pobliżu New Coffee-House, polegali prawie całkowicie na nim, by zaopatrywać się w towary, i jego biznesy z lokalnymi warzelniami alkoholu, mleczarzami i hodowcami konii z plemienia Narragansett i twórcami świec z Newport, uczyniły z niego jednego z głównych eksporterów w całej Kolonii.

Choć był on niewątpliwie ofiarą ostracyzmu społecznego, potrafił być miły, w pewien sposób. Kiedy Dom Kolonialny spalił się do fundamentów, pomógł wtedy skrzywdzonym pokaxną sumą, która pozwoliła wybudować nowy budynek, z cegieł - ciągle stojący niczym przewodniczący paradzie na starej głównej ulicy. Działo się to w 1761 r. W tym samym roku, pomógł on odbudować Wielki Most po załamaniu chmury w październiku. Odbudował on wielką część księgozbioru biblioteki publicznej, którą pochłonął pożar Domu Kolonialnego. Wsparł także loterię, która dała błotnistemu Market Parade i Town Street chodniki z wielkich, obłych kamieni, aby można było się komfortowo poruszać na piechotę. Mniej-więcej w tym samym czasie, wybudował on prosty lecz piękny nowy dom, którego drzwi były tak wielką zdobyczą sztuki płaskorzeźby. Kiedy wierni z Whitefield odseparowali się od kościoła na wzgórzu Dr. Cottona w 1743 i założyli kościół w Deacon Snow po drugiej stornie Mostu, Curwen wyruszył z nimi, choć jego wiara religijna szybko nie okazała się zbyt żarliwa. Teraz, jednak, stał się wierzący po raz kolejny, jakby próbując odrzucić cień, który wymusił na nim izolację i który wkrótce miał zacząć niszczyć jego biznesy, gdyby czegoś z tym nie zrobił. 

Widok tego dziwnego, bladego mężczyzny, wyglądającego na będącego ledwie w wieku średnim, lecz z pewnością nie młodszym niż stulecie, który zapragnął wreszcie wyłonić się z chmury lęku i zohydzenia zbyt ougólnionych, by je dookreślić lub zanalizować, był zarówno żałosny, dramatyczny i ohydny. Jednakże, taka jest moc bogactwa i powierzchownych gestów, że zaiste, objawiło się niskie obniżenie w widocznej awersji pokazywanej mu przez gmin - zwłaszcza po tym, jak nagłe zaginięcia jego żeglarzy przeminęły jak z bicza strzelił. Musiał on także zacząć stosować ekstremalną ostrożność i sekretność odnośnie swoich wypraw na cmentarz, gdyż nigdy później już go nie przyłapano na krążeniu po nim. Podobnie, pogłoski o dziwnych dźwiękach i światłach w jego posiadłości w Pawtuxet także stały się mniej częste. Jego zapotrzebowanie na żywność i bydło pozostało niewytłumaczalnie wysokie. Dopiero w czasach Charlesa Warda, którzy prześwietlił jego rachunki i dokumenty w Shepley Library, uderzyło go, jako jedynego, że można porównać wielką liczbę Czarnych z Guinei, których importował aż do 1766 z bardzo małą ich liczbą, która była oddawana handlarzom niewolników w Wielkim Moście lub do plantacji w Narragansett Country. Z pewnością, bystrość i geniusz tej okropnej, nieludzkiej osoby były oczywiste, gdy już uświadomiło się ich ogrom.

Ale oczywiście, efekt tych wszystkich zabiegów naprawczych był raczej znikomy. Curwen dalej był unikany przez swoich pobratymców - zaiste, wszyscy zazdrościli mu długowieczności. Mógł on dostrzec, że w efekcie jego bogactwa z pewnością musiały na tym ucierpieć. Jego skomplikowane studia i eksperymenty, niezależnie od tego, czego dotyczyły, najwyraźniej wymagały wielu pieniędzy, aby móc je kontynuować. Gdyby plotki wreszcie wpłynęły na jego zarobki - gdyby pozbawić go przychodu z handlu, które pozyskał, nie pomogłoby mu zaczęcia od nowa w innym miejscu. Osąd sytuacyjny wymagał, że powinien naprawić swoje relacje z mieszkańcami miasta Providence, tak, aby jego obecność już nie była powodem szeptanych konwersacji, wybitnie oczywistych wymówek nagłych obowiązków i ogólnej atmosfery powstrzymania się i niezręczności. Jego pracownicy, zredukowani obecnie do bezczynnych i zmęczonych cieni, których nikt inny by nie zatrudnił, przysparzali mu wielu problemów. Co zaś do jego kapitanów i marynarzy, trzymał ich przy sobie dzięki przebiegłości i przewagom, które pozyskał nad nimi - takiej jak weksle, hipotekę lub posiadanie informacji potrzebnych, by ich szantażować. W wielu przypadkach, twórcy pamiętników odnotowywali z niejakim zdziwieniem, że Curwen posiadał prawie magiczną moc śledzenia rodzinnych sekretów, które następnie wykorzystywał w wątpliwych celach. Podczas ostatnich 5 lat jego życia, wydawało się, że tylko bezpośrednie rozmowy z dawno zmarłymi ludźmi mogły mu zapewnić ogrom informacji, które trzymał na końcu swego języka. 

Mniej-więcej w tym samym czasie, zręczny uczony wykonał ostatnią próbę, by wrócić do łask lokalnej społeczności. Do tego momentu kompletny pustelnik, teraz zapragnął zawrzeć korzystny związek małżeński, biorać za żonę kobietę, której wysoka pozycja społeczna uczyniłaby wszelkie próby ostracyzmu względem niego niemożliwymi. Mozliwe, że miał również inne, ważniejsze powody, by zawrzeć takie małżeństwo - powody na tyle oddalone od normalnej logiki, że tylko zapiski znalezione półtora wieku później, po jego śmierci, sprawiły, że ktokolwiek mógł ich podejrzewać. O tym jednak nic pewnego nie może zostać w pełni poznane. Naturalnie, był on świadom horrorów i upokorzeń związanych z uwodzeniem, które spotkałyby jego osobę, tak więc szukał odpowiedniej kandydatki, na której rodzicach mógłby wywrzeć stosowną presję. Takie kandydatki, odkrył, nie były tak proste do odnalezienia, gdyż miał on swoje bardzo konkretne wymagania odnośnie ich piękna, osiągnięć i poważania w społeczeństwie. Jego wymagania zawęziły możliwy wybór do dom jednego z jego najlepszych i najstarszych kapitanów, wdowca wysokiego urodzenia i nieposzlakowanej opinii imieniem Dutie Tillinghast, którego jedyna córka Eliza posiadała wszelkie możliwe przewagi, z wyjatkiem zostania dziedziczką rodu. Kapitan Tillinghast był kompletnie zdominowany przez Curwena - i zgodził się, po koszmarnym wypytywaniu w swoim kopulastym domu na wzgórzu Power's Lane, pobłogosławić ten bluźnierczy mezalians. 

Eliza Tillinghast miała w tamtym czasie 18 lat i została wychowana tak łagodnie, jak pozwoliły na to warunki w skromnym domu jej ojca. Chodziła do szkoły Stephena Jacksona po drugiej stronie Court House Parade i była uczona przez swoją matkę, zanim ta umarła na ospę w 1757 r., odnośnie wszelkich sztuk domowego żywota. Przykład jej kunsztu, stworzony w 1753 r. można dalej znaleźć w pomieszczeniach Towarzystwa Historycznego Rhode Island. Po śmierci jej motki zajmowała się ona domem, wspierana tylko przez jedną starą, czarnoskórą kobietę. Jej kłótnie z jej ojcem odnośnie propozycji malżeńskiej Curwena musiały być zaiste burzliwe, lecz nie mamy o nich żadnych świadectw. Jest jednak pewne, że jej narzeczeństwo względem młodego Ezry Weedena, drugiego oficera na statku Enterprise pod dowództwem Crawforda, zostało odwołane, a jej związek z Josephem Curwenem wszedł w życie siódmego marca 1763 r. Miało to miejscu w kościele Baptystów, w obecności najznamienitszych gości, jakich tylko miasto mogło zgromadzić, ceremonii zaś przewodził młody Samuel Winson. The Gazette wspomniała o zawarciu małżeństwa w paru zdaniach, a w większości kopii, które przetrwały, ta krótka wzmianka była wycięta lub wyrwana. Ward odnalazł jedną jedyną kompletną kopię po wielu godzinach poszukiwać w archiwach prywatnego kolekcjonera, rozbawiony bezsensownym miejskim językiem wzmianki:

\begin{displayquote}

W poniedziałek wieczorem, Pan Joseph Curwen, z naszego Miasta, Kupiec, ożenił się z Panną Elizą Tillinghast, Córką Kapitana Dutie'go Tillinghasta, młodą panienką prawdziwej Wartości, dodanej do Piękna jej Osoby, co uświętni Stan Małżeński i pomnoży jego Szczęście.

\end{displayquote}

Kolekcja listów Durfee-Arnolda, odkryta przez Charlesa Warda tuż przed jego pierwszym załamaniem nerwowym w prywatnej kolekcji Melville'a F. Petersa z George Street, i obejmująca tamten i poniekąd wcześniejszy okres, rzuca wyraźne światło na publiczne oburzenie na ten źle dobrany związek małżeński. Społeczne wpływy Tillinghastów, jednakże, były niezaprzeczalne, i p oraz kolejny Joseph Curwen odkrył, że jego dom stał się odwiedzany przez osoby, których w innych okolicznościach w ogóle nie przepuściłby przez swój próg.  Jego akceptacja tego faktu nie była wszak kompletna, a jego żona cierpiała społecznie ze względu na wymuszone przedsięwzięcie, ale poprzez kolejne wydarzenia udało im się uniknąć ostracyzmy społecznego. W jego traktowaniu swojej żony, dziwny pan młody zaskoczył zarówno nią, jak i szerszą społeczność, poprzez ukazywanie ekstremalnego taktu i wdzięczności. Nowy dom w Olney Court był teraz zupełnie wolny od dziwnych manifestacji, a choć Curwen w dużej mierze porzucił farmę w Pawtuxet, której jego żona nigdy nie odwiedzała, wyglądał teraz bardziej jak normalny człowiek niż w jakimkolwiek innym momencie swojego wcześniejszego, długiego życia. Tylko jedna osoba pozostała z nim w otwartej wrogości - wspomniany młody oficer, z którym zaręczyny Eliza Tillinghast zerwała. Ezra Weeden całkiem otwarcie poprzysiągł zemstę Curwenowi i, choć zazwyczaj był łagodnego usposobienia i manier, nabierał teraz nienawistnego poczucia celu, które nie wróżyło dobrze mężowi-uzurpatorowi.

Siódmego dnia maja 1765 roku, urodziło się jedyne dziecko Curwena, Ann, ochrzczona przez Wielebnego Johna Gravesa z King's Church, do którego zarówno moąż, jak i żona zapisali się wkrótce po ślubie, jako kompromis pomiędzy ich dwoma wcześniejszymi wspólnotami religijnymi - Congregationalnej i Baptystów. Zapisy tych narodzin, jak i ślubu 2 lata wcześniej, zostały usunięte z większości kopii kościelnych i dokumentów miejskich, gdzie powinny były się pojawić. Charles Ward zdołał odnalexć obydwa z wielką trudnością po jego odkryciu, iż wdowa zmieniła swe imię, co poinformowało go o jego własnej relacji z nim, i zagroziło gorączkowymi poszukiwaniami, które miały swoją kulminację w jego szaleństwie. Zapiski o porodzie, zaiste, zostały odkryte bardzo ciekawe poprzez korespondencję ze spadkobiercami lojalisty Dr. Gravesa, który zabrał ze sobą duplikaty zapisków kiedy opuścił miasto na początku Rewolucji. Ward spróbował zasięgnąć informacji u tego źródła, będąc w pełni świadomym, że jego pra-pra-babka, Ann Tillinghast Potter, należała do wspólnoty Episkopalnej.

Wkrótce po narodzinach swojej córki, wydarzeniu, które świętował z przytupem odmiennym od swojej zwykłej oziębłości, Curwen postanowił zapozować do portretu. Został on namalowany przez bardzo utalentowanego Szkota imieniem Cosmo Alexander, które w tamtym czasie zamieszkiwał Newport, słynny od tamtego czasu jako nauczyciel Gilberta Stuarta. Podobiezna, mówili ludzie, została wykonana na drewnianym panelu ściennym w bibliotece domu w Olney Court, lecz żaden z dwóch pamiętników, które o nim wspominały, nie informował o jego ostatecznym losie. W tamtym okresie nasz kapryśny uczony wykazywał oznaki nienaturalnej abstrakcji, i spędzał tak wiele czasu, jak to tylko możliwe na swojej farmie przy Pawtuxet Road. Zdawał się on, było mówione, być w stanie ukrywanego podniecenia lub oczekiwania; tak jakby spodziewał się jakiejś fenomenalnej rzeczy lub był w przededniu ważnego odkrycia. Zdawało się, że chemia lub alchemia odgrywa w tym wielką rolę, gdyż zabierał on ze swojego domu do farmy wielką liczbę woluminów w tych właśnie tematach.

Jgo zainteresowanie sprawami miasta się nie zmniejszyło, i nie tracił żadnej okazji, by pomóc liderom pokroju Stephena Hopkinsa, Josepha Browna lub Benjamina Westa w ich wysiłkach, by ubogacić kulturowo miasto, które w tamtych czasach było znacznie poniżej Newport jeśli chodzi o patronat nad sztukami wyzwolonymi. Pomógł on Danielowi Jenckesowi otworzyć jego księgarnię w 1763, i był później jej najlepszym klientem - pomógł także Gazette w kłopotach, ukazującej się każdej środy w Pod Głowa Szekspira. W polityce, żarliwie wspierał gubernatora Hopkinsa przeciwko Wardom, którzy głównie siedzieli w Newport, a jego naprawdę elokwentna przemowa w Hacher's Hall w 1765 przeciwko separacji Północnego Prowidence jako osobnego miasta, stanowiła główny z powodów, dla których spadły  mieście uprzedzenia względem niego. Ale Ezra Weeden, który obserwował go z bliska i uważnie, prychał cynicznie pod nosem na jego wszelkie aktywności życia publicznego, i jawnie przysięgał, że było to nic więcej jak ledwie maska, skrywająca pod sobą najczarniejszą z istot Tartaru. Mściwy młodzianin rozpoczął systematyczne studium mężczyzny i jego działań, kiedykolwiek był w porcie, spędzając godziny w nocy razem z łowcami wielorybów, w oczekiwaniu na światła z posiadłości Curwena, i podążając za małą łódeczką, która czasami usuwała się po cichu z zatoki. Trzymał on też rękę na pulsie jeśli chodzi o farmę Pawtuxet, co kiedyś zakończyło się poważnym pogryzieniem przez psy, które stara para Indian puściła na niego. 

\begin{center}
2
\end{center}

Do jesieni roku 1770 Weeden zadecydował, że należy działać nagle i zwrócił na siebie uwagę ciekawskich mieszczan, gdyż w powietrzu unosił się aromat niespodziewanych, nagłych wydarzeń, zrzucony niczym stary płaszcz, dając miejsce trudno ukrywanemu wyniesieniu perfekcyjnego triumfu. Curwen wydawał się mieć problem z powstrzymaniem siebie samego przed publicznymi przemówieniami o tym, co on odkrył lub czego się dowiedział, lecz najwyraźniej, potrzeba utrzymanie owej rzeczy w tajemnicy była większa niż jego pragnienie, ażeby się pochwalić odkryciem, gdyż nigdy nie zaoferował on żadnego wyjaśnienia. To po tych wydarzeniach, które najpewniej miały miejsce we wczesnym lipcu, złowróżebny badacz zaczął zaskakiwać ludzi posiadaniem informacji, które tylko ich dawno zmarli przodkowie mogli posiadać.

Lecz sekretne aktywności Curwena w żadnym wypadku nie zostały wstrzymane wraz z tą zmianą. Wprost przeciwnie, raczej się one zwiększyły, tak, że więcej i więcej jego biznesów pozostawało w rękach kapitanów statków, którzy byli teraz z nim związani zarówno więzami strachu, równie potężnymi jak i wizja bankructwa. Porzucił on w zupełności handel niewolnikami, twierdząc, że przychody z niego ciągle się zmniejszały. Każdą wolną chwilę spędzał on na farmie w Pawtuxet, choć tu i ówdzie pojawiały się pogłoski o jego obecności w miejscach, które, choć nei były technicznie w pobliżu cmentarzy, to były z nimi powiązane na tyle mocno, iż ludzie zaczęli się dziwić, czy aby stary kupiec nie powrócił do swoich starych nawyków - a może po prostu nigdy z nimi nie zerwał? Ezra Weeden, choć szpiegował go w krótkich okresach, ze względu na swoje morskie podróże, śledził go z mściwą wytrzymałością, której większość mieszczan i rolników nie posiadało, i badał sprawy Curwena z dokładnością, z którą nigdy wcześniej nie były one badane. 

Wiele dziwacznych manewrów statków kupca można było wytłumaczyć niespokojnością owych czasów, w których każdy kolonista starał się oprzeć założeniom Sugar Act, co skutkowało problemami w transporcie. Smuglowanie i unikanie zadomowiły się dobrze w Narragansett Bay, a nocne cumowania nielegalnych ładunków były bardzo upowszechnione. Ale Weeden, noc po nocy, podążał za światłami statków i małych łodzi, które wracały od magazynów Curwena przy docu Town Street, wkrótce też zdał sobie sprawę z tego, że nie tylko statki Jego Królewskiej Mości były tym, czego chciał uniknąć Curwen. Przed zmianą z 1766, te łodzie najczęściej zbierały skutych Murzynów, którzy byli transportowani poprzez zatokę i lądowali w mało znanym miejscu tuż na północ od Pawtuxet, skąd przemieszczali się do farmy Curwena, gdzie ich zamykano w ogromnym kamiennym budynku, który posiadał wąskie szczeliny zamiast okien. Potem jednak, cały program zmieniono. Zaprzestano importu niewolników, i przez pewien czas Curwen osierocił swe transporty o północy. Wtedy, około wiosny 1767, nowa polityka weszła w życie. Po raz kolejny jego statki opuściły ciche, ciemne doki, i tym razem opuściły zatokę, unosząc się na morzu w pewnym dystansie, może nawet na wysokości Nanquit Point, gdzie miały się spotkać z dziwnymi statki wielkiego rozmiaru i różnorodnego wyglądu, ażeby odebrać od nich ładunki. Następnie, marynarze Curwena mieli zdeponować te ładunki w zwykłym miejscu na wybrzeżu, i przetransportować je lądem do jego farmy, zamykając je w tym samym tajemniczym budynku z kamienia, w którym wcześniej przebywali Murzyni. Ładunki te prawie w całości składały się z pudeł i pudełek, z których duża część była wielka i ciężka, oraz zaskakująco przypominająca wyglądem trumny.

Weeden zawsze obserwował farmę z nieustanną pilnością, odwiedzając ją każdej nocy przez dłuższy czas, i rzadko kiedy pozwalając, by minął tydzień bez jego obserwacji, z wyjątkiem dni, gdy grunt był pokryty śniegiem ukazującym ślady piechura. Nawet wtedy jednak, często podchodził tak blisko, jak to tylko możliwe, korzystając z dobrze uczęszczanych dróg lub lodu na pobliskiej rzece, by patrzeć na ślady, które inni mogli zostawić. Odkrywszy, że jego własne śledztwo jest na drodze jego morskich obowiązków, zatrudnił on lokalnego bywalca karczmy, imieniem Eleazar Smith, aby kontynuował badania, gdy zleceniodawca przebywał daleko. Obydwoje mogliby puścić w głos pewne niezwykle dzikie pogłoski. Nie zrobili tego tylko i wyłącznie dlatego, że znali efekt, jaki rozgłos mógłby wywołać na ich ofierze i uczynić dalszy postęp śledztwa niemożliwym. Zamiast tego, chcieli się dowiedzieć czegoś konkretnego zanim by zaczęli działać. To, czego się dowiedzieli, musiało zaiste być przerażające, i Charles Ward mówił wiele razy swoim rodzicom o swoim żalu z powodu tego, że Weeden później spalił swoje notatniki. Wszystko, co można powiedzieć o ich odkryciach to to, co Eleazar Smith zanotował w swoim niezbyt rozsądnym dzienniku,  i co inni pisarze pamiętników i listów skromnie powtarzali, gdy już pewne słowa się rozniosły - i zgodnie z którymi farma była tylko zewnętrzną skorupą jakiejś szeroko rozpowszechnionej i obrzydliwej okropności, której zasięg i głębia była zbyt ważna i nieuchwytna, by pozyskać coś więcej niż iluzję zrozumienia. 

Udało się potwierdzić, że Weeden i Smith byli przekonani, że obszerna sieć tuneli i katakumb znajdowała się pod farmą, i była zamieszkana przez rozlicznych ludzi, poza starą parą indian. Dom ten był starym reliktem środka XVII wieku z pokaźnym kominem i oknami z kratami w kształcie diamentów, a laboratorium wysuwało się ku północy, w miejscu, gdzie grunt prawie stykał się z dachem. Ten budynek zdawał się przypominać każdy inny, lecz zważywszy na różne głoszy słyszana czasami w jgo wnętrzu, z pewnością musiał do niego prowadzić sekretny, podziemny korytarz. Te głosy, przed 1766, były ledwie szeptami i majakami Murzynów, niekiedy przechodzącymi w paniczne wrzaski, przemieszane z intrygującymi inwokacjami i modłami. PO tej dacie, jednakże, przyjęły one formę bardzo dziwnego i okropnego dźwięku, mieszaniny tępego przyzwolenia i wybuchów szaleńczej furii, wraz z konwersacjami i sapaniami oraz okrzykami protestu. Wydawało się, że mówione były różne języki, wszystkie znane Curwenowi, których akcenty były często nierozróżnialne - odpowiedzi, potwierdzenia czy groźby?

Czasami zdawało się, że parę osób musi być w domu: Curwen, pewni pojmani i strażnicy owych pojmanych. Były ten głosy, których ani Weeden, ani Smith nigdy wcześniej nie słyszeli pomimo ich rozległej wiedzy o obcych portach, i wielu zdawało się należeć do tej czy innej narodowości. Natura tych konwersacja zawsze zdawała się być rodzajem katechizmu, tak jakby Curwen odzyskiwał pewne informacje od przerażonych lub buntowniczych więźniów. 

Weeden posiadał wiele dosłownych raportów z podsłuchanych fragmentów wypowiedzi w swoich notatkach, mówionych po angielsku, francusku i hiszpańsku, które to języki znał i które były często używane - lecz z owych notatek nic się nie ostało. Jednakże, powiedział on, że paro paroma grobowymi dialogami dotyczącymi przeszłych wydarzeń w życiach rodzin z Providence, większość pytań i odpowiedzi, które mógł zrozumieć, było historycznych lub naukowych; czasami dotyczyły one bardzo odległych miejsc lub czasów. Przy jednak okazji, przykładowo, na zmianę wściekła i pokorna osoba były przesłuchiwana po francusku o masakrę Czarnego Księcia w Limoges w 1370, zupełnie jakby był w tej historii jakiś sekret, który powinna ona znać. Curwen zapytał więźnia - jeśli był on więźniem - o to, czy rozkaz zabicia ył wydany ze względu na Znak Kozy znaleziony na ołtarzu w antycznej rzymskiej krypcie pod katedrą, lub czy Czarny Człowiek z kowenu Haute w Wiedniu wymówił Trzy Słowa. Nie mogąc uzyskać odpowiedzi, inkwizytor najwyraźniej posunął się do ekstremalnych środków, gdyż rozległ się przerażający wrzask, a następnie cisza, szeptanie i odgłosy uderzania.

Żadne z tych rozmów nie zostały nigdy wizualnie potwierdzone, gdyż zasłony w oknach były zawsze ciężko zasunięte. Jednak raz, podczas rozmowy w nieznanym języku, można było dostrzec zza zasłon cień w oknie, który przeraził Weedera znacząco. Przypominał mu on o marionetkach w teatrzyku dla lalek, który zaobserwował na jesień 1764 w Hacher's Hall, kiedy mężczyzna z Germantown w Pensylwanii zaprezentował sprytny mechaniczny spektakl reklamowany jako ``Widok na Słynne Miasto Jerozolimę,, w którym można dostrzec Jerozolimę, Świątynię Salomona, Królewski Tron, ważne Wieże i Wzgórza, jak i Cierpienie Naszego Zbawcy z Ogrodu w Gethsemane na Krzyżu na wzgórzu Golgota, prawdziwie artystyczny Pomnik. Warte zobaczenia przez Ciekawych Świata''. To przy tamtej sposobności podsłuchiwacz, który zakradł się blisko okna z przodu, gdzie miała miejsce rozmowa, został przerażony przez parę starych Indian, którzy puścili na niego swoje psy. Po tym wydarzeniu nie było już słychać bluźnierczych konwersacji w domu, a Weeden i Smith uznali, że Curwen musiał przenieść swoje operacje w podziemne regiony domostwa.

To, że takie regiony musiały po prawdzie istnieć, wydawało się oczywiste z wielu powodów/ Odległe krzyki i jęki dochodziły tu i ówdzie ze ,zdawałoby się, solidniej ziemi pod stopami, w miejscach odległych od jakichkolwiek budowli. Jednocześnie, ukryta pośród krzaków przy rzece z tyłu, gdzie wysoki grunt przechodził w dolinę Pawtuxet, znajdowała się drewniana brama, którą, w momencie odkrycia, uznano za będącą wejściem do jaskiń pod wzgórzem. Kiedy i jak owe katakumby zostały wzniesione, Weeden nie mógł powiedzieć, ale często wskazywał na to, jak łatwo owe miejsce mogłoby zostać dosięgnięte przez bandy niewidzialnych pracowników rzecznych. Joseph Curwen pokierował swoich morskich przyjaciół do rożnych celów, zaiste! Podczas ciężkich deszczy roku 1769 dwójka obserwatorów trzymała oko na rzece by sprawdzić, czy jakiekolwiek podziemne sekrety ujrzą światło dzienne, i nagrodzono ich, gdy zobaczyli zarówno kości zwierzęce, jak i ludzkie na brzegu rzeki. Oczywiście, mogło to mieć wiele wyjaśnień, w pobliżu farmy i starego cmentarza Indiańskiego, ale Weeden i Smith doszli do swoich własnych konkluzji. 

It was in January 1770, whilst Weeden and Smith were still debating vainly on what, if anything, to think or do about the whole bewildering business, that the incident of the Fortaleza occurred. Exasperated by the burning of the revenue sloop Liberty at Newport during the previous summer, the custom fleet under Admiral Wallace had adopted an increased vigilance concerning strange vessels; and on this occasion His Majesty's armed schooner Cygnet, under Captain Harry Leshe, captured, after a short pursuit one early morning, the scow Fortaleza of Barcelona, Spain, under Captain Manuel Arruda, bound according to its log from Grand Cairo, Egypt, to Providence. When searched for contraband material, this ship revealed the astonishing fact that its cargo consisted exclusively of Egyptian mummies, consigned to "Sailor A. B. C.", who would come to remove his goods in a lighter just off Nanquit Point and whose identity Captain Arruda felt himself in honour bound not to reveal. The Vice-Admiralty Court at Newport, at a loss what to do in view of the non-contraband nature of the cargo on the one hand and of the unlawful secrecy of the entry on the other hand, compromised on Collector Robinson's recommendation by freeing the ship but forbidding it a port in Rhode Island waters. There were later rumours of its having been seen in Boston Harbour, though it never openly entered the Port of Boston.

This extraordinary incident did not fail of wide remark in Providence and there were not many who doubted the existence of some connection between the cargo of mummies and the sinister Joseph Curwen. His exotic studies and his curious chemical importations being common knowledge, and his fondness for graveyards being common suspicion; it did not take much imagination to link him with a freakish importation which could not conceivably have been destined for anyone else in the town. As if conscious of this natural belief, Curwen took care to speak casually on several occasions of the chemical value of the balsams found in mummies; thinking perhaps that he might make the affair seem less unnatural, yet stopping just short of admitting his participation. Weeden and Smith, of course, felt no doubt whatsoever of the significance of the thing; and indulged in the wildest theories concerning Curwen and his monstrous labours.

The following spring, like that of the year before, had heavy rains; and the watchers kept careful track of the river-bank behind the Curwen farm. Large sections were washed away, and a certain number of bones discovered; but no glimpse was afforded of any actual subterranean chambers or burrows. Something was rumoured, however, at the village of Pawtuxet about a mile below, where the river flows in falls over a rocky terrace to join the placid landlocked cover. There, where quaint old cottages climbed the hill from the rustic bridge, and fishing-smacks lay anchored at their sleepy docks, a vague report went round of things that were floating down the river and flashing into sight for a minute as they went over the falls. Of course the Pawtuxet is a long river which winds through many settled regions abounding in graveyards, and of course the spring rains had been very heavy; but the fisherfolk about the bridge did not like the wild way that one of the things stared as it shot down to the still water below, or the way that another half cried out, although its condition had greatly departed from that of objects which normally cry out. That rumour sent Smith—for Weeden was just then at sea—in haste to the river-bank behind the farm; where surely enough there remained the evidences of an extensive cave-in. There was, however, no trace of a passage into the steep bank; for the miniature avalanche had left behind a solid wall of mixed earth and shrubbery from aloft. Smith went to the extent of some experimental digging, but was deterred by lack of success—or perhaps by fear of possible success. It is interesting to speculate on what the persistent and revengeful Weeden would have done had he been ashore at the time.