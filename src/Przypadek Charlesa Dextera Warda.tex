\part{Powieść}

\chapter{Przypadek Charlesa Dextera Warda}\index{Dzieła!Przypadek Charlesa Dextera Warda}\index{Osoby!Charles Dexter Ward}\index{Osoby!Dr. Willett}

\section{Rozdział Pierwszy: Wynik i prolog}

\begin{center}
1
\end{center}

Z prywatnego szpitala dla chorych psychicznie w pobliżu Providence, Rhode Island, ostatnio zaginęła pewna szczególna osoba. Nosiła ona imię Charlesa Dextera Warda, i została umieszczona w zakładzie z wielkim żalem przez rozpaczającego ojca, któremu przyszło patrzeć, jak aberracja wzrasta ze zwykłego ekscentryzmu w mroczną manię zawierającą w sobie zarówno możliwość morderczych tendencji, jak i specyficznej zmiany w zawartości umysłu Charlesa. Psychiatrzy byli dosyć zaskoczeni tym przypadkiem, gdyż prezentował on dziwactwa zarówno psychologicznej, jak i fizjologicznej natury

Po pierwsze, pacjent wydawał się dziwnie starszy niż swoje 26 lat. Psychiczne rany, jeśli są prawdziwe, mogą postarzyć znacząco, lecz twarz tego młodego człowieka zawierała subtelne cechy, które tylko starzy ludzie normalnie uzyskują. Po drugie, jego organiczne procesy wykazywały pewne dziwne proporcje, bez porównania z doświadczeniem medycznym doktorów. Oddychanie i praca serca posiadały zaskakujący brak symetrii, głos jego został utracony, tak, żem ógł on jedynie szeptać, trawienie trwało zaskakująco długo, w dodatku odbywało się w stopniu minimalnym, a reakcje neuronów na standardowe bodźce były nieporównywalne z wcześniejszą wiedzą medyczną, o organizmach zdrowych czy też chorych. Skóra Charlesa była sucha i o szarawym, nieświeżym odcieniu, a struktura komórkowa jego tkanek wydawała się być bardzo uszkodzona i ledwo się trzymająca razem. Nawet wielkie znamię w kształcie oliwki na jego prawym biodrze zaniknęło. W tym samym czasie, na jego klatce piersiowej uformował się wielki pieprzyk lub czarny punkt, którego tam wcześniej nie było. Ogólnie rzecz ujmując, wszyscy doktorzy zgadzali się ze sobą, że Ward posiadał metabolizm, który został uszkodzony w stopniu wykraczającym poza wcześniejsze precedensy.

Psychologicznie Charles Ward także był wyjątkowy. Jego szaleństwo nie miało sobie równych w zapiskach nawet najnowszych i najdokładniejszych rozpraw naukowych i encyklopedii medycznych, i towarzyszyła mu moc mentalna która mogłaby z niego uczynić geniusza lub przywódcę, gdyby nie wykrzywiono jej w dziwną i groteskową formę. Dr. Willett, który był lekarzem rodzinnym Wardów, potwierdza, że czyste zdolności psychiczne pacjenta, mierzalne poprzez jego reakcje na świat poza sferą jego szaleństwa, w zasadzie wzrosły od momentu pierwszego ataku. Ward, jest to prawdą, był zawsze uczonym i bibliofilem, ale nawet jego najgenialniejsze z wczesnych prac nie pokazywały jego jasności intelektu i przenikliwości, którymi sie wykazywał podczas swoich rozmów z alienistami. 