\part{Powieść}

\chapter{Przypadek Charlesa Dextera Warda}\index{Dzieła!Przypadek Charlesa Dextera Warda}\index{Osoby!Charles Dexter Ward}\index{Osoby!Dr. Willett}\index{Osoby!Dr. Waite}\index{Osoby! Dr. Lyman}\index{Osoby!Joseph Curwen}\index{Osoby!Jabez Bowen}

\section{Rozdział Pierwszy: Wynik i prolog}

\begin{center}
1
\end{center}

Z prywatnego szpitala dla chorych psychicznie w pobliżu Providence, Rhode Island, ostatnio zaginęła pewna szczególna osoba. Nosiła ona imię Charlesa Dextera Warda, i została umieszczona w zakładzie z wielkim żalem przez rozpaczającego ojca, któremu przyszło patrzeć, jak aberracja wzrasta ze zwykłego ekscentryzmu w mroczną manię zawierającą w sobie zarówno możliwość morderczych tendencji, jak i specyficznej zmiany w zawartości umysłu Charlesa. Psychiatrzy byli dosyć zaskoczeni tym przypadkiem, gdyż prezentował on dziwactwa zarówno psychologicznej, jak i fizjologicznej natury

Po pierwsze, pacjent wydawał się dziwnie starszy niż swoje 26 lat. Psychiczne rany, jeśli są prawdziwe, mogą postarzyć znacząco, lecz twarz tego młodego człowieka zawierała subtelne cechy, które tylko starzy ludzie normalnie uzyskują. Po drugie, jego organiczne procesy wykazywały pewne dziwne proporcje, bez porównania z doświadczeniem medycznym doktorów. Oddychanie i praca serca posiadały zaskakujący brak symetrii, głos jego został utracony, tak, żem ógł on jedynie szeptać, trawienie trwało zaskakująco długo, w dodatku odbywało się w stopniu minimalnym, a reakcje neuronów na standardowe bodźce były nieporównywalne z wcześniejszą wiedzą medyczną, o organizmach zdrowych czy też chorych. Skóra Charlesa była sucha i o szarawym, nieświeżym odcieniu, a struktura komórkowa jego tkanek wydawała się być bardzo uszkodzona i ledwo się trzymająca razem. Nawet wielkie znamię w kształcie oliwki na jego prawym biodrze zaniknęło. W tym samym czasie, na jego klatce piersiowej uformował się wielki pieprzyk lub czarny punkt, którego tam wcześniej nie było. Ogólnie rzecz ujmując, wszyscy doktorzy zgadzali się ze sobą, że Ward posiadał metabolizm, który został uszkodzony w stopniu wykraczającym poza wcześniejsze precedensy.

Psychologicznie Charles Ward także był wyjątkowy. Jego szaleństwo nie miało sobie równych w zapiskach nawet najnowszych i najdokładniejszych rozpraw naukowych i encyklopedii medycznych, i towarzyszyła mu moc mentalna która mogłaby z niego uczynić geniusza lub przywódcę, gdyby nie wykrzywiono jej w dziwną i groteskową formę. Dr. Willett, który był lekarzem rodzinnym Wardów, potwierdza, że czyste zdolności psychiczne pacjenta, mierzalne poprzez jego reakcje na świat poza sferą jego szaleństwa, w zasadzie wzrosły od momentu pierwszego ataku. Ward, jest to prawdą, był zawsze uczonym i bibliofilem, ale nawet jego najgenialniejsze z wczesnych prac nie pokazywały jego jasności intelektu i przenikliwości, którymi sie wykazywał podczas swoich rozmów z alienistami. Zaiste, było trudną rzeczą pozyskanie prawniczej zgody na przyjęcie go do szpitala, tak potężny i jasny zdawał się umysł tego młodego człowieka - tylko przy pomocy silnych dowodów i wielu anormalnym dziurom w jego zasobie informacji w kontraście do jego inteligencji, został on ostatecznie przyjęty do ośrodka. Aż do momentu jego zniknięcia był omni-czytelnikiem i tak dobrym rozmówcą, jak jego biedny głos tylko zezwolił. Uważni obserwatorzy, nie mogąc przewidzieć jego ucieczki, przewidzieli jednak, że nie minie dużo czasu, nim zostanie zwolniony ze szpitala.

\begin{center}
2
\end{center}

Tylko Dr. Willett, który odebrał poród Charlesa Warda i opiekował się wzrostem jego ciała i umysłu od tamtego czasu, wydawał się być przerażony na myśl o jego potencjalnej wolności. Miał on okropne doświadczenie i dokonał straszliwego odkrycia, którego nie ważył się wyjawić swym kolegom-sceptykom. Willett, zaiste, jest pomniejszą tajemnicą na swoich własnych zasadach w połączeniu z tym przypadkiem. Był ostatnim, który widział swego pacjenta przed jego ucieczką, i wyszedł z owej ostatniej konwersacji w stanie będącym mieszanką przerażenia i ulgi, co niektórzy wspominali, gdy ucieczka Warda stała się znane 3 godziny później. Ta ucieczka jest jedną z nierozwiązanych tajemnic szpitala Dr. Waite'a. Okno otwarte wysoko, na wysokości sześciu stóp raczej nie oferowało wyjaśnienia, lecz dalej, po tamtej rozmowie z Willettem młodzieniec niewątpliwie zniknął. Willett nie miał żadnego publicznego wyjaśnienia do zaoferowania, choć wydawał się by ć dziwnie zrelaksowany w stosunku do czasu sprzed ucieczki. Wielu, owszem, czuło, że chciałby powiedzieć coś więcej, lecz milczał z obawy przed niewiarą. Odnalazł on Warda w jego domu, lecz krótko po jego odejściu pielęgniarze pukali na nic. Kiedy otworzyli drzwi, pacjenta już tam nie było, a wszystko, co odkryli, to otwarte okno, przez które wiał chłodny, kwietniowy wietrzyk, rozsiewając wokół chmurę błękitno-szarawego pyłu, który prawie ich zadusił. Owszem, psy wyły jakiś czas wcześniej, ale to było wtedy, gdy Willett dalej rozmawiał z pacjentem, a nikt nie znalazł żadnego człowieka lub innej rzeczy, która wyjaśniałaby zniknięcie. Ojciec Warda został poinformowany natychmiastowo za pomocą telefonu, ale wydawało się, że jest on bardziej zasmucony niż zaskoczony. Do czasu, gdy Dr. Waite zadzwonił osobiście, Dr. WIllet już z nim zdążył porozmawiać i obydwoje nie wiedzieli nic o ucieczce lub pomocy w niej. Tylko z pewnych bardzo sekretnych przyjaciół Willetta i starszego Warda pozyskano pewne wskazówki, a nawet one były zbyt fantastyczne, by brać je poważnie. Jedyny fakt który nam pozostaje to to, że na czas obecny nie ma żadnego śladu uciekającego szaleńca.

Charles Ward był antykwariuszem od swoich lat dziecięcych, niewątpliwie pozyskując swój gust ze starego miasta wokół niego i z reliktów przeszłości, które wypełniały każdy kąt starej willi jego rodziców przy Prospect Street na wzgórzu. Wraz z latami, jego oddanie antycznym przedmiotom tylko się zwiększyło, tak, że historia, genealogia i studiowanie kolonialnej architektury, mebli i rzemiosła wyrzuciły z jego umysłu wszelkie inne zajęcia. Te gusta są ważne do zapamiętania, rozważając jego szaleństwo, gdyż choć nie stanowią jego całkowitego centrum, odgrywają w nim znaczącą rolę. Luki w wiedzy, które alieniści dostrzegli były wszystkie powiązane z współczesnymi sprawami i były zakryte przez znaczącą wiedzę o sprawach dawno minionych, co wykazały przesłuchania: ktoś mógłby pomyśleć, że pacjent przeniósł się tutaj dosłownie z jednej ze starych epok poprzez jakaś osobliwą formę auto-hipnozy. Dziwną rzeczą było to, że Wad nie wydawał się już dłużej zainteresowany antykami, które znał tak dobrze. Miał, wynika, stracić wszelkie uznanie dla nich ze względu na dobrą znajomość, i jego ostatnie wysiłki było wszystkie poświęcone zdobyciu mistrzostwa w powszechnych faktach dnia codziennego, które kompletnie wyparowały z jego umysłu. To, że to kompletnie zaćmienie zdolności mentalnych nastąpiło, było czymś, co starał się on ukryć. Było to jednak oczywiste dla każdego, kto go obserwował, że jego cały program czytania i konwersacji był zdeterminowany przez życzenie pozyskania wiedzy o swoim własnym życiu i zwykłych praktycznych i kulturowych praktykach XX wieku, zgodnie z jego narodzinami w 1902 i edukacją w szkołach naszych czasów. Alieniści teraz się zastanawiają jak, z punktu widzenia jego uszkodzonej wiedzy, pacjent-uciekinier radzi sobie we współczesnym świecie; dominuje opinia, że ``zaszył się gdzieś'' do czasu, aż jego wiedza o świecie współczesnym powróci.

Początki szaleństwa Warda są kwestią dyskusyjną pośród alienistów. Dr. Lyman, eminentny autorytet z Bostonu, umieścił ją gdzieś w 1919 lub 1920, podczas ostatniego roku chłopaka w Moses Brown School, gdy nagle zwrócił się od studiów nad historią w stronę studiów okultystycznych, i odmówił wysłaniu dokumentów na uniwersytet na bazie indywidualnych badań o znacznie większym znaczeniu dla niego. To z pewnością jest powiązane z jego zmienionymi nawykami w tamtym czasie, zwłaszcza jego ciągłemu przeczesywaniu zapisków historii miejskiej i starych miejsc pochówku szukając pewnego konkretnego grobu z 1771 - grobu jego przodka imieniem Joseph Curwen, którego niektóre dokumenty znalazł ukryte w starym domu na Olnej Court, na wzgórzu Stempers, o którym jest wiadome, że Curwen go zamieszkiwał. 

Jest niewątpliwie oczywiste, że zima 1919-1920 przyniosła wielką zmianę w Wardzie, gdyż zaprzestał swoich badań antykwariusza i skupił się wyłącznie na desperackich badaniach nauk okultystycznych zarówno w domu jak i za granicą, przetykane tylko jego dziwnie stałym poszukiwaniem grobu swego przodka.

Jednakże, w tym momencie pojawia się sprzeciw Dr. Willetta, bazując swoją opinię na bliskim i ciągłym kontakcie z pacjentem, i na pewnych przerażających badaniach i odkryciach, których sam dokonał. Te badania i odkrycia odcisnęły na nim swoje piętno - jego głos sie załamuje, gdy o nich opowiada, takoż jego dłoń drży, gdy próbuje o nich pisać. Willett przyznaje, że zmiana z 1919-1920 normalnie oznaczałaby początek postępującej dekadencji, która zakończyła się okropnie smutną i niespotykaną alienację z 1928, ale wierzy osobiście, że bardziej jasny rozdział powinien zostać poczyniony. Zgadza się on, że temperament młodzieńca zawsze był podatny na choroby psychiczne, i że był nadmiernie chętny w swoich odpowiedziach na zjawiska wokół niego, odmawia on przyznania, że te wczesna zmiana oznaczała przejście od poczytalności do szaleństwa, powołując się na słowa samego Warda, że odkrył coś, co najpewniej było wspaniała i kluczowe dla historii myśli ludzkiej.

Prawdziwe szaleństwo, doktor jest pewien, nastąpiło wraz z późniejszą zmianą - po tym jak portret Curwena i jego antyczne dokumenty ujrzały światło dzienne, po tym, jak wykonano podróże do dziwnych, zagranicznych lokacji, i pewne straszliwe inwokacje zostały wyrecytowane w dziwnych i sekretnych okolicznościach, po tym, jak pewne \textit{odpowiedzi} na te inwokacje zostały jawnie wskazane, i pośpieszny list napisany w kondycji niewyjaśnionej agonii został napisany, po fali wampiryzmu i tajemniczej plotce o Pawtuxet, i po tym ,jak z pamięci pacjenta usunięte zostały wszelkie wspomnienia współczesnych czasów  przy jednoczesnym osłabieniu jego głosu i jego fizyczności subtelnie zmienionej, so już wykazano wyżej.

Dopiero w tamtym czasie, Willett wskazuje z dokładnością, koszmarne cechy stały się stalą częścią Warda, i doktor jest przerażająco pewny, że istnieje dostatecznie dużo twardych dowodów, by potwierdzić twierdzenie młodzieńca o ważnym odkryciu. Po pierwsze, dwóch pracowników wysokiej inteligencji widziało antyczne dokumenty Josepha Curwena. PO drugie, sam chłopak kiedyś pokazał mu owe dokumenty i stronę z pamiętnika Curwena, i każdy z tych dokumentów posiadał autentyczny wygląd. Znana jest lokacja dziury, w której Ward znalazł owe zapiski, a sam Willett miał bardzo przekonujący ostatni rzut oka na nie w okolicznościach, którym ciężko jest dać wiarę i których być może nigdy nie udowodnimy. Dalej, są tajemnice i zbiegi okoliczności listów Orne'a i Hutchinsona, i problem pisma odręcznego Curwena i to, co detektywi odkryli o Dr. Allenie. Te rzeczy i okropna wiadomość w średniowiecznych zapiskach odnalezionych w kieszeni Willetta po tym, jak odzyskał świadomość po swoim szokującym doświadczeniu.

Najważniejsze jednak są dwa okropne \textit{wyniki}, które doktor pozyskał z pewnej pary mistycznych formuł podczas swojego ostatecznego śledztwa - wyniki które udowodniły autentyczność dokumentów i ich potworne implikacje w tym samym czasie, gdyż te dokumenty zrodzone były z ludzkiej wiedzy.

\begin{center}
3
\end{center}

Należy spojrzeć na wcześniejsze życie Charlesa Warda jak na coś, co należy do przeszłości tak mocno, jak antyki, który sobie tam mocno ukochał. W jesieni 1918, i z pokazał wyraźnego zapału względem treningu wojskowego tamtego okresu, zaczął on swój pierwszy rok w Moses Brown School, która znajduje się nieopodal jego domu. Stary budynek główny, wzniesiony w 1819, zawsze był urokliwy dla oczu młodego antykwariusza - a rozległy park w którym mieści się Akademia oferował piękne widoki. Jego aktywności społeczne były mocno ograniczone, a swe godziny spędzał głównie w domu, na spacerach, w swoich klasach i w poszukiwaniu danych o antykach i genealogii w Urzędzie Miejskim, Domie Państwowym, bibliotece publicznej, Ateneum, Towarzystwie Historycznym, bibliotekach Uniwersytetu Brown, i nowo otwartej Bibliotece Shepley'a na Benefit Street. Można go sobie łatwo wyobrazić w tych dniach - wysokiego, chudego i o blond włosach, z uczonym okiem i lekko pochylonego, ubranego nieco niezgrabnie, i dającego wrażeniem raczej niegroźnej niezdarności niż atrakcyjności. 

Jego spacery były zawsze przygodami powiązanymi z historią, podczas których był w stanie wyobrazić sobie miliony reliktów chwalebnej historii starego miasta i połączyć je z minionymi wiekami. Jego domem była okazała posiadłość w stylu georgiańskim, na szczycie stromego wzgórza, które wznosiło się na wschód przy rzece, a z jego tylnich okiem mógł na ciasno upakowane wieżyce, stropy, dachy i drapacze chmur niższego miasta aż do purpurowych wzgórz poza miastem. To tutaj się urodził i z kochanego, klasycznego ganku fasady z podwójnie wypiekanych cegieł jego opiekunka prowadziła go w wózku, obok małej, białej farmy założonej 200 lat wcześniej i w kierunku budynków akademickich przy wystawnej ulicy, której stare, kwadratowe posiadłości z cegieł i mniejsze drewniane domy z wąskimi gankami opierającymi się na ciężkich kolumnach doryckich, były ustawione pośród przestronnych podwórek i ogrodów.

Był także prowadzony na wózku wzdłuż sennej uliczki Congdon Street, jeden poziom niżej na stromym wzgórzu, wraz z jej wschodnimi domami z wysokimi tarasami. Małe drewniane domki były najstarsze w tym miejscu, gdyż z tego właśnie wzgórza wzrastało rosnące miasteczko. To w tych podróżach przesiąkał on czymś w rodzaju koloru starej, kolonialnej wioski. Opiekunka miała w zwyczaju zatrzymać się i usiąść na ławkach Tarasu Prospekt, by porozmawiać z policjantami; jednym z pierwszym wspomnień tego dziecka był wielki, zachodni ocean dachów, kopuł i wież oraz odległe wzgórza które dostrzegł pewnego zimowego popołudnia z wielkiego wału z szynami, wszystkie fioletowe i mistyczne naprzeciwko rozgrzanego, apokaliptycznego zachodu słońca, malującego niebo czerwienią, złotem, purpurą i różnymi odcieniami zieleni. Szeroka, marmurowa kopuła Domu Stanowego wyróżniała się swoją masywną sylwetką, posąg będący jego ukoronowaniem otoczony fantastyczną aureolą poprzez szparę w jednej z stratusowych chmur, które lśniły na ognistych niebiosach. 

Gdy był starszy, rozpoczęły się jego słynne spacery; najpierw z jego niecierpliwie zaciągniętą opiekunką, a potem samotnie, w marzycielskiej medytacji. Głębiej i głębiej w ulice tego prawie pionowego wzgórza miał się zapuszczać, za każdym razem sięgając starszych poziomów tego antycznego miasta. Zatrzymywałby się niepewnie przy wejściu do Janckes Street z jej murami z tyłu i kolonialnymi szczytami sięgającymi aż do ciemnej Benefit Street, gdzie ukazywał mu się drewniany antyczny dom z parą drzwi w stylu jońskim, a obok niego był prehistoryczny dom z dachem mansardowym, gdzie zachowała się jeszcze resztka farmy, a także dom wielkiego sędziego Durfee z jego upadłymi pozostałościami stylu Georgiańskiego. To tutaj miały powstać slumsy, ale wielkie drzewa wiązów rzucały odżywczy cień na to miejsce, a chłopiec zwykł udawać się stąd na południe, wzdłuż długiej linii domów sprzed czasów Rewolucji Amerykańskiej, z wielkimi strychami i klasycznymi portalami. Po wschodniej stronie budynki te były osadzone wysoko ponad piwnicami, z podwójnymi schodami z kamiennymi stopniami i młody Charles mógł z wielką łatwością wyobrazić je sobie takie, jakie musiały być, gdy ulica ta była jeszcze młoda, gdy frontony były świeżo pomalowane, a nie widocznie zużyte, jak w dniu obecnym. 

Od strony zachodniej, wzgórze schodziło w dół prawie tak stromo jak wyżej, aż do starej ``Town Street'', którą założyciele miasta umieścili przy krawędzi rzeki w 1636 r. To tutaj mieściły się niezliczone małe pasy z pochylonymi małymi domkami wielkiej antyczności; i, choć był wielce zafascynowany, minęło wiele czasu, nim odważył się spenetrować ich archaiczny wzrost z lęku przed wejście w sen lub wrota do nieznanych horrorów. Odkrył on, iż jest znacznie mniej straszliwe spacerowanie dalej wzdłuż Benefit Street aż do żelaznych wrót ukrytego kościółka Św. Johna i tyłu Domu Kolonialnego z 1761 r. i do zbutwiałego cielska karczmy Golden Ball, gdzie niegdyś zatrzymał się Waszyngton. Na Meeting Street  - zwanej Gaol Lane i King Street w innych czasach - szukałby na wschodzie i dostrzegłby łuk schodów, wspinających się na wzgórze, a w dół, w kierunku zachodnim, dostrzegłby Szkołę Kolonialną ze starych cegieł,  śmiejącą się z antycznego znaku z głową Szekspira, gdzie \textit{Providence Gazette} i \textit{Country-Journal} były drukowane przed Rewolucją. Następnie pojawiał się przecudowny Kościół Pierwszych Baptystów z 1775 r.,, luksusowy widok z jego niepowtarzalną wierzą Gibbsa i Georgianskimi dachami i kopułami, unoszącymi się wysoko. Tutaj i w kierunku południowym sąsiedztwo stało się lepsze, w ostateczności wykwiłwszy w cudowną grupę wczesnych posiadłości; lecz dalej małe antyczne paski prowadziły w kierunku zachodnim, wydarłwszy się duchami w ich wielo-spiczastymi archaizmami, i ociekającymi do błyszczącego rozkładu, gdzie stare nadbrzeża wspominają dumnie czasy Indii Wschodnich wśród poliglotów, rozkładających się sterów, sklepów żeglarskich z zamglonymi witrynami i nazwami ulic z dawna, które przetrwały aż po dziś dzień, takimi jak Packet, Bullion, Gold, Silver, Coin, Doubloon, Sovereign, Guilder, Dollar, Dime, i Cent.

Czasami, po tym, jak urósł wyższy i bardziej żądny przygód, młody Ward ruszał głębiej w dół, ku burzy chwiejnych domów, złamanych poprzecznic, schodów, wygiętych balustrad, ciemnych twarzy i nienazwanych zapachów, ciągnących się od South Main do South Water, szukając doków, gdzie zatoka i statki parowe się dotykały, i wracał w stronę północną niższymi poziomami magazynów z 1816 o stromych dachach i szeroką drogą Wielkiego Mostu. To tutaj rynek z 1773 dalej stoi na jego antycznych łukach. Na tej szerokiej drodze zwykł się zatrzymywać, by wypić z czary piękna starego miasta gdy to wznosiło się w stronę wschodnią, pełne Georgiańskich szczytów i ukoronowane nową kopułą Christian Science tak, jak Londyn jest koronowany kopułą katedry św. Paula. Najbardziej lubił docierać do tego punktu późnym popołudniem, gdy światło słoneczne dotykało rynku i antycznych dachów domów na wzgórzu i ich dzwonnic, malując je złotem, i rzuca swą magię pośród wyśnionych nadbrzeży, gdzie niegdyś indianie z Providence zwykli zarzucać kotwice. Po dłuższym przyjrzeniu się temu widokowi, zakręciłoby mu się w głowie z poetycką miłością, i wtedy zacząłby drogę powrotną do domu,, mijając po drodze stary, biały kościół i wąskie, strome uliczki, gdzie złote błyski odbijałyby się w oknach i naświetlach umieszczonych wysoko ponad podwójnymi schodami z balustradami z kutego żelaza. 

Innymi razy, i w późniejszych latach, szukał on wyrazistych kontrastów: spędzając połowę swojego spaceru w rozpadających się kolonialnych regionach na północny wschód od swego domu, gdzie wzgórze opada na najniższy poziom do Stempers Hill z jego gettem murzyńskim\footnote{Ogólnie rzecz ujmując nie lubię ``słowa na M'' i wierzę, że nie powinno się z niego korzystać w języku polskim. Tutaj użyłem go, niejako wbrew sobie, gdyż (A) Lovecraft użył w oryginale słowa ``Negro'' i (B) wiem doskonale, że za życia posiadał on uprzedzenia rasowe. Pragnę jednak zaznaczyć, że właściwy język we współczesnej polszczyźnie to ``osoba czarnoskóra'' a nie słowo na M. (przyp. tłum.)}, gromadzącym się wokół miejsca, gdzie dyliżansy do Bostonu zwykły rozpoczynać swą podróż przed czasami Rewolucji. Drugą połowę swoich spacerów spędzał on w na południowych ziemiach wokół ulic George'a, Benevolent, Power i Williams'a, gdzie stare wzgórze jest gruntem dla pięknych włości i odgrodzonych murami ogrodów oraz zielonych ścieżek, z którymi wiąże się tyle drogich wspomnień. Te podróże, razem z pilnymi studiami, które im towarzyszyły, z pewnością przyczyniły się do wielkiej wiedzy historycznej zamieszkującej umysł Charlesa Warda. Ilustruje to także mentalną podstawę, niejaki grunt, na który padłu nasiona tej pamiętnej zimy 1919-1920, która to poruszyła wydarzenia, które miały tak dziwny i straszliwy finał.

Dr. Willett jest pewien, że do tej okropnej zimy, zainteresowania historyczne Charlesa Warda były wolne od zaburzonej psychiki. Cmentarze były wtedy dla niego bez szczególnego znaczenia, poza ich wartością historyczną, a cokolwiek w stylu przemocy lub dzikiego instynktu było poza jego psyche. Wtedy, sunąc powoli, acz nieubłaganie, pojawił się ostateczny wynik jednego z jego badań genealogicznych z poprzedniego roku; wtedy, gdy odkrył wśród swoich przodków po stronie matki pewnego długowiecznego człowieka imieniem Joseph Curwen, który przybył z Salem w marcu 1692, i o którym krążyły plotki i historie, które trudno powtarzać w dobrym towarzystwie.

Pra-pra-pradziadek Warda, Welcome Potter, w 1785 wziął ślub z pewną ``Ann Tillinghast, córką Pani Elizy oraz Kapitana Jamesa Tallinghasta'', o której rodowodzie rodzina nie posiadała żadnych szczegółowych informacji. Pod koniec 1918, podczas szukania woluminu o pierwotnych danych historycznych, młody genealog natrafił na wpis opisujący legalną zmianę nazwiska, które w 1772 pani Eliza Curwen, wdowa po Josephcie Curwenie, przyjęła razem ze swoją siedmioletnią córką Ann, której nazwiskiem panieńskim było Tillinghast, uzasadniając ową zmianę ``iż imię jej Męża stało się publiczną Obrazą dla Rozumu, bazując na wiedzy o jego Chorobie, która potwierdza pewne powszechne, antyczne Plotki, tak więc nie pragnę być znana jako jego lojalna Żona, dopóki owe plotki nie okażą się Bzdurne ponad wszelką wątpliwość''. Ten wpis ujrzał światło dzienne po przypadkowej separacji dwóch kartek, które ostrożnie sklejono razem i które były wyłączone z skądinąd poprawnej numeracji stron.

Było od razu oczywiste dla Charlesa Warda, że odkrył nieznany wcześniej sekret swojego pra-pra-pra-pradziadka. Odkrycie podnieciło go szczególnie, gdyż już słyszał ogólne raporty i widział rozsiane pogłoski o tej osobie, odnośnie której pozostało tak mało weryfikowalnych danych, pomijając te nieliczne, które ujrzały światło dzienne dopiero we współczesnych czasach, tak, że prawie zdawało się to być spiskiem, mającym na celu usunięcie go z ludzkiej pamięci. To, co jednak wydawało się płynąć z tych danych, miało tak prowokacyjną naturę, że nie dało się wyobrazić sobie, co powodowało, że owi kolonialni kronikarze byli tak pełni lęku i chętni, by ukryć i zapomnieć, lub by podejrzewać, że powody owego wykasowania były aż zanadto właściwe. 

Przed tym, Ward był wielce kontent, mogąc wyobrażać sobie starego Josepha Curwena jako coś zawieszonego w powietrzu - ale odkrywszy swoją własną relację z ową ``wymazaną'' figurą, zaczął szukać o nim danych tak systematycznie, jak to tylko możliwe. W tym ekscytującym zadaniu, ostatecznie odniósł sukces poza jego największymi spekulacjami, gdyż stare listy, pamiętniki i nieopublikowane wspomnienia znalezione w zakurzonych strychach Providence i innych miast i miasteczek były obfite w wiele oświecających zapisków, odnośnie których ich autorzy nie uznali za za stosowne, by je zniszczyć. Jedna ważna informacja pochodziła ze źródła tak odległego jak Nowy York, gdzie jakieś zapiski kolonizatora z Rhode Island przechowywane były w Muzeum Frances' Tavern. Najważniejszą rzeczą jednak, i było to coś, co zdaniem Doktora WIlleta stanowiło ostateczne źródło klęski Warda, było coś, co znalazł w sierpniu 1919 za panelami starego domu w Olney Court. To było to, ponad wszelką wątpliwość, co otworzyło przed nim owe ciemne wizje, które kończyły się daleko poza dnem piekielnym. 

\section{Rozdział Drugi: Przodek i Horror}

\begin{center}
1
\end{center}

Joseph Curwen, co wyjawiły szeptane legendy odkryte przez Warda, był niezwykłym, enigmatycznym, w tajemniczy sposób okrutnym indywiduum. Uciekł z Salem do Providence - ostatecznego schronienia dla dziwnych, wolnych i niezgadzających się z powszechnie przyjętym konsensusem - na początku wielkiej paniki wiedźm. Bał się on prześladowań ze względu na samotniczy tryb życia i dziwne chemiczne czy też alchemiczne eksperymenty. Był on osobą o szarej skórze około trzydziestki, i szybko został obywatelem Providence, zakupiwszy później dom na północ od włości Gregory'ego Dextera na wysokości Olnejh Street. Jego dom był wybudowany na Stempers Hill na zachód od Town Street, w miejscu, które później zostało Olney Court. W 1761 zamienił ten dom na większy, na tej samej ulicy, który dalej stoi w tamtym miejscu.

Warto zaznaczyć, że pierwszą dziwną rzeczą odnośnie Josepha Curwena było to, że zdawał się nie starzeć. Był zaangażowany w handel, zakupił miejsce na łódkę przy Mile-End Cove, pomógł odbudować Wielki Most w 1713 i Kościół Kongregacji na wzgórzu, ale zawsze posiadał wygląd mężczyzny, który nie miał więcej niż trzydzieści, może trzydzieści pięć lat. W miarę, jak dekady upływały jedna za drugą, ta cecha zaczęła przyciągać uwagę ludu. Curwen zawsze wyjaśniał, że pochodzi z długoletniego rodu, a za pośrednictwem prostego żywota uzyskał świetne zdrowie. Jak ową prostotę można pogodzić z niewyjaśnionymi zakupami sekretnego kupca i z dziwnymi światłami w oknach jego domu przez całą noc, nie było zbyt oczywiste dla mieszkańców miasta. Zamiast tego, postulowali one inne wyjaśnienie jego ciągłej młodości i długoletności. Istniał powszechny konsensus, że mieszanie przez Curwena tajemniczych chemikaliów było odpowiedzialne za jego stan. Plotki głosiły o dziwnych substancjach, które kupował z Londynu i Indii, transportowanych na jego statkach lub zakupionych w Newport, Bostonie lub Nowym Yorku, a gdy stary doktor Jabez Bowen przybył z Rehoboth i otworzył swoją aptekę po drugiej stronie Wielkiego Mostu, zwaną Pod Jednorożcem i Moździerzem, gorące szepty nie mogły zamilknąć o lekach, kwasach i metalach, które długowieczny odludek zakupił lub zamówił u niego. Działając w przekonaniu, iż Curwen posiadał cudowne i tajemnicze medyczne zdolności, wielu chorych ciągnęło do niego z prośbami o pomoc, lecz choć zdawał się on zachęcać ich wierzenia odnośnie samego siebie w luźny sposób, i zawwsze dawał im mikstury w dziwnych kolorach w odpowiedzi na ich prośby, zauważono, że jego dary dla innych rzadko kiedy przynosiły im korzyści. Po upływie lat pięćdziesięciu, i bez większej zmiany niż 5 lat na jego licu, szepty ludzi stały się znacznie mroczniejsze. W efekcie, stał się jeszcze większym odludkiem.

