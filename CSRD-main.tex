\documentclass[10pt, a4paper, twocolumn, openright]{book}

\usepackage[pdftex, breaklinks=true]{hyperref}
\usepackage{polski}
\usepackage[utf8]{inputenc}
\usepackage{hyperref}
\usepackage{makeidx}
\usepackage{tabularx}
\usepackage{xcolor }
\usepackage{soul}
\usepackage{afterpage}
\usepackage[T1]{fontenc}
\usepackage{csquotes}
\usepackage{longtable}
\usepackage{supertabular,booktabs}
\usepackage{pdfpages}

% TABLES WIDTH !!!
% 0.10 and 0.45

\definecolor{purple}{HTML}{92268F}
\definecolor{red}{HTML}{FF0000}
% \definecolor{gray}{HTML}{D3D3D3}

% \sethlcolor{gray} 

\newcommand{\mytext }[1] {{\color{purple} \textbf{ \texttt {#1}}}}

% \title{Cypher System Reference Document 2024-07-02 (Edycja Polska)}
% \author{Zespół Monte Cook Games\thanks{Strona projektu: \url{https://www.montecookgames.com/cypher-system-open-license/}} \and Szymon ``Kaworu'' Brycki\thanks{\href{mailto:szymon.brycki@gmail.com}{\tt szymon.brycki@gmail.com}}}

\makeindex

\begin{document}

\includepdf[pages=1, noautoscale=true, width=\paperwidth]{graphics/okładka 3.pdf}

\begin{titlepage}
	\centering
	{\Huge \bfseries \title  _Dokument Referencyjny  \break Cypher System \par}
	\vspace{1cm}
	{\large \itshape 2024-07-02 \par}
	{\large \itshape  Edycja polska \par}
	\vspace{1cm}
	{\normalsize \textbf{Oryginalne zasady}: Monte Cook Games \par}
	{\normalsize \textbf{Polskie tłumaczenie}: Szymon ``Kaworu'' Brycki \par}
	\vspace{1cm}
	{\normalsize Licencja: \bfseries Cypher System Open License\par}
	\vspace{1cm}
	{\large \textbf{Copyright © 2024 Monte Cook Games. Some rights reserved.} \par}
	\vspace{1cm}
	{\large Stworzono w technologii \LaTeX \par}
	\vspace{1cm}
	{\large \today \par}
\end{titlepage}

% \maketitle

\tableofcontents

% here go all the chapters

\input{src/Jak grać w Cypher System.tex}
\input{src/Tworzenie własnej postaci.tex}
\input{src/Typ.tex}
\input{src/Wojownik.tex}
\input{src/Adept.tex}
\input{src/Odkrywca.tex}
\input{src/Mówca.tex}
\input{src/Opcje Tworzenia Postaci - Fantasy.tex}
\input{src/Dalsza Customizacja.tex}
\input{src/Deskryptory.tex}
\input{src/Specjalizacje.tex}
\input{src/Tworzenie nowych Specjalizacji.tex}
\input{src/Zdolności.tex}

% below go the alphabetic abilities lists

\input{src/Zdolności/A (E) v2.tex}

% below go licenses

\input{src/Licencje.tex}

\printindex

\listoftables

\end{document}